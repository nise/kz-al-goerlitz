\chapter{Literatursuswahl weiterer Erinnerungs- und Gedenkorte im Dreiländereck Polen-Tschechien-Deutschland}					

[Karte]

\subsection{KZ Außenlager Groß Koschen}

\subsection{KZ Außenlager Kamenz / Herrental}

\subsection{KZ Außenlager Niesky / Wiesengrund}

\subsection{KZ Hainewalde}

\subsection{KZ Leschwitz / Görlitz}

\subsection{Sowjetisches Speziallager Bautzen}

\subsection{Katharinenhof Großhennersdorf}

\subsection{Großschweidnitz}

\subsection{Stasigefängnis Bautzen I und II}

\subsection{KZ Außenlager Brandhofen / Spohla}

\subsection{Stalag VIIIA Görlitz Moys}
\begin{itemize}
	\item Lauerwald ...
	\item Zglomobiki
\end{itemize}

\subsection{Ghetto Tormersdorf}
\begin{itemize}
	\item Pastor Curt Zitzmann: Chronik Zoar - Martinshof (1898 - 1951). \url{http://www.buss-martinshof.de/geschi-3.htm}
	\item Roland Otto: Die Verfolgung der Juden in Görlitz unter der faschistischen Diktatur 1933 -1945, S. 61ff. Stadtverwaltung Görlitz, 1990.
	\item Martinshof: Wir haben 100 Jahre Geschichte. \url{http://www.buss-martinshof.de/geschi-5.htm}
	\item Bernhard Brilling: Evakuierung der Breslauer Juden nach Tormersdorf bei Görlitz, Kreis Rothenburg, Oberlausitz, in: Mitteilungen des Verbandes ehemaliger Breslauer und Schlesier Juden in Israel, 46/47, 1980. 
	\item Abraham Ascher: A community under siege - the Jews of Breslau under Nazism, 2007.
	\item Reinhard Leue: Preisgegebene Menschen - Zwangslager und Judenghetto Zoar / Martinshof in Rothenburg 1941 / 1942. Martinshof Rothenburg, Diakoniewerk, 2004.
\end{itemize}

\subsection{Zittwerke: Kriegsgefangenenlager / KZ Außenlager Zittau}
\begin{itemize}
\item Dorota Sula, Andrea Rudorff: Zittau, in: Wolf­gang Benz / Bar­bara Die­s­tel (Hgs.): Orte des Ter­rors. Ge­schichte der na­tio­nal­so­zia­lis­ti­schen Kon­zen­tra­ti­onsla­ger Band 6. Natz­wei­ler. Groß-Rosen. Stutt­hof. S. 470-473.  Ver­lag C. H. Beck, München 2008
\item Herbert Bauer: Rund um die Historie eines Anschlussgleises. Interessenverband der Zittauer Schmalspurbahnen e.V. (Hrsg.), Oybin 2003.
\item Obóz przymusowej pracy w Sieniawce-Kleinschönau 1936-1945 (Zwangsarbeiterlager in Siniawka-Kleinschönau 1936-1945). \url{http://www.lgp.org.pl/}
\end{itemize}
