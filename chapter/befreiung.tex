
%%%%%%%%%%%%%%%%%%%%%%%%%%%%%%%%%%%%%%%%%%%%%

Während der letzten 60 Tage vor der Befreiung starben durchschnittlich mindestens zwei Menschen pro Tag. Der Grund für diese, im Vergleich zu den vorherigen Monaten, extrem hohe Sterblichkeit waren die miserablen Arbeitsbedingungen im Freien. Obwohl kurz nach der Rückkehr der Häftlinge einige sofort für den erneuten Arbeitseinsatz bei der WUMAG bestimmt wurden und fast bis Kriegsende in den Fabrikhallen arbeiteten\footnote{Janek Schilid. BStU MfS ASt 13/48 Bd. 2 / 397.}, schuftete das Gros der Gefangenen bei Schanzarbeiten sowie beim Bau von Barrikaden. Andere waren auf dem Görlitzer Flugplatz (siehe Karte S.~\pageref{map_goerlitz}) mit der Verladung von Munition beschäftigt\footnote{Harry (Hirsch) Silbering. LArchB B Rep 058 Bd. 6. Interview mit Anna Hynr\'akov\'a vom 29.03.2005 in Prag.}.
\newline
Die Art der Arbeit bestimmte abermals die Überlebenschancen. Um jene, die Panzergräben ausheben mussten, war es besonders schlecht bestellt. Bis zu den Knien versanken sie mit ihren Holzschuhen im Schnee und Matsch\footnote{Abram Rajchbart. ZIH 301/715. Siehe auch: Schloime Monczink (Sol Munn) in einem Interview mit seinem Neffen Samson Munn im Juni 1984 in den USA.}. Niederschlag und Kälte taten ihr Übriges. Hinzu kam noch die lange Wegstrecke bis in die Oststadt. Folglich stieg der Krankenstand und die Zahl derer, die vor Entkräftung aufgaben und durch die ukrainischen Hilfskräfte bei der Arbeit erschossen wurden. 
~\newline
Die ehemalige Gefangene Mila Weisberg\index{p}{Weisberg, Mila} erinnert sich, wie sie nach der Evakuierung zu Schanzarbeiten gezwungen wurde: 
\begin{leftbar} 
Um 5 Uhr treten wir zum Appell an. Um 6 Uhr gehen wir zu unserer Arbeit. Wir arbeiten sehr schwer beim Schanzengraben. Wir stehen im Schmutz bis an die Knöchel. Wir haben nur Holzschuhe, deshalb verfallen viele Leute in Krankheit. Unsere tägliche Nahrungsration besteht aus einem Liter Suppe und 20\,dag Brot [200\,g].\footnote{Mila Weisberg, ZIH 301/923.}
\end{leftbar}
Jechiel Rappaport\index{p}{Rappaport, Jechiel}:
\begin{leftbar} 
Ein Häftling namens Frenkel konnte vor Ermüdung nicht schritthalten als sie vom Ausheben der Schützengräber zurück kamen, woraufhin ihn ein ukrainischer SS-Mann erschoß. Jenseits der Neiße, nicht weit von der Arbeitsstelle wurde er an Ort und Stelle begraben. Solche Vorfälle wiederholten sich noch einige Male.\footnote{Jechiel Rappaport. LArchB B Rep 058 Bd. 2.}
\end{leftbar}

\begin{tabular}{p{.47\linewidth}p{.47\linewidth}}
\myfigure{ps2}{}{}{}{.84}	&
\myfigure{ps7}{}{}{}{.84}	\\[-12pt]
\mypics[BStU MFs Ast StKs 13/48 Bd.3.]{Panzersperre an der Luisenstraße mit Blick auf den Reichenbacher Turm}{Panzersperre auf der Luisenstraße} &
\mypics[BStU MFs Ast StKs 13/48 Bd.3.]{Panzersperre an der Einfahrt zur Lunitz, Ecke Schanze, Richtung Friedhofsverwaltung}{Panzersperre an der Einfahrt zur Lunitz}		
\label{nolabel}
\end{tabular}

\newpage

%% Fotos der von Haeftlingen errichteten Panzersperren
\myfigure{ps1}{BStU MFs Ast StKs 13/48 Bd.3.}{Panzersperre an der Promenade, heute Kahlbaumallee}{Panzersperre an der Promenade}{0}

\myfigure{ps3}{BStU MFs Ast StKs 13/48 Bd.3.}{Panzersperre an der Schanze}{Panzersperre an der Schanze}{0}

\newpage

\myfigure{ps4}{BStU MFs Ast StKs 13/48 Bd.3.}{Panzersperre auf der Jakobstraße Richtung Postplatz}{Panzersperre auf der Jakobstraße}{0}

\myfigure{ps5}{BStU MFs Ast StKs 13/48 Bd.3.}{Panzersperre an der Heiligen Grabstraße 84}{Panzersperre auf der Heiligen Grabstraße}{0}

\newpage

\myfigure{ps6}{BStU MFs Ast StKs 13/48 Bd.3.}{Panzersperre an der Helmut-von-Moltke-Straße, heute James-von-Moltke-Straße}{Panzersperre auf der Helmut-von-Moltke-Straße}{0}

\myfigure{ps8}{BStU MFs Ast StKs 13/48 Bd.3.}{Panzersperre an der Sonnenstraße Richtung Waggonbau}{Panzersperre auf der Sonnenstraße}{0}


%%%%%%%%%%%%%%%%%%%%%%%%%%%%%%%%%%%%%%%%%%%%%
\subsection{Kapitulation des KZ-Außenlagers Görlitz}

Nach Einstellung der Produktion in der WUMAG in den letzten Kriegstagen wurden auch kaum noch Arbeitskommandos für den Barrikadenbau entsandt. Die Unruhe unter den Wachleuten und der andauernde Fliegeralarm deuteten darüber hinaus an, dass \glqq irgend etwas nicht stimmte\grqq\footnote{Jakob Kinrus. RAG Sammelgut KZ Biesnitzer Grund.}. Kurz vor der Befreiung ergriffen die Wachleute die Flucht.
~\newline
In einem Interview beschreibt Anna Hyndr\'akov\'a\index{p}{Hyndr\'akov\'a, Anna} die Ereignisse wie folgt:
\begin{leftbar} 
Wir arbeiteten auf einem Flughafen. Auf einmal haben wir einen Befehl bekommen, wir sollen zurück marschieren. Wir sind zurückgekommen und haben gesehen, dass nirgends die Wachen sind. Am 6. Mai war das. Die Männer waren im Frauenlager und umgekehrt. Zunker\index{p}{Zunker, Winfried} hat zum Appell gepfiffen. Alle sollten aus den Blöcken heraus. Ich hatte meinen Mantel drinnen und wollte ihn holen, da hat er mir eine Ohrfeige gegeben... Ich konnte drei Tage nicht hören, ich dachte, ich habe mein Gehör verloren. Und er hat geschrien: \glqq Ihr glaubt wohl, ich würde euch alle einzeln herausholen\grqq. Und dann beim Appell hat er gesagt: \glqq Ihr wisst, wir haben euch nie geschlagen\grqq. Er sagte, die SS wird uns zu den Amerikanern bringen und wenn wir hier bleiben, kommen die Russen -- die werden uns vergewaltigen und die Nasen abschneiden und so weiter und so fort. Wir wollten nicht mit ihnen gehen, aber wir haben auch Angst gehabt dort zu bleiben. Man hat gesagt die Deutschen brennen die Lager ab, damit die Spuren vernichtet werden.\newline
Wir waren 12, also wir drei [tschechischen Mädchen], Stella Beck\index{p}{Beck, Stella} [Lagerälteste im Frauenlager] und ein paar Leute aus dem Männerlager. Einer von denen war ein Kutscher [Samuel Kessler\index{p}{Kessler, Samuel}]. [...] Wir haben uns verabredet, mit Zunker\index{p}{Zunker, Winfried} und den SS-Leuten mit zu gehen und dann wegzulaufen. Der Kutscher hatte so einen Heuwagen und ein Pferd. Dort auf dem Wagen war Brot, Margarine und Marmelade für die Deutschen. Eine große Menge. Damit sind wir gegangen. In der ersten Nacht hat uns Kessler\index{p}{Kessler, Samuel} alle aufgeweckt und mit dem Wagen und dem Brot sind wir weggegangen.\newline
In einer Stadt, durch die wir fuhren, waren die SS-Männer von Görlitz und die haben unsere Männer erkannt. Wir waren sehr gespannt. Sie wollten die Zivilkleider mit ihren Uniformen tauschen. Da haben wir gesagt: Ja bitte, aber wir sind verlaust und es besteht die Gefahr, dass wir mit Typhus infiziert sind. Sie ließen uns durch. Auf einer Wiese haben wir noch ein Pferd gesehen. Dann haben wir noch einen Wagen gesehen. Das Pferd war so schwach -- am Ende hatten wir zwei Wagen und zwei Pferde. So sind wir nach Prag gekommen. Dort haben wir uns verloren. Ich bin nach Prag gekommen mit Stella Beck\index{p}{Beck, Stella}, Samuel Kessler\index{p}{Kessler, Samuel} und einem Mann aus Frankfurt.\footnote{Interview mit Anna Hyndr\'akov\'a vom 29.03.2005 in Prag}
\end{leftbar}
\newpage
Wohin die Wachmannschaften geflohen sind, ist nicht ganz sicher. Lagerkommandant Erich Rechenberg\index{p}{Rechenberg, Erich}\index{p}{Rechenberg\index{p}{Rechenberg, Erich}, Erich} wurde jedenfalls am 11. Mai 1945 zusammen mit dem Wachmann Werner Weiss\index{p}{Weiss, Werner} in Deutsch-Gabel\index{o}{Deutsch-Gabel} (heute: Jablonné v Podještědí, Tschechien) durch die Rote Armee gefasst.\newline

Das Lager überließ man dem Chaos. Der ehemalige Häftling Leon Hostig beschreibt wie sich Gefangene sich nun an ihren Folteren aus den eigenen Reihen rechten:
%says Hostig. It was thus that the stories of many "bad Kapos" came to an end. Inmates made short work of hundreds of their former tormenters, using whatever weapons came to hand.
%Hostig had been transferred from the kitchen to theWagon Bau and had sharpened the handle of his metal soup spoon on a grindstone
\begin{leftbar} 
In dieser Zeit konntest du jeden umbringen und musstest dir deshalb keine Sorgen machen. Ich sah den ein [Kapo] aus dem Lager kommen; er sah uns auch. Er war von Baracke 7 und er kannte mich, weil ich eine Zeit lang in der Küche arbeitete. Er wirkte so gnädig und lächelte, ich sah ihn an, wie brutal er vorher war und wie er niemals lächelte. Ich nahm den [an einem Schleifstein geschärften] Löffel, stieß ihn ihm in die Seite und ging weiter. Ich weiß nicht, ob ich ihn getötet habe, aber ich ließ den Löffel in ihm stecken. Ich ließ ihn am Straßenrand liegen. 
%That time you could kill anybody, and you didn't have to worry about it. I saw one [Kapo] coming from the camp, and he saw us. He was from Barrack Number 7, and he knew me specifically, because I worked awhile in the kitchen. He was so mild and nice and smiling, and I looked at him, how vicious he was before, and how he never smiled before. And I took that spoon [which he had sharpened on a grindstone], and I sticked him in the side, and we walked away. I don't know if I killed him, but I left that spoon sticking in him. I left him on the side of the road.
\footnote{Leon Hostig, in: David van Biema: Poisoned Lives.}
\end{leftbar} 

Die Befreiung des Lagers und der Stadt Görlitz geschah zeitgleich am 8. Mai 1945. Leider lässt sich nicht mehr genau feststellen, welche Einheit der Roten Armee an jenem Morgen im Lager eintraf. Somit konnten auch keine Informationen über die Anzahl und die körperliche Verfassung der Gefangenen in Erfahrung gebracht werden.\newline

Jakob Kinrus'\index{p}{Kinrus, Dr. Jakob} Erlebnisse am Tag der Befreiung:
\begin{leftbar} 
Am 8. Mai 1945 früh um 6 Uhr sind wir aufgestanden, aber haben gesehen, dass Lagerführer und Lagerältester\index{p}{Czech, Hermann} und alle, die uns bisher bewacht hatten, verschwunden waren. Von Weitem hörten wir Schüsse und sahen, dass russische Truppen sich näherten. Deshalb hängten wir eine weiße Fahne heraus und gingen ihnen entgegen. [...] einer, der russisch konnte, sagte ihnen: \glqq Hört mal zu, das ist ein Arbeitslager und wir sind Juden ...\grqq. Ich war immer negativ zu Russland eingestellt. Aber Tatsache ist, dass sie mir das Leben gerettet haben. Ich bin der Roten Armee dankbar, die mich befreit hat.\footnote{Aussage von Jakob Kinrus, in Gräfe / Töpfer: Ausgesondert und fast vergessen.}
\end{leftbar}


\myfigure[befreier]{YV-Kinrus-Befreier.png}{YV 1869/251.}{Soldaten und Offiziere der Roten Armee, welche das Lager befreiten. Hier zusammen mit Jakob Kinrus (vierter von links).}{Soldaten und Offiziere der Roten Armee.}{0}

\myfigure[lager3]{lager3}{USHMM 16474.}{Leon Greenwald\index{p}{Greenwald, Leon} (rechts) und sein Bruder am 12. Mai 1945 im ehemaligen KZ-Lager}{Lager am 12. Mai 1945}{0}

%%%
\subsection{Die Zeit danach}
Viele der ehemaligen Häftlinge schlossen sich unmittelbar nach der Befreiung zusammen und gingen in die Stadt, um dort zunächst ihre Grundbedürfnisse nach sauberer Kleidung, Essen und einer Unterkunft zu befriedigen. Die Stadt war menschenleer und somit standen in vielen Wohnungen und Häusern Tür und Tor geöffnet. Manche sahen es als ihr gutes Recht an, für das durch die Deutschen entstandene Leid einen materiellen Ausgleich zu suchen\footnote{Vgl. Schlomo Graber: Schlajme, S. 88ff. Vgl. auch Interview mit Sol Munn,  Juni 1984.}. Dr. Jakob Kinrus\index{p}{Kinrus, Dr. Jakob} und andere kümmerte sich zunächst um die noch immer zahlreichen schwer kranken Menschen im Lager. Der Zahnarzt Kinrus trat deshalb in Verhandlung mit sowjetischen Ärzten. Für besonders schwere Fälle erwirkten die Lagerärzte bereits vor der Befreiung Einweisungen in ein örtliches Krankenhaus (siehe Abschnitt über das Krankenrevier, S.~\pageref{krankenhaus}). Weitere Einlieferungen hat es mit Sicherheit gegeben. 

Keiner der Gefangenen stammte aus der unmittelbaren Umgebung von Görlitz, die meisten waren hunderte, ja sogar über tausend Kilometer von daheim entfernt. Die Ungewissheit über das Schicksal ihrer Familienangehörigen mag viele dazu bewogen haben, schnellstmöglich nach ihren Verwandten zu suchen und in ihre Heimat zurück zu kehren. Um die Heimkehr zu erleichtern, gründete man in Zusammenarbeit mit der Roten Armee ein Ortskomitee\footnote{Aussage von Jakob Kinrus. Vgl. Gräfe / Töpfer: Ausgesondert und fast vergessen.}. Für viele der ehemaligen Gefangenen verlief die Heimreise über eine Reihe von Auffanglagern. Andere ließen sich jedoch für einige Jahre in Görlitz und Umgebung nieder\footnote{Janek Oborowicz zum Beispiel.}. 

\myfigure[aerzte1945]{YV-1945-Aerzte.png}{YV 1869/252.}{Jakob Kinrus (rechts) mit zwei anderen Ärtzen des Lagers, ein Tag nach der Befreiung.}{Jakob Kinrus mit zwei anderen Ärtzen des Lagers.}{0.85}

\myfigure[benleave]{h2_color300}{Später in den USA nannte er sich Ben Munn. Das Bild entstammt aus dem Privatbesitz von Samson Munn, dem Sohn des Zeichners.}{Ben Mącznik\index{p}{Mącznik, Ben} -- Selbstportrait vom 3. April 1945}{Ben Mącznik -- Selbstportrait (2)}{1.3}

\myfigure[kinrus-juedinnen]{YV-1945-Kinrus-UngJuedinnen.png}{YV 1869/250.}{Jakob Kinrus mit sowjetischen Soldaten und ungarischen Jüdinnen.}{Jakob Kinrus mit sowjetischen Soldaten und ungarischen Jüdinnen.}{0.9}

\paragraph{Unter Tulpen und Narzissen}
Im Jahre 1946 begann man einige der Holzbaracken im Biesnitzer Grund zu demontieren und anderenorts wieder aufzubauen\footnote{Zumindest eine solche Baracke stand, laut Hildegard Ecke und Kurt Wolf, später in Ebersbach bei Neugersdorf.}. Im selben Jahr bemühten sich die ersten Kleingärtner, Steine und Zementblöcke weg zu schleppen und ein Stück Land zu gewinnen, auf dem sie sich in jener Notzeit der Nachkriegsjahre zunächst Kartoffeln und Gemüse anbauen konnten. In der unmittelbaren Nachbarschaft gab es bereits einige Gärten, weshalb es offenbar naheliegend erschien, für weitere Gartengrundstücke auf diesem mit \glqq Schweiß und Blut gedünkten Boden\grqq\footnote{Zitiert aus einem Artikel der Sächsischen Zeitung vom 8. Juli 1955: Mit Schweiß und Blut gedünkter Boden im Biesnitzer Grund.} Platz zu machen.

\myfigure[garten]{nk_biesnitz}{}{Die Kleingartenanlage Biesnitzer Grund im Jahre 2007.}{}{0}
\newpage
Spricht man heute mit Hobbygärtnern der Kleingartenanlage \glqq Biesnitzer Grund\grqq~(siehe Bild ~\mypicsref{garten}), so sind sich fast alle der Tatsache bewusst, ihr Gemüse und ihre Blumen dort zu säen, wo einst hunderte hungrige Häftlinge ihrem Verderben entgegen sahen. So verwundert es manche gar nicht mehr, ein paar Patronenhülsen beim Umgraben zu finden oder beim Bau einer Wasserleitung ein Gewehr zu entdecken. Schließlich hat man auch einige Lauben mit den Zementbrocken der alten Baracken errichtet.



