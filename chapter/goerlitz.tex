
Die ersten jüdischen Zwangsarbeiter verschleppte man noch vor der Entstehung des KZ-Außen\-lagers\index{p}{Rechenberg, Erich} in das Barackenlager im Biesnitzer Grund; damals im Rahmen der Organisation Schmelt. Zuvor waren am selben Ort bereits Kriegsgefangene inhaftiert. Alle drei Lagertypen entstanden auf Veranlassung der Waggon- und Maschinenbau AG (WUMAG) Görlitz, welche das Lager errichten ließ und dessen unfreiwillige Bewohner im Sinne der nationalsozialistischen Ideologie für die Kriegsproduktion ausbeutete. Verantwortungslos und zynisch bezeichnete das Unternehmen diese Haftstätten als \glqq Menschenlager\grqq\footnote{In einem Geschäftsbericht der WUMAG vom 15.03.1944 heißt es: \glqq Durch die Errichtung der Barackenstadt, die Sie uns in der Sitzung vom 11. November 1943 genehmigt hatten, wird ein Teil der anderen Menschenlager frei werden.\grqq, StArchD 11693 WUMAG / 1258.}, deren Bestände möglichst erschöpfend und gewinnbringend für den Arbeitseinsatz genutzt wurden. \newline
Als das Lager im Sommer 1944 dem KZ Groß-Rosen\index{o}{Groß-Rosen} unterstellt wurde und folglich die SS die Verantwortung für die Beschaffung und Disziplinierung der Häftlinge übernahm, rückte eine Gesellschaft von zunehmend erschöpften, innerlich gebrochenen und apathischen Wesen ins Auge der Görlitzer Öffentlichkeit. Es handelte sich um eine Gruppe von 1500 Juden -- um Männer und Frauen, die sich hinsichtlich ihres Alters, ihrer Nationalität und ihrer einstigen gesellschaftlichen Stellung wesentlich unterschieden, jedoch dem System und dem Terror des Konzentrationslagers, wenn überhaupt, nur schwerlich Stand halten konnten. Bedingt durch mangelnde medizinische Versorgung, widerliche hygienische Verhältnisse, schlechte Kleidung und unzureichende Ernährung häuften sich Krankheiten und Todesfälle. Auch die harte und unmenschliche Arbeit in den Werken der WUMAG trug wesentlich zur Sterblichkeit der Häftlinge bei.
\\
%%
%%%%%%
\section{Zur Vorgeschichte}
Im Südwesten von Görlitz, unmittelbar an der Grenze zur kleinen Gemeinde Biesnitz, die erst 1951 in die Stadt Görlitz eingemeindet wurde, errichtete die WUMAG auf dem Gelände einer alten Ziegelei\footnote{Die Ziegelei war ehemals Teil der Maschinenfabrik Roscher, in welcher die damals bekannten Maro Ziegeleimaschinen hergestellt wurden. Laut einer Flurstückskarte aus dem Jahr 1935 befand sich die Ziegelei und damit auch das spätere Lager auf Görlitzer Flur.} ein Barackenlager mit der Absicht, dort Zwangsarbeiter einzuquartieren. Das Grundstück ist bis heute städtisches Eigentum\footnote{Grundbuchblatt 14286 Görlitz: Flurstück 38/15, Amtsgericht Görlitz.}, wurde jedoch vor dem Krieg an den Fabrikanten Roscher\index{p}{Roscher} und einen Bauern namens Ecke\index{p}{Ecke, Erich} verpachtet.
Im Jahre 1939 kündigte die Stadt alle Pachtverträge\footnote{Erich Ecke konnte ab dem Jahre 1939 die Wiese im Biesnitzer Grund nicht mehr nutzen. Laut Aussage von Alfred Ecke, dem Sohn des Bauern.} zugunsten der WUMAG, welche bereits anderenorts Barackenlager errichtet hatte\footnote{Erste Erkundigungen über Barackenfabrikationen und deren Preisbindung erfolgten am 16.10.1939. Am 24.10.1939 tritt man in Verhandlungen mit dem Reichsarbeitsdienst wegen Übernahme von Baracken und erteilt einen Auftrag für 20 Stück. StArchD 11693, 1258 (Bild-Chronik III).}. Die Kosten und die Verantwortung für diese Bauprojekte trug die WUMAG allein\footnote{In der Bilanz 1940/41 ist der Kauf von fünf Baracken für das Russenlager mit 43.000 RM verbucht. StArchD 11693 / 1027 und 1258 (Bild-Chronik III).}. Im Biesnitzer Grund führte man zunächst ein Lager für 300 französische Kriegsgefangene\footnote{StArchD 11693 WUMAG / 1035 1164: Meldung der WUMAG-Lager an die Rüstungsinspektion VIIIa Breslau: WUMAG-Liste über gemietete Räumlichkeiten.}.
Nach dem Russlandfeldzug entstand ein zweites Lager in unmittelbarer Nachbarschaft für 450 sogenannte \glqq Ostarbeiter\grqq\footnote{Ebenda. Vgl. auch Thomas Warkus: Kriegsgefangene und Fremdarbeiter im nationalsozialistischen Deutschland 1939--1945. Das Beispiel Görlitz. Ostarbeiter ist seit dem 2. Februar 1942 die amtliche Bezeichnung für sowjetische Fremdarbeiter und Fremdarbeiterinnen. Vgl. Martin Weinmann: Das Nationalsozialistische Lagersystem, S. LIX.}. Dergleichen galten als slawische Untermenschen, für die die nationalsozialistische Rassenlehre, genauso wie für die Juden, allenfalls ein Dasein als Staatssklaven vorsah\footnote{Wolfgang Sofsky: Die Ordnung des Terrors -- Das Konzentrationslager, S. 139f.}. Über ihr Schicksal ist in Görlitz allerdings nur wenig bekannt.
\newline
Im Geschäftsbericht der WUMAG vom November 1943 beklagt sich die Geschäftsführung über die hohen Kosten, die die vom Arbeitsamt zugewiesenen Zwangsarbeiter verursachten. Nicht nur über den Aufwand die ungelernten Arbeitskräfte anzulernen wird geklagt, sondern auch über Belastung für Unterbringung, Verpflegung, Überwachung, Errichtung einer Entlausungsstation und einer Sanitätsstube. Weiter heißt es: \glqq Im Gegensatz zu den Franzosen sind wir mit den russischen Kriegsgefangenen in Bezug auf ihren Arbeitswillen und ihr Verhalten sehr zufrieden. Leider können sie wegen ihres Gesundheitszustandes im Durchschnitt aber nur als 50 Prozent leistungsfähige Kräfte eingesetzt werden.\grqq\footnote{StArchD 11693 / 1027.}. Anstatt sich der Ursachen anzunehmen, setzte man auf gewaltsame Mittel, um eine ökonomisch rationale Kostenminimierung zu erreichen.
Bereits am 2. August 1942 kam es im \glqq Russenlager\grqq~zu einer Arbeitsverweigerung wegen schlechter Kost. Unter Androhung von Erschießung gaben die extra angerückten Soldaten den Streikenden zwei Minuten Bedenkzeit, um wieder an die Arbeit zu gehen. Sie gaben nach, doch drohte man mit sofortiger Hinrichtung, sollte sich solch ein Vorfall wiederholen\footnote{StArchD 11693 / 1258 Bild-Chronik III. Vgl. auch: Thomas Warkus: Kriegsgefangene und Fremdarbeiter im nationalsozialistischen Deutschland 1939--1945. Das Beispiel Görlitz.}.

%%%%%%%%%%%%%%%%%%%%%%%%
\subsection{Das Zentrale Arbeitslager der Organisation Schmelt}
Der SS-Oberführer und Breslauer\index{o}{Breslau} Polizeipräsident Albrecht Schmelt\index{p}{Schmelt, Albrecht} koordinierte in seiner Position als \glqq Sonderbeauftragter des Reichsführers der SS für fremdvölkischen Arbeitseinsatz in Oberschlesien\grqq, seit 1940 die Zwangsarbeit in Ghettos und Arbeitslagern. Die nach ihm benannte Organisation operierte anfangs nur in Oberschlesien weitete sich jedoch schon bald nach Niederschlesien und ins Sudetenland aus. Es entstanden so genannte Zentrale Arbeitslager (ZAL) in der Nähe von kriegswichtigen Unternehmen wie der WUMAG Görlitz\footnote{Bei einer Wirtschaftsprüfung vom 29. Oktober bis 9. November 1940 wurde die WUMAG zum kriegswichtigen Unternehmen erklärt. StArchD 11693 / 1258 Bild-Chronik III. Das ZAL Görlitz wird in Zusammenhang mit dem Rüstungsbetrieb WUMAG genannt. Vgl. Alfred Konieczny: Die Zwangsarbeit der Juden in Schlesien im Rahmen der Organisation Schmelt, S. 104.}, in denen zumeist die jüdische Bevölkerung Oberschlesiens Zwangsarbeit verrichtete\footnote{Vgl. Martin Weinmann: Das Nationalsozialistische Lagersystem, S. LVIII.}. \glqq Die organisatorische Umsetzung des Zwangsarbeitereinsatzes oblag den Jüdischen Ältestenräten in Oberschlesien, die auf deutsche Verordnung in Sosnowiec zusammengefasst waren [...]\grqq, schreibt Andrea Rudorff\footnote{Andrea Rudorff: Das Lagersystem der Organisation Schmelt in Oberschlesien. S. 159.}. Die Verteilung jüdischer Arbeitskräfte richtete sich nach den Bedarfsmeldungen der Betriebe, anhand derer der Jüdische Ältestenrat der Organisation Schmelt schriftliche Aufforderungen zum Arbeitseinsatz versandt. Die Verteilung jüdischer Arbeitskräfte beruhte auf einer zwischen Schmelt und den jeweiligen Betrieben getroffenen Vereinbarung, die im Sinne strikter Aufwandsminimierung die Löhne und die innere Organisation der Arbeitslager festlegte\footnote{Vgl. Israel Gutman: Enzyklopädie des Holocaust -- Band 2, S. 1070.}. Schmelts\index{p}{Schmelt, Albrecht} Lagerführung stützte sich auf eine Häftlingsselbstverwaltung, die sich bereits in Konzentrationslagern bewährte hatte, und sparte somit an Wachpersonal. Es gab einen Ältesten, einen Krankenbehandler und einen Stenotypisten für je 150 Zwangsarbeiter, genauso wie einen Schuster, einen Schneider und einen Koch samt Küchengehilfen\footnote{Vgl. Martin Weinmann: Das nationalsozialistische Lagersystem (CCP), S. LVIII. Vgl. Alfred Konieczny: Die Zwangsarbeit der Juden in Schlesien im Rahmen der Organisation Schmelt, S. 101.}. Somit genügte ein Wachmann bzw. Hilfspolizist oder Angehöriger des Werkschutzes zur Bewachung von je 40 Gefangenen.
\newline
Der bisherige Erkenntnisstand über das Lagersystem der Organisation Schmelt insgesamt sowie über das ZAL Görlitz im Speziellen ist äußerst unzureichend und beschränkt sich auf wenige Quellen\footnote{Vgl. Andrea Rudorff: Das Lagersystem der Organisation Schmelt in Oberschlesien. S. 159.}.
Das Lager der Organisation Schmelt in Görlitz entstand wahrscheinlich am 26. April 1943\footnote{Entsprechend einer Notiz in der \glqq Chronik zur Geschichte des antifaschistischen Widerstandskampfes\grqq. Nathan Klajman\index{p}{Klajman, Nathan} gibt an, im Mai desselben Jahres in das Lager gekommen zu sein. ZHI 301/2765.} im Biesnitzer Grund. Hinweise, wonach im April 1943 die Häftlinge des ZAL Görlitz nach Kittliztreben (Trzebień, Polen)\index{o}{Kittliztreben} überstellt wurden\footnote{Martin Weinmann: Das nationalsozialistische Lagersystem (CCP), S. 579.} konnten ebenso wenig bestätigt werden wie die Existenz eines \glqq Judenlagers\grqq~innerhalb des Maschinenbaukomplexes der WUMAG (siehe Luftaufnahme S.~\pageref{maschinenbaufoto})\footnote{Nach Ermittlungen von Kurt Wolf aus Löbau, der mir seine Erkenntnisse im Zusammenhang mit dem Schmelt-Lager und dem Außenlager Görlitz zuteil werden ließ.}.
Zeitzeugenberichte aus dem Biesnitzer Lager sind äußerst rar. Die Familie Ecke des benachbarten Bauernhofes hatte während dieser Zeit jedenfalls Zugang zum Lager, um dort die Jauche aus der Latrine abzuschöpfen. Der Bauer Ecke\index{p}{Ecke, Erich} pflegte sogar eine persönliche Beziehung zu einem der Insassen\footnote{Es kam während dessen zu Tauschgeschäften (Leder und Stoffe) zwischen Eckes und dem Gefangenen. Nach der Befreiung des Lagers kam es zu einem Wiedersehen auf dem Hof der Eckes. Laut Aussage von Alfred Ecke.}.~\newline
Der Schmelt-Häftling Nathan Klajman\index{p}{Klajman, Nathan}:
\begin{leftbar}
Im Mai 1943 wurde ich zusammen mit 70 Leuten ins Konzentrationslager nach Görlitz verschleppt. Dort waren 200 Juden. Wir arbeiteten in der Munitionsfabrik, wir wurden unbarmherzig geschlagen. Der Judenälteste Babinger\index{p}{Babinger} war zu uns sehr gut.
Ich blieb dort 10 Monate. Wir bekamen 40 dkg [= 400 Gramm] Brot und Suppe. Die Meister misshandelten uns.\footnote{Nathan Klajman ist in Berlin geboren und durchlebte eine ganze Reihe von Arbeitslagern, darunter Annaberg (auch \glqq Kretschamberg\grqq, heute Cha\l upki, Polen), Graditz (Oberschlesien), bevor er in das KZ-Außenlager Kittlitzstreben (Trzebień, Polen) gebracht wurde. Anfang Februar zwang man ihn und viele andere, einen Todesmarsch über Görlitz nach Zittau anzutreten. Zu Kriegsende befreite man ihn im KZ-Außenlager Zittau (heute: Sieniawka, Polen). ZIH 301/2765.}
\end{leftbar}
Der Schmelt-Häftling Jakob F.\index{p}{F., Jakob}:
\begin{leftbar}
Nach 5--6 Monaten wurde eine Gruppe von 50 Mann, darunter auch ich, [vom ZAL Neukirchen\index{o}{Neukirchen} bei Breslau\index{o}{Breslau}] nach Görlitz abtransportiert. [...] Görlitz war ein kleines Judenlager, ca. 300--350 Juden. Gearbeitet wurde bei der Firma WUMAG -- Waggon- und Maschinenbau -- ein großer Betrieb. [...] Ich mit noch acht Kameraden waren beschäftigt bei Dudel\index{p}{Dudel} als Schachtarbeiter und beim Bau großer Fabrikhallen. Kleidung: Zivil. [...] Bewachung: Wehrmacht. Bei der Arbeit bewacht durch Werkschutzleute des Betriebes Waggonbau. Judenältester: Benjamin R.\index{p}{R., Benjamin} [...] Judenfrauen: 12--13Frauen als Lagerpersonal. [...] Nach 2--3 Monaten wurden die Frauen abtransportiert, wohin weiß ich nicht. Anfang 1944 wurde Görlitz liquidiert und wir kamen ins KZ Kittlitztreben\index{o}{Kittlitztreben} [Trzebień, Polen].\footnote{Der Nachname des Zeugen wurde aus datenschutzrechtlichen Gründen vom Internationalen Suchdienst in Bad Arolsen nicht bekannt gegeben. ISD Sachdokumente M 9 Sakrau, S. 1 (2006).}
\end{leftbar}

\glqq Das Schicksal der jüdischen Gefangenen in den Lagern der Organisation Schmelt unterschied sich zum größten Teil nicht von dem aller anderen Insassen von Konzentrationslagern.\grqq, heißt es in der Enzyklopädie des Holocaust\footnote{Israel Gutman (Hrsg.): Enzyklopädie des Holocaust, Band 2, S. 1071.}.
\newline
Nathan Klajman\index{p}{Klajman, Nathan} gab an, zwischen Mai 1943 und März 1944 im ZAL Görlitz inhaftiert gewesen zu sein
\footnote{Nathan Klajman, ZIH 301/2765.}. Anhand der Belegschaftsstatistiken der WUMAG kann keine Belegung mit Schmelt-Häftlingen belegt werden. Offenbar wurden jene Zwangsarbeiter entgegen späterer Statistiken entsprechend ihrer Nationalität, statt vermeintlichen Religionszugehörigkeit verzeichnet\footnote{Zwischen Februar und März 1944 reduziert sich die Zahl der unter \glqq Sonstige\grqq~erfassten Ausländer um 85 Personen. Neben Belgiern, Franzosen, Polen und Russen erscheint die Zeile \glqq Sonstige\grqq~ab September 1944 mit dem handschriftlichen Vermerk \glqq Juden und Häftlinge\grqq. StArchD 11693 / 1258 (Bild-Chronik III).}.
\newline
Bereits Ende 1941 beschloss man im Zusammenhang mit der Endlösung der Judenfrage erstmals die Auflösung der ohnehin zeitlich befristeten ZAL. Die Umsetzung des Beschlusses verschob sich aufgrund erheblicher Einwände seitens der Wehrmacht und des Reichssicherheitshauptamtes, bis Himmler\index{p}{Himmler, Heinrich} 1943 die Auflösung der Lager, in denen die Arbeitskraft der Häftlinge als nicht kriegswichtig angesehen wurde, durchsetzte. Die WUMAG Görlitz befand sich in dieser Zeit in einer starken Abhängigkeit von ausländischen und jüdischen Fachkräften, um die von der Wehrmacht geforderten Rüstungsgüter liefern zu können\footnote{Vgl. Thomas Warkus: Kriegsgefangene und Fremdarbeiter im nationalsozialistischen Deutschland 1939--1945. Das Beispiel Görlitz.}. Es bestand also eine Notwendigkeit möglichst alle Arbeitskräfte zu behalten.
Unter der Führung von Adolf Eichmann\index{p}{Eichmann, Adolf} wurden 28 Lager in Niederschlesien und im Sudetenland -- nach vorheriger \glqq Selektion\grqq~der Häftlinge -- dem KZ Groß-Rosen\index{o}{Groß-Rosen} angegliedert, 20 davon als Außenlager\footnote{15 dieser Lager übergab man der Kommandantur von Auschwitz.}. Häftlinge aus über 125 Lagern wurden auf die Außenlager\footnote{Im folgenden soll nun von \glqq Außenlagern\grqq~oder \glqq Nebenlagern\grqq~gesprochen werden. Im SS-Sprachgebrauch hießen diese Lager im Falle von Männerlagern \glqq Arbeitslager\grqq~(abgekürzt AL) und bei Frauenlagern \glqq Frauenarbeitslager\grqq~(FAL), zum Teil auch \glqq Außenkommandos\grqq. Vgl. Isabell Sprenger: Groß-Rosen, S. 227.} von Groß-Rosen\index{o}{Groß-Rosen} und Auschwitz\index{o}{Auschwitz} verteilt\footnote{Vgl. Alfred Konieczny: Die Zwangsarbeit der Juden in Schlesien im Rahmen der Organisation Schmelt, S. 107.}. Das ZAL in Görlitz muss als eines der letzten Lager, spätestens Mitte 1944, von Groß-Rosen\index{o}{Groß-Rosen} übernommen worden sein.


\myfigure[lagerluft]{lager0}{Luftbilddatenbank, Ing.-Büro Dr. H.G. Carls.}{Luftaufnahme des Barackenlagers im Biesnitzer Grund vom 30. Mai 1944}{Luftaufnahme des Lagers, 30. Mai 1945}{0}%0.78

\myfigure[grossrosen0]{grossrosen0}{}{KZ Groß-Rosen}{KZ Groß-Rosen}{0}



\begin{fshaded}\vspace{-.5cm}\subsection*{Das Konzentrationslager Groß-Rosen\index{o}{Groß-Rosen}}
Etwa 60km südwestlich von Breslau in der heutigen Kreis Wa\l brzych befindet sich das Dorf Rogo\'znica, dessen deutscher Name Groß-Rosen\index{o}{Groß-Rosen}  während des Zweiten Weltkrieges eines der größten und berüchtigsten nationalsozialistischen Konzentrationslager bezeichnete. Die gute Eisenbahnanbindung an der Strecke von Jauer nach Striegau und insbesondere der nahe der Ortschaft gelegene Steinbruch mit seinem großen Vorkommen an schwarz-weißen Schlesischen Granits weckte bald schon das Interesse des SS-Unternehmens \glqq Deutsche Erd- und Steinwerke GmbH\grqq~\-(DEST). Im Jahre 1940 erwarb bzw. pachtete die DEST den Steinbruch und das Steinbruchgelände. Geplant war eine Vergrößerung des Steinbruchs und die rentable Herstellung von Werksteinen im Umfang von 50.000 Kubikmetern jährlich. Die zum Abbau benötigten Arbeitskräfte konnten aufgrund der bestehenden Zusammenarbeit mit der \emph{Inspektion der Konzentrationslager} im KZ Sachsenhausen in gleicher Weise durch den Einsatz von KZ-Häftlinge aufgebracht werden. In Groß-Rosen\index{o}{Groß-Rosen} wurde ein Außenkommando des KZ-Sachsenhausen errichtet. Am 2. August 1940 verlegte man die ersten 100 Gefangenen vom Hauptlager dorthin. Bereits 1941 beabsichtigte man eine Erweiterung des Steinbruchs auf 400\,m Länge, 140\,m Breite und 70--80\,m Tiefe. Die Zuführung von Arbeitskräften und Leitung des Lagers von Sachsenhausen aus, erwies sich zunehmend als schwierig, so dass Groß-Rosen\index{o}{Groß-Rosen} schlussendlich am 1. Mai 1941 in ein selbständiges Konzentrationslager umgewandelt wurde. Bis zum Januar 1945 stieg die Zahl der Inhaftierten von anfangs 900 auf 76728 an. Die alleinige Beschäftigung der Häftlinge im Steinbruch und im Granitwerk sowie in verschiedenen Kommandos innerhalb des Lagers (Maurer-, Tischler-, Schuster-, Gärtner- und Webereikommando) konnte aufgrund der steigenden Anforderungen der totalen Kriegsführung nicht aufrecht erhalten werden, so dass bereits 1942 das erste Außenlager in Breslau-Lissa (Wroc\l aw-Leśnica, Polen)\index{o}{Breslau-Lissa} eingerichtet wurde. Die meisten der ungefähr 100 Außenlager (siehe Karte~\mymapsref{grnebenlager}) entstanden bei bestehenden oder aus luftbedrohten Gebieten verlegten Betrieben der Kriegswirtschaft bzw. Rüstungsindustrie.
Eine gesonderte Stellung hatten die 12 Lager des Komplex Riese im Eulengebirge, in denen 13.000 jüdische Häftlinge unterirdische Fertigungsstätten und ein neues Führerhauptquartier bauten. Mit Herannahen der sowjetischen Truppen im Januar 1945 begann die schrittweise Evakuierung der bedrohten Außenlager, während das Hauptlager durch mehrere Kolonnen der von Auschwitz kommenden Todesmärsche maßlos überfüllt war. Um den 8./9. Februar 1945 erfolgte der Abtransport der Häftlinge aus Groß-Rosen\index{o}{Groß-Rosen} nach Buchenwald\index{o}{Buchenwald}, Flossenbürg\index{o}{Flossenbürg} (bzw. dessen Außenlager Hersbruck\index{o}{Hersbruck} und Leitmeritz\index{o}{Leitmeritz}), Mauthausen\index{o}{Mauthausen} und die Außenlager Dora\index{o}{Dora} und Nordhausen\index{o}{Nordhausen} des KZ Mittelbau. Nur wenige Häftlinge blieben in Groß-Rosen\index{o}{Groß-Rosen} zurück und wurden am 13. Februar 1945 von der sowjetischen Armee befreit.
\end{fshaded}

\mymapfigure[grnebenlager]{map_nebenlager300}{}{Außenlager des KZ Groß-Rosen}{Außenlager des KZ Groß-Rosen}{0}{}%0.262


%%%%%%%%%%%%%%%%%%%%%%%%%%%%%%%%%%%%%%%%%%%%%%%%
\section{Entstehung des KZ-Außenlagers Görlitz}
Die Auflösung der großen Ghettos in Schlesien und des Generalgouvernements sowie die \glqq Säuberungsaktionen\grqq~in Transkarpathien und Ungarn bedingten, ebenso wie die Auflösung des KZ Plaszow\index{o}{Plaszow}, seit März 1944 die Errichtung zahlreicher Männer- und Frauenlager unter dem Kommando Groß-Rosens\index{o}{Groß-Rosen}\footnote{Vgl. Alfred Konieczny: Frauen im Konzentrationslager Groß-Rosen in den Jahren 1944-1945, S. 7. Vgl. auch: Israel Gutman: Enzyklopädie des Holocaust, Band 1, S. 570f.} (siehe Karte~\mymapsref{grnebenlager}). Die erste Erwähnung fand das Außenlager Görlitz am 09.06.1944 durch das Wirtschaftsverwaltungshauptamt der SS. Am 08. August 1944 übernahm SS-Oberscharführer Zunker\index{p}{Zunker, Winfried} in der Funktion des Lagerführers\index{p}{Zunker, Winfried}, zunächst das Lager für Männer. Ab September stand auch das neu gebildete Frauenlager unter seiner Leitung.\newline

Sowohl in der Literatur als auch im Volksmund wird das Barackenlager im Biesnitzer Grund, in dem unter anderem KZ-Häftlinge untergebracht waren, als \glqq KZ Biesnitzer Grund\grqq~\linebreak oder \glqq KZ Görlitz\grqq~bezeichnet.
Die Bezeichnung \glqq Biesnitzer Grund\grqq~ mag den Görlitzer Bürgern ein Begriff sein, doch ist der Name sowohl international als auch seitens der ehemaligen Gefangenen gänzlich unbekannt.\newline

Wenngleich die Wehrmacht oder später die SS an diesem Ort bestimmte Personengruppen \glqq konzentrierte\grqq, wurde es von amtlichen Stellen stets unter anderen Namen geführt. Zunächst als Außenlager des Stalag VIII A\footnote{Stammlager für Kriegsgefangene im Wehrbezirk VIII A (1. Lager im Wehrkreis VIII Breslau).}, als Arbeitslager der Organisation Schmelt, anschließend wahrscheinlich wieder als Außenlager des Stalag.
Erst durch die Übernahme des Lagers durch das KZ Groß-Rosen\index{o}{Groß-Rosen} galt es offiziell als Außenlager des KZ Groß-Rosen\index{o}{Groß-Rosen}, aber nicht als Konzentrationslager, da ausschließlich jene 25 Lager im definierten Sinne zu den Konzentrationslagern gehörten, die von der Inspektion der Konzentrationslager als solche geführt wurden\footnote{Diese Differenzierung ist für die Geschichtsschreibung vorteilhaft, um strukturelle Unterschiede und Verknüpfungen der Lager zu verdeutlichen. Im übrigen gelten auch die frühen Schutzhaftlager der Jahre 1933--1935, wie etwa jene in Leschwitz oder Hainewalde, nicht als Konzentrationslager. Verzeichnis der Konzentrationslager und ihrer Außenkommandos gemäß \textsection~42 Abs. 2 BEG, in Bundesgesetzblatt (1982), S. 1571--1579; Vgl. Gudrun Schwarz: Die nationalsozialistischen Lager.
An anderer Stelle ist von 30 Konzentrationslegern die Rede. Vgl. Karin Orth: Die Konzentrationslager-SS und die Shoah, S. 93 in Gerhard Paul (Hrsg.): Experten des Terrors.}. Dieser namentliche Unterschied soll jedoch nicht über den Terror und Vernichtungscharakter der Außenlager hinwegtäuschen.

\paragraph{Der Aufbau des Lagers} Das Lager war umgeben von vier Wachtürmen (siehe Bild~\mypicsref{wachturm}) und einem fünf Meter hohen, elektrisch geladenen Stracheldrahtzaun\footnote{Aussage von Bendet Perle (Michal Zylberberg). LArchB B Rep 058 Bd. 3.} inklusive teilweisem Sichtschutz. Der Zaun bestand aus langen und kurzen Holzpfählen. Zwischen den langen Pfählen war Leitungsdraht gespannt und die versetzt stehenden kurzen Pfähle waren kreuz und quer mit Stacheldraht versehen\footnote{Vgl. Roland Otto: Die Verfolgung der Juden in Görlitz unter der faschistischen Diktatur, S. 67.}. Ein weiterer Zaun teilte das Lager in zwei ungleich große Hälften\footnote{Entsprechend dem einstigen \glqq Franzosen-\grqq~und \glqq Russenlager\grqq.}. Der kleinere südliche Teil bestand lediglich aus drei Baracken. Dort quartierte man zunächst die ersten ankommenden Männer ein, später jedoch ausschließlich Frauen. Demnach war es das Frauenlager. Das Männerlager hingegen bildeten die verbleibenden neun Baracken, einschließlich der stillgelegten Ziegelei mit Maschinenraum, Stallung sowie Brennöfen, Trockenanlagen für Rohlinge und kleineren Holzschuppen (siehe Bild ~\mypicsref{lagerluft} und Karte ~\mymapsref{lagermap}).\newline

\myfigure[wachturm]{YV-1945-Wachturm.png}{Bildausschnitt: YV 1869/252.}{Wachturm auf dem Lehmhügel}{Wachturm auf dem Lehmhügel}{0}

Eine jede Baracke stand auf einem massiven Fundament, war weitestgehend aus Holz gefertigt und mit einem Satteldach versehen. Im Inneren gab es meist einen Schlafbereich mit dreistöckigen Betten, und einen Wohnbereich, in welchem zumindest im Frauenlager Tische und Bänke Platz fanden\footnote{Interview mit Anna Hyndr\'akov\'a vom 29.04.2005 / Prag.}. Die Skizze vom Lageraufbau verdeutlicht jedoch, dass fast jede Baracke ungleiche Abmessungen aufweist und demzufolge die Gestaltung im Inneren nicht einheitlich gewesen sein kann.

\mymapfigure[lagermap]{map_lager}{Skizze des Lagers}{Anhand von Luftbildern und Bleistiftskizzen von ehemaligen Häftlingen erstellt.}{Skizze des Lagers}{0}{}

%%%%%%%%%%%%%%%%%%%%%%%%%%%%%%%%%%%%%%%%%%%%%%%%
\section{Die Häftlinge}
Ausgehend von der so genannten Gefangenen-Gestellung soll nun in einem kurzen Abriss der Weg der nach Görlitz deportierten Juden in ihre Heimat zurückverfolgt werden. Anhand der Zusammensetzung der Häftlingsgesellschaft wird deutlich, in welcher Weise die SS selbige unterwanderte und mittels Funktionshäftlingen kontrollierte. Die daraus resultierende Gefangenenhierarchie ist ausschlaggebend für die Existenzbedingungen innerhalb des Lagers und machte sich schon bald außerhalb des Lagers bemerkbar.

Der Weg aller in Görlitz inhaftierten KZ-Gefangenen führte über die geographisch nahegelegenen Konzentrationslager in Groß-Rosen\index{o}{Groß-Rosen} und Auschwitz\index{o}{Auschwitz} bzw. deren Außenstellen (siehe Karte \mymapsref{grnebenlager}). Eine direkte \glqq Überstellung\grqq aus der Heimat der Menschen bzw. den Ghettos war aus verwaltungstechnischen Gründen nicht möglich. Zudem wollte die SS sichergehen, dass keine Seuchen in die Außenlager eingeschleppt wurden und den Arbeitseinsatz gefährdeten.
\newline
Die von Roland Otto\index{p}{Otto, Roland} vertretene Annahme, wonach das Außenlager Görlitz eine \glqq Außenstelle der Lager Auschwitz-Birkenau\index{o}{Auschwitz} und Groß-Rosen\grqq~gewesen sei, ist nicht korrekt\footnote{Vgl. Roland Otto: Die Verfolgung der Juden in Görlitz unter der faschistischen Diktatur, S. 66.}. Allein die Tatsache, dass die Häftlingstransporte aus den beiden eben genannten Konzentrationslagern erfolgten, entbehrt der Implikation einer Doppelzuständigkeit für das Görlitzer Außenlager.
\newline
Die Gestellung der Häftlinge aus zwei Hauptlagern hatte organisatorische Gründe, da man frühestens bei Ankunft der Gefangenentransporte über deren Einsatzort entscheiden konnte, wenn die arbeitsfähigen Häftlinge \glqq selektiert\grqq und durch Nummern registriert waren. Darüber hinaus gab es noch weitere Ursachen für Gestellung aus den beiden Konzentrationslagern, die jedoch nachfolgend Erwähnung finden.
\newline
Die Zahl der Häftlinge im Lager Görlitz stieg bis Oktober 1944 auf 1.500 Personen an und dezimierte sich aufgrund der unmenschlichen Bedingungen im Lager auf höchstens 1.328. Erst im Zuge der Evakuierung anderer Lager und der damit verbundenen Überstellung nach Görlitz erreichte die Zahl der erfassten Inhaftierten im Februar 1945 kurzzeitig sogar die Marke von 1.750 Insassen\footnote{Der Lagerschreiber Emmrich Schiffer gibt diese Zahl am 5. Juli 1947 in einer selbst verfaßten Anklageschrift gegen den Lagerältesten Czech an. Dabei ist es möglich, dass Schiffer alle Häftlingsgestellungen aufsummierte und jene bis Februar 1945 verstorbenen (140) und in andere Lager überstellten (etwa 30) Gefangenen nicht abzog. In Anbetracht dessen und der Aufnahme von mindestens 120 Frauen im Februar sowie weiterer, kranker Menschen, ist die Zahl von 1.750 durchaus realistisch. PMGR 4702/14/DP.}.

% Die Einlieferungen von Häftlingen ins KZ Görlitz oblag dem SS Angehörigen Max Bürger. (B Rep 058 2231/4, Max Bürger)
\myfigure[benarrive]{h1_color300}{Später in den USA nannte er sich Ben Munn. Das Bild entstammt aus dem Privatbesitz von Samson Munn, dem Sohn des Zeichners.}{Ben Mącznik\index{p}{Mącznik, Ben} -- Selbstportrait nach Ankunft im Lager}{}{0}


\addtocounter{footnote}{1}
\begin{table}[h!b!p!]
\begin{tabularx}{\textwidth}{rlrlX|}
\hline
\textbf{Zeitraum} & \textbf{Lager} & \textbf{Anzahl} & \textbf{Herkunft}\\
\hline
10.08.1944
&Groß-Rosen\index{o}{Groß-Rosen}
& 25 & Deutschland u.a\\

Mitte 08.1944
& Auschwitz\index{o}{Auschwitz}
& 225 &
Slowakai, Nordungarn, Karpato-Ukraine\\

Ende 08.1944
& Fünfteichen\index{o}{Fünfteichen}\footnotemark[\value{footnote}]
& 400
& Ungarn, Polen\\

18./19.09.1944
& Auschwitz\index{o}{Auschwitz}
& 550
& Litzmannstadt (\L \'od\'z, Polen)\index{o}{Litzmannstadt}\\
\hline
\end{tabularx}
\caption{Gefangenentransporte in das Männerlager\label{transporte}}
\end{table}

%\addtocounter{footnote}{1}
\footnotetext[\value{footnote}]{Juda Widawski traf Litzmannstädter Juden bereits im Lager an, als er aus dem Außenlager Fünfteichen\index{o}{Fünfteichen} kam. LArchB B Rep 058 Bd. 3. Schlomo Graber schreibt außerdem, dass die Transporte mit Lastwagen erfolgten. Vgl. Schlomo Graber: Schlajme, S. 70.}



\paragraph{Das Männerlager}
Wie vormals erwähnt, erfolgte die erste \glqq Häftlingsgestellung\grqq, zwei Tage nachdem der\label{vorauskommando} Lagerführer\index{p}{Zunker, Winfried} Zunker\index{p}{Zunker, Winfried} in Görlitz eintraf, am 10. August 1944 (siehe Tabelle~\ref{transporte}). Dieser Transport bestand aus 25 Häftlingen, welche man in Groß-Rosen\index{o}{Groß-Rosen} gezielt für die Organisation des neu entstehenden Außenlagers ausgesucht hatte. Unter ihnen befand sich auch der spätere Lagerälteste\index{p}{Czech, Hermann} Hermann Czech\index{p}{Czech, Hermann} und der Lagerschreiber\index{p}{Schiffer, Emmrich} Emmrich Schiffer\index{p}{Schiffer, Emmrich} sowie weitere Funktionshäftlinge. Bis zum 1. September stieg die Zahl der Gefangenen auf 650 an. Zunächst kamen 225 Menschen aus Nordungarn, der Slowakei und Transkarpatien\index{o}{Transkarpatien} (Karpato-Ukraine) über Auschwitz\index{o}{Auschwitz} ins Außenlager Görlitz.
\newline
Etwa 400 ausgesonderte Juden brachte man Ende August vom Außenlager Fünfteichen\index{o}{Fünfteichen} nach Görlitz. Nach der Niederschlagung des Warschauer Aufstandes im April 1944 war das Lager Fünfteichen\index{o}{Fünfteichen} überfüllt, weshalb man sich nach Beendigung der Bauarbeiten der \glqq Berta-Werke\grqq~(Krupp) der vielen Kranken und Schwachen entledigen wollte\footnote{Aussage von Abram Rajchbart, ZIH 301/2311.}. Eine solche Verfahrensweise schien durchaus üblich. Bereits Ende 1942 / Anfang 1943 entsprach die SS den Wünschen der Unternehmen, schwache Häftlinge auszutauschen, was aber nicht unbedingt bedeutet, selbige andernorts nicht einzusetzen.\newline

%Ende November 1944:	550 Mann, die aus \L odz stammten\\schiffer

Efraim Hermann\index{p}{Hermann, Efraim} erinnert sich dessen, dass er am Tage des jüdischen Neujahrsfestes (18./19. September 1944) aus Auschwitz-\index{o}{Auschwitz}Birkenau kommend in Görlitz eintraf, auffallend präzise\footnote{Efraim Hermann und Mordechai (Mordko) Kozusarz aus Litzmannstadt (\L \'od\'z, Polen). LArchB B Rep 058 Bd. 6. In Übereinstimmung mit Abram Rajchbart, der mit dem letzten Transport Ende August 1944 aus dem Ghetto Litzmannstadt nach Auschwitz gebracht wurde und drei Wochen später in Görlitz ankam. ZIH 301/715.}. Der Transport bestand aus 550 Männern, die man nach der Auflösung des Ghettos Litzmannstadt\index{o}{Litzmannstadt} in Auschwitz\index{o}{Auschwitz} für den Arbeitseinsatz in Görlitz bestimmt hatte. Vielfach wird auch berichtet, dass gleichzeitig 300 ungarisch sprechende Frauen Görlitz erreichten.\newline

Neben den Aussagen von ehemaligen Häftlingen gibt es noch zwei weitere Quellen, die diese Transporte bestätigen. Zum einen ist es die Belegschaftsstatistik der WUMAG (siehe S.~\pageref{wumag_pers}), und zum anderen geben auch die Einäscherungsbücher des Görlitzer Krematoriums anhand der Häftlingsnummern Aufschluss über die verschiedenen Gestellungen.\newline


Zwischen dem 21. und 26. August sind den Toten ausschließlich vierstellige Nummern zugeordnet. Dies könnte ein Teil jener 25 Häftlinge gewesen sein, die am 10. August ins Lager kamen und offensichtlich schon länger in Konzentrationslagern inhaftiert waren, als die Nachfolgenden.\newline

Ab dem 26. August sind tote Gefangene verzeichnet, deren Nummern zwischen 12738 und 14300 liegen. Seit dem 3. bzw. 9. September sind jene Tote mit den Häftlingsnummern zwischen 24148 und 27802 sowie zwischen 42546 und 42976 verzeichnet. Deren zahlenmäßig große Varianz kann ein Hinweis auf die im Außenlager Fünfteichen\index{o}{Fünfteichen} erfolgte Selektion sein. Erstere Gruppe stammte aus den Ghettos in Oberschlesien, letztere aus Ungarn.
Ab dem 28. September 1944 häufen sich im Einäscherungsbuch die Nummern zwischen 56832 und 57282, welche mehrfach in Verbindung mit Gefangenen aus dem Ghetto Litzmannstadt\index{o}{Litzmannstadt} stehen\footnote{Aussage von Shlomo Moncznik (Bild ~\mypicsref{benarrive}). Videointerview von Samson Munn 1985.}.


\paragraph{Das Frauenlager}
Das KZ Groß-Rosen\index{o}{Groß-Rosen} hatte im Vergleich zu anderen Konzentrationslagern einen erheblich höheren Frauenanteil, wenngleich im Hauptlager selbst keine Frauen inhaftiert waren\footnote{Alfred Konieczny: Frauen im Konzentrationslager Groß-Rosen in den Jahren 1944--1945, S. 7.}. Die Historikerin Isabell Sprenger schreibt von 38 reinen Frauenlagern und vier Lagern, in denen sowohl Frauen als auch Männer untergebracht waren und die als ein Lager geführt wurden\footnote{Isabell Sprenger: Groß-Rosen, S. 232f.}. Neben Brünnlitz\index{o}{Brünnlitz}, Langenbielau\index{o}{Langenbielau} und Zittau\index{o}{Zittau} existierte seit September 1944 auch in Görlitz ein separates Lager für Frauen\footnote{Das Männer- und Frauenlager in Zittau entstand im Oktober 1944 in einem kleinen angrenzenden Dorf bzw. Stadtteil namens Kleinschönau (heute: Sieniawka, Polen). Israel Gutman: Enzyklopädie des Holocaust, Band 1, S. 571.}. Alle Transporte in das Görlitzer Lager erfolgten jedoch über Auschwitz\index{o}{Auschwitz}. Eine Selektion und Zusammenstellung für den Arbeitseinsatz konnte in Groß-Rosen\index{o}{Groß-Rosen} aufgrund des nichtvorhandenen Frauenlagers unmöglich erfolgen. Es gab weder weibliches Wachpersonal, noch separate Baracken, in denen die Frauen für die Zeit der \glqq Quarantäne\grqq~hätten untergebracht werden können.\newline

Am 5. oder 18./19. September 1944 trafen die ersten 300 Frauen im KZ-Außenlager Görlitz ein\footnote{Aussage von Abram Rajchbart, ZIH 301/2311.}. Zuvor hatte man ihnen in Auschwitz\index{o}{Auschwitz} die Häftlingsnummern 57301--57600 zugeteilt. Es handelte sich dabei wahrscheinlich um ungarisch sprechende Frauen aus Nordungarn und Transkarpatien\index{o}{Transkarpatien} (Karpatoukraine).
Über weitere Transporte ins Frauenlager gibt es widersprüchliche Aussagen. Der von Emmrich Schiffer\index{p}{Schiffer, Emmrich} bekundete Transport von 584 Frauen aus Transkarpatien\index{o}{Transkarpatien} im Oktober ist möglicher Weise identisch mit dem zuvor erwähnten Ende September. Als verlässlich gilt jedoch die Aussage der Tschechin Anna Hyndr\'akov\'a\index{p}{Hyndr\'akov\'a, Anna}, wonach sich Mitte Februar 300 ungarisch sprechende Frauen im Lager befanden\footnote{Aus einem Interview mit ihr ging hervor, dass sie nach ihrer Flucht von einem Todesmarsch des KZ-Außenlagers Christianstadt (Krzystkowice, Polen) zusammen mit zwei weiteren Landsmänninen am 13. Februar 1945 ins Lager Görlitz kam.}.\newline
Weitere Transporte ins Frauenlager erfolgten erst ab Februar 1945 im Zuge der Evakuierung anderer Außenlager (siehe S. \pageref{eva_2}).

\paragraph{Die Ankunft}
Folgt man in Görlitz dem Schienenverlauf vom Bahnhof in Richtung Osten, so gelangt man kurz vor der Neißebrücke an das sogenannte Blockhaus. In unmittelbarer Nähe dieses Backsteinhauses befinden sich noch heute Eisenbahnrampen (siehe Bild~\mypicsref{rampe}), wo in den Kriegsjahren nicht ausschließlich nur Waren, sondern auch Gefangene verladen wurden. Es ist davon auszugehen, dass auch die Häftlinge des KZ-Außenlagers an diesem Ort ankamen.%\footnote{Die Ankunft von Gefangenentransporten am Blockhaus steht außer Frage, wohingegen die Zugehörigkeit bzw. Nationalität der ankommenden Menschen von keinem Augenzeugen identifiziert werden konnte.}.\newline

Samuel Reifer\index{p}{Reifer, Samuel} kam im Sommer 1944 zusammen mit 400 anderen Häftlingen aus dem Groß-Rosener\index{o}{Groß-Rosen} Außenkommando Fünfteichen\index{o}{Fünfteichen}. Seine kurz nach dem Krieg gemachte Aussage soll exemplarisch das Geschehen bei der Ankunft von Häftlingstransporten andeuten:
\begin{leftbar}
Am Montag trieb man uns von den Baracken mit Gewehrkolben zum Bahnhof und lud uns in drei Waggons.
Sie waren sehr überfüllt. Wir reisten zwei Tage unter furchtbaren Bedingungen ohne Essen oder Wasser. Die Hitze war nur schwer zu ertragen. Ich werde diese Reise niemals vergessen. Vor unserer Abfahrt in Fünfteichen\index{o}{Fünfteichen} nahm man uns die gestreifte Uniform ab, wenn jemand Lederschuh besaß, so nahm man auch diese. Man gab uns gefärbte Zivilkleidung.
Bei unserer Ankunft wurden wir durch Schläge des Lagerältesten\index{p}{Czech, Hermann} aus dem Zug getrieben. [...]
Am Ende kamen wir ins Lager, welches aus mehreren Baracken bestand, in welchen man schon polnische und ungarische Juden eingesperrt hatte. Man gab uns nichts zu essen. Sofort wurde ein Appell angeordnet. Meister aus dem Waggon- und Maschinbauwerk kamen ins Lager und wählten uns für die Arbeit in der Fabrik aus. Wir brüllten, dass wir hungrig sind. Sie antworteten, dass sie auf die Ankunft eines neuen Transports nicht vorbereitet waren.
Als nächstes wurde ein Bad abgehalten. Wir zogen uns auf dem Appellplatz nackt aus und warfen unsere Uniform auf einen Haufen. Die Deutschen richteten dann einen Wasserstrahl aus einem Gummischlauch auf uns. Anschließend führte man uns nackt zu den Baracken. Um 2 Uhr morgens weckte man uns, um unsere Sachen aus dem Haufen auf dem Appellplatz zu suchen. Die Deutschen erlaubten uns nicht, Unterwäsche zu tragen, weshalb der Lagerleiter jeden persönlich kontrollierte, um zu sehen ob jemand ein Unterhemd trägt. Man gab uns schwarzen Kaffee ohne Brot und nötigte uns in die Fabrik zu gehen, wo uns die Meister in verschiedene Abteilungen einteilten.\footnote{Samuel Reifer, ZIH 301/2311.}
\end{leftbar}


\myfigure[rampe]{lager_rampe}{}{Rampe am Blockhaus}{Rampe am Blockhaus}{0}



%%%%%%%%%%%%%%%%%%%%%%%%%%%%%%%%%%%%%%%%%%%%%%%%
\subsection{Häftlingsgruppen}

Nachdem bisher nur zwischen weiblichen und männlichen Gefangenen unterschieden wurde, stellt das nun Folgende einen Versuch dar, die Zusammensetzung der Häftlingsgesellschaft und ihre Gruppierungen genauer zu analysieren. \newline
Entgegen einer natürlichen, sich allmählich entwickelnden Gesellschaftsstruktur, kam es in den Konzentrationslagern innerhalb kürzester Zeit zu einer neuen Konstellation von Alter, Religion, Nationalität, politischer Überzeugung, ideologischem und moralischem Ausrichtung und Bildung\footnote{Vgl. Adolf Gawalewicz: Die Funktionshäftlinge in nationalsozialistischen Konzentrationslagern, S. 232. In: Die Auschwitz-Hefte.}. Um ein besseres Gesamtbild von der Zusammensetzung der Häftlingsgesellschaft zu vermitteln, sollen die eben genannten Faktoren in den Mittelpunkt der Betrachtung rücken und somit die Unterschiede hinsichtlich der Gewöhnung an die vorherrschenden Lebensumstände sowie die physische und psychische Widerstandsfähigkeit aufzeigen.


\subsubsection{Die Altersstruktur}
Obwohl kein vollständiges Verzeichnis der Görlitzer Häftlinge mehr existiert, geben zahlreiche Berichte von Überlebenden sowie die Liste der bekannten Opfer Aufschluss über die Altersverteilung innerhalb der Häftlingsgesellschaft.
\newline
Es ist anzunehmen, dass sich unter den Häftlingen keine Kinder unter zehn Jahren befanden, da diese bereits im Stammlager Groß-Rosen\index{o}{Groß-Rosen} und in Auschwitz\index{o}{Auschwitz} per Selektion zum Tode verurteilt wurden.
Doch ist es kaum vorstellbar wie E. \mbox{Mendel} \mbox{Rubin}\index{p}{Rubin, Mendel}, mit seinen gerade einmal 11 Jahren den Terror des Konzentrationslagers überstehen konnte. Der aus dem nordungarischen Encs\index{o}{Encs, Ungarn} stammende E. Mendel Rubin\index{p}{Rubin, Mendel} war höchstwahrscheinlich der jüngste Gefangene des Außenlagers Görlitz. Auch den drei Jahre älteren Tuvia Altmann\index{p}{Altmann, Tuvia} hielt man mit seinem Bruder und seinem Vater dort gefangen\footnote{Aussage von Tuvia Altmann, geboren am 26.07.1930 in \L \'od\'z. LArchB B Rep 058 Bd. 5.}. Ein anderer, Murray Ravitt\index{p}{Ravitt, Murray}, war zum Zeitpunkt der Deportation ins Lager Görlitz erst 15 Jahre alt. Sieben weitere Häftlinge hatten eben das 16. Lebensjahr überschritten\footnote{Cesia Finkel, Samuel Mandelbaum, Salamon Steinmetz, Max Wachsmann, Max Wodesmann, Anna Herringer, Arnold Genad. LArchB B Rep 058 Bd. 1, 2, 3, 4 und 5 sowie Jüdisch Historischen Institut Warschau 301/924.}. Einer von ihnen, Arnold Genad\index{p}{Genad, Arnold}, starb noch vor dem Evakuierungsmarsch\footnote{Arnold Genad starb am 6. November 1944, wurde jedoch erst am 22. November eingeäschert. Seine Urne verblieb in Görlitz.  Einäscherungsbücher der Friedhofsverwaltung Görlitz.}, alle anderen hier genannten überlebten diese schreckliche Zeit, doch verloren sie die kostbarsten Jahre ihrer Jugend hinter dem Stacheldraht\footnote{Die hier gemachten Angaben stützen sich auf Aussagen der nach 1928 Geborenen im LArchB B Rep 058 Bd. 1, 2, 3, 4 und 5. und im Jüdisch Historischen Institut Warschau 301/ 924. Neben diesen zehn jungen Menschen werden in den eben genannten Quellen sowie in den Einäscherungsbüchern der Görlitzer Friedhofsverwaltung weitere fünf 15-jährige und sechs 16-Jährige erwähnt. Einige von ihnen kamen erst in den letzten Monaten vor Kriegsende aus anderen Groß-Rosener\index{o}{Groß-Rosen} Außenlagern nach Görlitz. Es ist jedoch durchaus realistisch, dass es darüber hinaus noch weitere Jugendliche im Lager gab.}.

Neben den ganz jungen, waren auch die Menschen gehobeneren Alters von den verachtenden Selektionen bei Ankunft in Groß-Rosen\index{o}{Groß-Rosen} und Auschwitz\index{o}{Auschwitz} bedroht. Hierbei fällt es jedoch noch schwerer festzustellen, wie viele letztlich ins Lager Görlitz verwiesen wurden. Überlebendenberichte dieser Personen gibt es kaum, da die polizeilichen Befragungen meist erst 20 Jahre nach der Befreiung erfolgten. Einen Hinweis geben die Einäscherungsbücher der Görlitzer Friedhofsverwaltung, in denen die Todesfälle von August 1944 bis Februar 1945 verzeichnet sind. In diesem Zeitraum sind acht von 140 gelisteten Personen nach Vollendung ihres 50. Lebensjahres umgekommen. Der älteste von ihnen hieß Josef Schlesinger\index{p}{Schlesinger, Josef}. Er starb 25 Tage nach seinem 65. Geburtstag\footnote{Josef Schlesinger, geboren am 15. Dezember 1879 an einem unbekannten Ort, verstarb am 9. Januar 1945. Die Einäscherung erfolgte jedoch erst am 6. Februar. Einäscherungsbücher der Görlitzer Friedhofsverwaltung.}.\label{alter}

\subsubsection{Verschiedene Religionen}
Die große Mehrheit der Gefangenen bekannte sich zum jüdischen Glauben, wenngleich es gewisse Unterschiede in der Ausprägung und Ausübung vor und während der Zeit im Konzentrationslager gegeben haben wird. Von Schlomo Graber\index{p}{Graber, Schlomo} ist überliefert, dass sich einige von ihnen bemühten, Glaubensgesetze und Traditionen zu wahren. Das galt insbesondere für die jüdischen Feiertage\footnote{Schlomo Graber berichtet unter anderem von einem Vater mit seinen Söhnen, denen er mündlichen Talmudunterricht gab und \glqq wusste als einziger im Lager immer, wann die jüdischen Feiertage waren.\grqq, Schlomo Graber: Schlajme, S. 79.}. Aufgrund der strengen Arbeitszeiten war es jedoch niemals möglich, den Sabbat in geeigneter Weise zu begehen, ebenso gab es keine koschere Kost, um hier nur zwei Beispiele zu nennen.
Die Ausübung religiöser Rituale wurde von der SS in keiner Weise geduldet. Sogar einige der Funktionshäftlinge setzten diesen Traditionen Gewalt entgegen\footnote{\glqq [Ein Häftling namens] Hochmitz wollte den Sabbat verrichten, doch [der Blockälteste] Eichner sagte, dies hier sei keine Synagoge und versetzte ihm mehrere harte Schläge\grqq, heißt es in der Aussage von Leib Wyszegrodzki. LArchB B Rep 058 Bd. 1.}.

Unter den Gefangen gab es nur wenige Christen. Der Stammbacher\index{o}{Stammbach} Georg Rost\index{p}{Rost, Georg} behauptet zwar, er sei der einzige Christ im Lager gewesen und habe sich deshalb von den Juden etwas ausgeschlossen gefühlt\footnote{Georg Rost kehrte nach dem Krieg wieder nach Stammbach zurück. LArchB B Rep 058 Bd. 4.}, doch Hermann Czech war, wenn auch in der Funktion des Lagerältesten, sehr wahrscheinlich auch christlich getauft.

\subsubsection{Die verschiedenen Nationalitäten unter den Gefangenen}
Die Verfolgung der Juden in Deutschland setzte mit der Machtergreifung Hitlers\index{p}{Hitler, Adolf} in Deutschland ein und weitete sich ab 1938 auf fast ganz Europa aus. Am gravierendsten äußerte sich dies in Ost- und Südosteuropa, wo Menschen jüdischen Glaubens prozentual sehr stark vertreten waren. Die Art und Weise der Deportationen in die Konzentrations- und Vernichtungslager unterschied sich in den einzelnen Ländern und variierte teilweise auch in den verschieden Regionen eines Landes.
Während im Generalgouvernement, dem einstigen Polen, seit Kriegsbeginn eine Ghettoisierung im eigenen Land voran ging, lebten, wenn auch durch antijüdische Gesetze eingeschränkt, ungarische Juden bis Frühjahr 1944 mehr oder weniger in Freiheit.
Es scheint klar, dass die mentale und physische Verfassung der Menschen jener Nationalitäten bei Ankunft in den Konzentrationslagern 1944 nicht unbedingt gleich gewesen sein kann. Anhand einiger ausgewählter Regionen soll diese Annahme im folgenden noch verdeutlicht werden.


\mymapfigure[europakarte]{map_geburt}{}{Geburtsorte der Häftlinge des Görlitzer KZ-Außenlagers}{Geburtsorte der Häftlinge}{0}{}%0.488
%\vspace{-10pt}
%\mymaps{Geburtsorte der Häftlinge im Göerlitzer KZ-Außenlager}{Geburtsorte}

%~\ref{europakarte}

\paragraph{Polen}
Die Juden aus Polen stellten die größte Gruppe der Gefangenen im KZ-Außenlager Görlitz dar. Den Deportationen in die Konzentrations- und Vernichtungslager ging eine groß angelegte Verfolgung und Konzentrierung in den größten Städten voraus. Exemplarisch soll hier die Ghettoisierung in den Städten Litzmannstadt (\L \'od\'z)\index{o}{Litzmannstadt} sowie Sosnowitz\index{o}{Sosnowitz} und Bendsburg\index{o}{Bendsburg} (Będzin) betrachtet werden, da ein erheblicher Teil der Görlitzer KZ-Häftlinge aus diesen Gegenden stammte. An dieser Stelle unerwähnt, deshalb jedoch nicht weniger bedeutend, bleiben die Städte Krenau (Chrzanow, Polen)\index{o}{Krenau}, Husst (Chust, Karpato-Ukraine unter ungarischer Verwaltung)\index{o}{Husst}, Dombrowa (Dąbrowa Górnicza, Polen)\index{o}{Dombrowa}, Hindenburg (Zabrze, Polen)\index{o}{Hindenburg}, Jaworzno (Polen)\index{o}{Jaworzno},Plaszow\index{o}{Plaszow} (Vorort von Krakau, Polen), Radom (Polen)\index{o}{Radom}, Oppeln (Opole, Polen)\index{o}{Oppeln} sowie Tarnopol (Polen)\index{o}{Tarnopol} und Warschau\index{o}{Warschau}, aus denen ebenfalls Menschen jüdischen Glaubens nach Görlitz verschleppt wurden. Ihre genaue Zahl läßt sich nicht mehr bestimmen.

\subparagraph{Ghetto Litzmannstadt (\L \'od\'z, Polen)\index{o}{Litzmannstadt}}
Im Dezember 1939 befahl man die Errichtung eines Ghettos im \L \'od\'zer\index{o}{Litzmannstadt} Armenviertel Baluty. Im Januar des darauf folgenden Jahres wies man annähernd 164.000 Juden der Stadt in das knapp vier Quadratkilometer große Ghetto ein, so dass sich die Bevölkerungsdichte versiebenfachte. Bis 1942 stieg die Zahl der Bewohner auf über 200.000 an. Etwa ein Viertel von ihnen starb an Hunger, Kälte und Krankheiten; am meisten betroffen waren Kinder und alte Menschen. Die ersten Deportationen erfolgten bereits im Dezember 1940 in Zwangsarbeiterlager bei Posen (Poznan, Polen)\index{o}{Posen}, ab Januar 1942 dann direkt in das Vernichtungslager Kulmhof (Che\l mno, Polen)\index{o}{Kulmhof}, darunter waren auch viele Sinti und Roma. In der Folgezeit entwickelte sich das Ghetto mehr und mehr zu einem Zwangsarbeiterlager, indem 90 Prozent der verbliebenen 77.000 Insassen beschäftigt waren. Im Frühjahr 1944 begann man das Ghetto zu räumen und schickte abermals Tausende nach Kulmhof (Che\l mno, Polen). Zwischen dem 7. und 30. August erfolgten Transporte nach Auschwitz\index{o}{Auschwitz}, von wo aus unter anderem auch eine Gestellung nach Görlitz erfolgte. Im Herbst 1944 brachte man die letzten 730 Menschen aus dem Ghetto mit Lastwagen nach Deutschland\footnote{Israel Gutman (Hrsg.): Enzyklopädie des Holocaust, Band 2, S. 892ff.}.

\subparagraph{Sosnowitz\index{o}{Sosnowitz} (Sosnowiec) und Bendsburg\index{o}{Bendsburg} (Będzin)} Die Städte im Südwesten Polens hatten ein ähnliches Schicksal. In Sosnowitz lebten zu Kriegsbeginn etwa 28.000 Juden und in Bendsburg\index{o}{Bendsburg} etwa 27.000 (über 50 Prozent der Bevölkerung). Drei Tage nach Kriegsbeginn wurden beide Städte durch die Wehrmacht besetzt. Die jüdische Bevölkerung litt unter gewaltsamen Übergriffen durch die Deutschen. Ihre Bewegungsfreiheit wurde eingeschränkt, ihr Vermögen beschlagnahmt oder enteignet und ihre Synagogen zerstört. Nach der Gründung der Judenräte entstand wenig später in Sosnowitz die \glqq Zentrale der Jüdischen Ältestenräte in Oberschlesien\grqq, die etwa 45 Gemeinden vertrat. Die ihr unterstellten Judenräte zeigten sich mitverantwortlich für die ersten Deportationen in die Arbeiterlager der Organisation Schmelt\footnote{Der größte Betrieb, der in Bendsburg\index{o}{Bendsburg} Zwangsarbeiter beschäftige, gehörte der SS und wurde von einem menschenfreundlichen Mann namens Alfred Roßner geleitet, der seine jüdischen Angestellten von der Organisation Schmelt zugewiesen bekam und dadurch vor der Deportation rettete. Vgl. Kitia Altman: Die Hoffnung auf das Überleben -- Alfred Roßners Hilfe für die Juden in Będzin. Erschienen in Wolfgang Benz (Hrsg): Dachauer Hefte Jahrgang 20, S. 194ff.}. Des Weiteren mußten sie bei der Errichtung der Deutschen Werkstätten helfen, in denen man Juden beschäftigte. Es ist umstritten, ob der Judenrat durch Kooperation und Bereitstellung von Arbeitskräften darauf spekulierte, die Juden in ihrer Stadt retten zu können. Es folgten im Mai und Juni (nur in Sosnowitz\index{o}{Sosnowitz}) 1942 Transporte nach Auschwitz\index{o}{Auschwitz}. Nach einer groß angelegten Selektion auf dem Sosnowitzer\index{o}{Sosnowitz} Marktplatz am 1. August 1942 überstellte man 8.000 Menschen nach Auschwitz\index{o}{Auschwitz}, um sie dort umzubringen. In Bendsburg\index{o}{Bendsburg} schickte man innerhalb von sechs Tagen bei einer vergleichbaren Aktion 5.000 Menschen in den Tod nach Auschwitz\index{o}{Auschwitz}. Insbesondere seitens der zionistischen Jugendorganisation regte sich beiderorts vermehrt Widerstand gegen die Anweisungen des Judenrats. Im Frühjahr 1943 brachte man die verbliebenen Juden in ein Ghetto im Vorort Srodula. Die Bendsburger\index{o}{Bendsburg} Juden sperrte man in ein Ghetto bei Kamionka\index{o}{Kamionka}. Beide Ghettos wurden kurz darauf zusammengeschlossen. Beim Versuch die Einwohner des Ghettos im August zu deportieren, stießen die Deutschen auf vergeblichen Widerstand der Jugendorganisation und einiger anderer. Hunderten gelang die Flucht nach Ungarn und in die Slowakei\footnote{Israel Gutman (Hrsg.): Enzyklopädie des Holocaust, Band 2, S. 892ff.}, doch der Großteil gerät in deutschen Konzentrationslagern; auffällig viele auch ins KZ-Außenlager Görlitz, wo allein neun von ihnen bis zum Februar 1945 verstarben \footnote{Einäscherungsbücher Friedhofsverwaltung Görlitz}.

\paragraph{Ungarn}\label{ungarn} Durch die Annektierungen von Felvid\'ek\index{o}{Felvid\'ek} (November 1938), Transkarpatien\index{o}{Transkarpatien} (März 1939) von der Tschechoslowakei und dem nördlichen Siebenbürgen\index{o}{Siebenbürgen} (August 1940) von Rumänien sowie B\'acska\index{o}{B\'acska} von Jugoslawien kam es zu zahlreichen Gebietserweiterungen und somit auch zu einem beachtlichen Bevölkerungszuwachs in Ungarn. Darunter befanden sich auch sehr viele Juden, über deren Schicksal fortan in Budapest\index{o}{Budapest} entschieden wurde. Ungarn trat selbst auf der Seite Deutschlands am 27. Juni 1941 in den Krieg ein, widersetzte
sich jedoch zwischen März 1942 und März 1944 unter der Regierung Mikl\'os K\'allay\index{p}{K\'allay, Mikl\'os } der deutschen Forderung nach der Endlösung der Judenfrage. Dieser zeitweilige Schutz der ungarischen Juden galt jedoch nicht für heimatlose ausländische Juden, die sich im Land aufhielten. Sie wurden auf Geheiß des KEOKH (Nationales Fremdenkontrollbüro) interniert und später der SS übergeben. Darüber hinaus wurden in Kamenetz-Podolski\index{o}{Kamenetz-Podolski} und Novi Sad\index{o}{Novi Sad} insgesamt 24.600, darunter auch ungarische Juden, massakriert. Schätzungsweise 40.000 starben im militärischen Arbeitsdienst an der ukrainischen Front.

Nach der deutschen Besatzung im März 1944 deportierten der Sicherheitsdienst (SD) und die SS hunderttausende Juden aus der ungarischen Provinz, insbesondere auch aus dem Karpatenraum und Siebenbürgen\index{o}{Siebenbürgen}, nach Auschwitz\index{o}{Auschwitz}\footnote{Adolf Eichmann: Götzen, S. 457-460.}.
\newpage
\paragraph{Übriges Europa} Menschen aus Süd- und Westeuropa waren eine Minderheit im KZ-Außenlager Görlitz. Über Franzosen und Italiener ist lediglich in wenigen Überlebendenberichten die Rede. Der SS-Mann Werner Weiss sagte sogar aus, dass Engländer oder zumindest Englischsprechende im Lager inhaftiert waren. Ein griechischer Staatsbürger wird ausdrücklich vom Lagerschreiber\index{p}{Schiffer, Emmrich} Emmrich Schiffer\index{p}{Schiffer, Emmrich} in einem Transport aus Groß-Rosen\index{o}{Groß-Rosen} erwähnt. In Übereinstimmung mit den Aussagen Sam Weinrybs\index{p}{Weinryb, Sam} und den Einäscherungsbüchern der Görlitzer Friedhofsverwaltung war mindestens ein Niederländer unter den Gefangenen\footnote{Sam Weinryb erinnert sich an zwei \glqq Holländer\grqq. LArchB B Rep 058 Bd. 6. Siehe auch die Eintragung vom 13. November 1944 im Einäscherungsbuch der Friedhofsverwaltung Görlitz (FVG), wo der am 7. Juni 1908 in Amsterdam geborene und am 2. November 1944 verstorbene Moses Dormitz verzeichnet ist. Seine Urne verblieb in Görlitz. Der Geburtsort lässt nicht mit Sicherheit auf die Nationalität schließen, gibt jedoch im Zusammenhang mit Sam Weinryb Anlass zu dieser Aussage.}. Die Tschechoslowakin Anna Hyndr\'akov\'a kam Anfang 1945 mit zwei weiteren Landsleuten ind Frauenlager in Görlitz.

%%%%%%%%%
\subsubsection{Andere Häftlingsgruppen}%Politische und moralisches Überzeugung}
Insbesondere die \glqq politischen Gefangenen\grqq~und die sogenannten\glqq Kriminellen\grqq, also jene mit rotem und grünem Winkel als Erkennungszeichen, bildeten eine absolute Ausnahme im Lager Görlitz.~\newline
Der Deutsche Fritz Wolff\index{p}{Wolff, Fritz} war aus politischen Gründen zunächst in verschiedenen Polizeigefängnissen und 1944 im KZ Groß-Rosen\index{o}{Groß-Rosen} inhaftiert bevor er für kurze Zeit ins Arbeitskommando nach Kunnerwitz\index{o}{Kunnerwitz} kam\footnote{LArchB B Rep 058 Bd. 4.}. Nachweislich galt lediglich der Lagerälteste\index{p}{Czech, Hermann} als \glqq Krimineller\grqq~(siehe S.~\pageref{czech}). Es ist nicht bekannt, ob es noch weitere Personen gab, die aus nicht rassistischen Gründen im Außenlager Görlitz festgehalten wurden.
\newline
Auch der Versuch Gruppen anhand der verschiedenen Bildungsniveaus zu identifizieren, erübrigt sich fast, denn man kann davon ausgehen, dass sich unter den deportierten Menschen nicht weniger gebildete Leute befanden, als in den Gegenden ihrer Herkunft. Unter ihnen waren z.B. Rechtsanwälte, Uhrmacher, Graveure, Elektriker, Krawattenmacher und Ärzte. Also ein recht breites Spektrum ausgebildeter Leute und keine bestimmte Berufsgruppe oder Klasse.

%%%%%%%%%%%%%%%%%%%%%%%%%%%%%%%%%%%%%%%%%%%%%%%%
\subsection{Häftlingsselbstverwaltung}
Die Lagerverwaltung stellte ihr Regime auf eine breite Grundlage und übertrug erhebliche Vollmachten an eine kleine Gruppe von Häftlingen. Dadurch entstand zwischen dem SS-Personal und der Häftlingsgesellschaft eine Instanz aus Komplizen, ohne welche die Aufrechterhaltung der Lagerordnung nicht möglich gewesen wäre. Jene Helfershelfer erfüllten Aufsichts- und Ordnungsfunktionen genauso wie Verwaltungs- und Versorgungsfunktionen. Der Aufgabenbereich der Häftlingsselbstverwaltung war eindeutig gegliedert und einer eigenen Hierarchie unterworfen. Seitens der Lagerverwaltung ergab sich dadurch eine erhebliche Arbeitserleichterung sowie eine Verringerung des Machtaufwandes\footnote{Vgl. Wolfgang Sofsky: Die Ordnung des Terrors. Das Konzentrationslager, S. 152f.}. Diese so genannten Funktionshäftlinge verpflichteten sich zu unbedingtem Gehorsam und erhielten dafür befristeten Verfolgungsschutz, sowie eine Reihe von anderen Vergünstigungen, wie etwa eine bessere Versorgung oder Unterkunft.

Im Gegensatz zur fast allgemeinen körperlichen Schwerstarbeit und den daraus resultierenden Folgen für die Gesundheit, hatten die Funktionshäftlinge nur leichte oder gar keine körperlichen Arbeiten zu verrichten\footnote{Vgl. Isabell Sprenger: Groß-Rosen, S. 138.}. Durch diese bevorzugte Behandlung, insbesondere die Unterbringung außerhalb der Blocks, entstand eine Kluft zwischen diesen \glqq elitären\grqq~und den \glqq normalen\grqq~Gefangenen\footnote{Wolfgang Sofsky schreibt hierzu auf S. 137 in seinem Buch \glqq Die Ordnung des Terrors. Das Konzentrationslager\grqq: Während Unzählige im Elend verhungerten, führten wenige Häftlinge ein geradezu luxuriöses Leben. Während viele an körperlicher Plackerei zugrunde gingen, brauchten andere gar nicht zu arbeiten. Während die meisten in ständiger Angst vor Gewalt lebten, konnten einige ungestraft quälen und töten.}.\newline
Im Lager Görlitz bildeten der Lagerälteste\index{p}{Czech, Hermann}, die Lagerkapos, Arbeitskapos\footnote{Nach Weinmann u.a. ist der Ausdruck \glqq Kapo\grqq~die Abkürzung der nationalsozialistischen Wortschöpfung \glqq Kameradschaftspolizei\grqq, vgl. M. Weinmann (Hg.): Das nationalsozialistische Lagersystem, Frankfurt a. M. 1990. Siehe auch: Isabell Sprenger: Groß-Rosen, S. 138. Vgl.: A. Kaminski: Konzentrationslager von 1896 bis heute, S. 247, Anmerkung 163.} und Blockältesten\footnote{Gefolgt von den weniger bedeutsamen Stubenältesten.} die Spitze der Häftlingsverwaltung.

Die Besetzung dieser Stellen hatte einen wesentlichen Einfluss auf den \glqq Lageralltag\grqq~und somit auf die Haftbedingungen. In Überlieferungen von ehemaligen Görlitzer Häftlingen finden sich keine Anzeichen für eine Zusammenarbeit der Funktionshäftlinge, die auf eine Verbesserung der Existenzbedingungen abzielte. Im Gegenteil: Bis auf wenige Ausnahmen wird die Spitze der Häftlingsselbstverwaltung mit den schlimmsten Folterungen und Mißhandlungen in Verbindung gebracht.
\newline Abram Rajchbart\index{p}{Rajchbart, Abram} beschreibt diese Situtation so:
\begin{leftbar}
Unter den Vorstehern gibt es [...] Leute, die um das Vertrauen der Deutschen zu erwerben, ihre Untergebenen misshandeln und foltern.
Diese Folterkunst bringen sie zur Vollkommenheit, es wird für sie eine Sportart, in der jeder seinen eigenen Stil hat.\footnote{Abram Rajchbart, ZIH 301/715.}
\end{leftbar}
Grund für die mangelnde Kooperation und Organisation der Funktionshäftlinge im Sinne der Gefangenen mag die von der SS beabsichtigte Zusammensetzung derselben gewesen sein. Während fast 95 Prozent der Häftlinge aus Ost- oder Südosteuropa nach Görlitz deportiert wurden, setzte man fast ausschließlich deutschstämmige Juden oder mit einem grünen Winkel gekennzeichnete \glqq Berufsverbrecher\grqq\footnote{Der Begriff des \glqq Berufsverbrecher\grqq~wurde bereits in der Weimarer Republik eingeführt und bezeichnete einen Wiederholungstäter, von dem angenommen wird, dass er das Verbrechen als Beruf ausübt. Gegen Berufsverbrecher wurde in den meisten Fällen \glqq vorbeugende Haft\grqq, ab 1937 jedoch die Einlieferung in Konzentrationslager verhängt. BV war ursprünglich die Abkürzung für \glqq Befristete Vorbeugehäftlinge\grqq, später für Berufsverbrecher. In diesem Zusammenhang sind die Begriffe \glqq Krimineller\grqq~oder \glqq Verbrecher\grqq~durch die nationalsozialistische Gesetzeslage definiert. Ausführlicheres dazu findet sich in: Gellately, Kapitel 1-3.} in den entscheidenden Positionen ein\footnote{Vgl. Isabell Sprenger: Groß-Rosen, S. 139.}. Da die zur Verfügung stehenden deutschen Häftlinge nicht ausreichten, bezog die die SS auch ungarische oder polnische Gefangene mit ein. Doch galt ihr Verantwortungsbereich einer anderen Nation unter den Häftlingen, anstatt ihrer eigenen.
Diese geschaffenen Hierarchien widersprachen zudem jeglichen Prinzipien einer zivilen Gesellschaftsordnung. So stand der Gesetzlose über dem Rechtsanwalt, der Ungelernte über dem Meister. Für viele war dies ein Schock.
Der SS gelang es durch die Schaffung einer Hierarchie in der Häftlingsgesellschaft einzelne Häftlinge und Häftlingsgruppen gegeneinander auszuspielen.
Der Historiker Hermann Langbein\footnote{Hermann Langbein war selbst mehrere Jahre im KZ Buchenwald inhaftiert.} schreibt dazu, dass nicht jeder Häftling im Stande war, diesen Zusammenhang zu überblicken und statt dessen dazu neigte, seine Peiniger vom System losgelöst zu betrachten\footnote{Vgl. Langbein: Widerstand in den nationalsozialistischen Konzentrationslagern, S. 36f.}. Sobald sich ein Funktionshäftling durch Gebrauch seiner Macht in den Augen seiner Mitgefangenen schuldig machte, war er der SS verfallen und musste alles daran setzen, seine Position zu halten, um der Rache des Lagers zu entgehen.
Samuel Kessler, der ehemalige Blockälteste von Block 5 und später Block 6, drückte dies so aus:
\begin{leftbar}
Wenn du heute nicht auf die Häftlinge einschlägst, bist du morgen selber dran.\footnote{Aussage von Samuel Kessler. LArchB B Rep 058 Bd 1.}
\end{leftbar}
%\textcolor{red}{Hieraqchiebild: Lagerältester >> Lagerkapo >> Blockältester (schwarze Armbinde) %Kapo - BV: moralisch labil, oftmals von der Gesellschaft als Außenseiter verachtet, erhielten sie %eine unvorstellbar große Macht.}
%\label{hierarchie}~\ref{hierarchie}

%%%%%%
\paragraph{Der Lagerälteste}\label{czech} Die Position des Lergerältesten galt als verantwortlicher Vertreter der Häftlinge gegenüber der Lagerführung und gleichzeitig höchster Befehlsempfänger unter den Häftlingen. Da er nicht von den Häftlingen, sondern von der SS eingesetzt wurde und ihr diente, erfüllte er die Funktion als Repräsentant der Häftlinge somit nicht in deren Interesse, sondern ausschließlich im Sinne der SS, die ihn des Öfteren bei Problemen und zur Aufrechterhaltung von Ordnung und Disziplin heranzog\footnote{Vgl. Sofsky: Die Ordnung des Terrors. Das Konzentrationslager, S. 153.}.
\newline
Im Lager Görlitz gab es zwei Lagerälteste. Die aus Wien stammende Jüdin Stella Beck\index{p}{Beck, Stella} übte diese Funktion im Frauenlager aus. Über sie ist nur wenig überliefert, doch spricht neben der geringen Sterblichkeit in ihrem Teil des Lager vieles dafür, dass sie sich gegenüber den Gefangenen human verhielt und sich um deren Wohlergehen bemühte.
\newline
Im Männerlager war die Situation eine andere.
Der wegen Mordes verurteilte Kaufmann Hermann Czech\index{p}{Czech, Hermann} (*27. Oktober 1892 in Breslau\index{o}{Breslau}) wurden von Groß-Rosen\index{o}{Groß-Rosen} als Lagerältester\index{p}{Czech, Hermann} für Görlitz bestimmt. Er saß schon 12 Jahre wegen Mordes an seiner Ehefrau in Haft und wurde zur Besserung ins KZ überstellt. Mit dem ersten Häftlingstransport kam er am 10. August 1944\footnote{Emmrich Schiffer. PMGR 4702/14/DP.} von Groß-Rosen\index{o}{Groß-Rosen} nach Görlitz\footnote{Aussage von Samuel Kessler. LArchB: B Rep 058 Bd. 1.} und bewohnte als Einzelner ein Zimmer neben dem Krankenrevier. Ihm allein war es gestattet, auch das Frauenlager zu betreten. In nahezu allen Berichten von Überlebenden wird er als der schlimmste Unmensch und Folterer des KZ-Außenlager Görlitz beschrieben.

\begin{leftbar}
Er [Hermann Czech\index{p}{Czech, Hermann}] misshandelte seine Untergebenen auf eine Weise, dass er sie mit dem ersten Schlag gegen den Kopf zu Boden warf und danach durch Fußtritte in den Bauch und in die Herzgegend dem Opfer den Rest gab.\footnote{Abram Rajchbart, ZIH 301/715.}
\end{leftbar}

Als man ihm im Jahre 1948 in Polen den Prozess machte, bezeichnete man ihn in der Urteilsbegründung als Sadisten\footnote{LArchB: B Rep 058 Bd. 2.}. Die Liste seiner Opfer ist lang; hier nur ein kurzer Auszug aus den Anklagepunkten, in denen man ihn für schuldig befand und zum Tode verurteilte:
\begin{itemize}
\item{\glqq Ein Häftling namens Miedzinski stahl Kartoffelschalen aus der Küche. Nachdem Czech\index{p}{Czech, Hermann} ihn im Hof verprügelte, sollte Miedzinski die Schalen aufessen.\grqq\footnote{ebenda.}}

\item{\glqq Im [Kranken-]Revier forderte Czech\index{p}{Czech, Hermann} die Kranken auf, in andere Teile der Baracke zu laufen, währenddessen er auf sie einschlug. Oftmals schlug er die Kranken grundlos. Bei der Gelegenheit teilte er die Kranken zur Arbeit ein, wodurch er die Sterblichkeit unter den Häftlingen erhöhte.\grqq\footnote{ebenda.}}

\item{\glqq Czech\index{p}{Czech, Hermann} ging so weit, dass er den Kranken befahl, sich in kaltes Wasser und Dreck zu legen -- stundenlang -- der Tod war vorprogrammiert.\grqq\footnote{ebenda.}}

\item{\glqq Czech\index{p}{Czech, Hermann} hetzte dressierte Hunde auf Häftlinge, so dass Fleischstücke herausgerissen wurden.\grqq\footnote{ebenda.}}

\item{\glqq Ein aus Sosnowiec stammender Häftling lag als Lungenkranker in der Krankenstube und hustete ununterbrochen einige Stunden. Daneben besaß Czech\index{p}{Czech, Hermann} seine Stube. Am Abend stürzte er mit einem Stock ins Revier schmiss ihn von der Pritsche und prügelte auf ihn ein, [so] dass er nach 20 Minuten starb. Die restlichen Kranken prügelte er ohne Kleidung und Holzschuh aus der Baracke in die Kälte. Drei von ihnen wurden in den darauf folgenden Tagen wieder in die Krankenstube eingewiesen, und verstarben kurze Zeit später. Unter ihnen ein Häftling namens Dab.\grqq\footnote{ebenda.}}

\item{\glqq Die Häftlinge starben durch [...] Czech\index{p}{Czech, Hermann}, der einen Stacheldraht in seine Lederpeitsche eingenäht hatte und damit Leuten in die Nieren schlug. Nach drei bis vier Tagen sind sie wegen Nierenbluten verstorben.\grqq\footnote{ebenda.}}

\end{itemize}

Doch damit nicht genug, der ehemalige Lagerschreiber\index{p}{Schiffer, Emmrich} Emmrich Schiffer\index{p}{Schiffer, Emmrich} erhebt noch weitere Vorwürfe gegen ihn:
\begin{leftbar}
Hermann Czech\index{p}{Czech, Hermann} ist zu verdanken, dass von 1750 Lagerstand 364 Häftlinge an Hungersnot verstorben sind, da er immer mindestens die Hälfte der Fassung weggestohlen hat und es durch die deutschen Meister, die im Lager verschiedene Arbeiten verrichteten, in die Stadt heraus schickte. Dafür brachten diese ihm verschiedene Schnäpse und diese hat er teils dem Lagerführer\index{p}{Zunker, Winfried} SS-Oberscharführer Winfried Zunker\index{p}{Zunker, Winfried} und dem Lagerkommandanten\index{p}{Rechenberg, Erich} SS-Obersturmführer Rechenberg\index{p}{Rechenberg, Erich} abgetreten und sie dadurch bestochen, dass sie ihm ausgeliefert waren. Hermann Czech\index{p}{Czech, Hermann} wurde durch die wachhabenden SS-Leute als \glqq Kleinkönig\grqq~bezeichnet, weil durch obige Fälle das geschah, was dieser abnormale Sadist und Verbrecher wollte. Er hat alle älteren SS-Leute korrumpiert, so dass auch die gegen ihn nichts machen konnten, durch das er doch die Führung erkauft hatte.\footnote{Emmrich Schiffer gibt zu dieser Behauptung eine Reihe von Zeugen seitens der SS-Wachmannschaften an und verweist zu dem auf seinen Kollegen Fritz Strauß, dem Arbeitseinsatzschreiber. PMGR 4702/14/DP.}
\end{leftbar}
Diese Anschuldigungen konnten bisweilen von keinem weiteren Häftling in dieser Form bestätigt werden, da wahrscheinlich kaum ein anderer einen besseren Einblick in das Lagergeschehen haben konnte, als der Lagerschreiber\index{p}{Schiffer, Emmrich} persönlich. Nachweislich ertappte man Czech\index{p}{Czech, Hermann}, wie er nach dem Evakuierungsmarsch im März 1945 mehrfach Lebensmittel ins Frauenlager trug. Dadurch verringerte sich die Essensration im Männerlager, woraufhin die SS veranlasste, Czech\index{p}{Czech, Hermann} im März 1945 in das Außenlager Zittau (heute: Sieniawka, Polen)\index{o}{Zittau}\footnote{Aussage von Heniek Zimberknopf, Samuel Kessler, Emmrich Schiffer, Fritz Strauß. LArchB: B Rep 058 Bd. 1-6.} und dann weiter ins Außenlager Reichenau (Rychnov u Jablonce nad Nison, Tschchien) zu verlegen\footnote{Vgl. Pavla Placha und Andrea Rudorff: Reichenau (Rychnov u Jablonce nad Nisou), in: Wolf­gang Benz / Bar­bara Die­s­tel (Hgs.): Orte des Ter­rors. Ge­schichte der na­tional­so­zia­lis­ti­schen Kon­zen­tra­ti­ons­la­ger Band 6. Natz­wei­ler. Groß-Rosen. Stutt­hof. S. 421-426.} (siehe auch Seite~\pageref{czech_ahndung}).

%%%%%%
\paragraph{Lagerkapos}
In der Häftlingshierarchie standen die Lagerkapos direkt unter dem Lagerältesten\index{p}{Czech, Hermann} und waren somit befugt, den Kapos und Blockältesten Weisungen zu erteilen.
Im Lager Görlitz übte der Pole Jacob Tannenbaum\index{p}{Tannenbaum, Jacob} und sein Landsmann Schneebaum\index{p}{Schneebaum}\footnote{Vgl. Schlomo Graber: Schlajme, S. 81. Aussage von Samuel Kessler. LArchB B Rep 058 Bd. 1.} diese Funktion aus (siehe auch S. ~\pageref{kapos_ahndung}). Über Tannenbaum gibt es eine beträchtliche Anzahl an Quellen, während von Schneebaum nicht mehr als sein Nachname und seine medizinische Profession\footnote{Vgl. Testimonium von Henryk Vogler, S. ~\pageref{vogler}} überliefert ist.

Jacob Tannenbaum kam am 13.08.1912 als zweitjüngstes Kind einer 11-köpfigen Familie in Sieniawa/Polen\index{o}{Sieniawa} zur Welt. Tannenbaum arbeitete außerhalb der Stadt und engagierte sich in der zionistischen Jugendorganisation Betar. 1941 hatte er mit seiner Frau Bornia ein Kind. Im gleichen Jahr besetzte die Wehrmacht die Gegend um Sieniawa. Als Einziger überlebte er eine Massenerschießung von 30 bis 40 Juden während eines Arbeitseinsatzes nahe der Stadt. Tannenbaum musste mit ansehen, wie die Deutschen den Leichnam seines Vater auf einem Lastwagen abtransportierten. Seine Frau und sein Kind, sowie seine Mutter wurden von den Deutschen in einem nahegelgenen Waldstück erschossen. In der Folgezeit durchlebte Tannenbaum mehrere Arbeitslager, zuletzt das in Görlitz, wo er wahrscheinlich wegen seiner blnoden Haare und seiner kräftigen, großen Statur zum Lagerkapo ernannt wurde.
Tannenbaum hielt sich im Lager einen chassidischen Jungen, für dessen Kleidung, Essen und Schutz er sorgte. Einige Mitgefangene sahen die beiden in einer Vater-Sohn-Beziehung stehen, andere sahen den Jungen als Tannenbaums Haustiers.\footnote{David van Biema: Poisoned Lives.}
Gemeinhin belegen einige Stimmen Tannenbaums brutalen Umgang mit den Häftlingen:
\begin{leftbar}
Eines Tages entdeckte er [Tannenbaum\index{p}{Tannenbaum, Jacob}], dass Vater, der sehr unter der furchtbaren Kälte litt, sich unter der Kleidung eine Wolldecke um den Leib gewickelt hatte. Tannenbaum führte ihn daraufhin an einen Block, befahl ihm, die Hose runter zu lassen und sich zu bücken, und schlug ihn mit 25 Peitschenhieben blutig.\footnote{Vgl. Schlomo Graber: Schlajme, S. 81.}
\end{leftbar}
Leon Zelig\index{p}{Zelig, Leon} berichtete, wie sein Vater Mozes Zelig\index{p}{Zelig, Mozes} als Rabbi am 12. Oktober 1944 anlässiglich eines jüdischen Feiertags (Sukkot/Laubhüttenfest) zehn Männer zum Minyan (Gebet) im Waschraum versammelte und deswegen von Tannenbaum zu Tode geprügelt wurde\footnote{David van Biema: Poisoned Lives.}.
%\begin{leftbar} Zelig's father's crime: He was a rabbi. At the time of his death, he was trying to collect a minyan, the minimum 10 Jews necessary for a prayer service, in the camp washroom. Tannenbaum beat him to death. "I couldn't say kaddish {the prayer for the dead} properly," says Zelig with sad irony. "I couldn't get a minyan. So I said kaddish by myself."\end{leftbar}


%%%%%%
\paragraph{Blockälteste und Arbeitskapos}
Ein Block bezeichnet eine Wohneinheit der Gefangenen. Ein Blockältester war gegenüber der Lagerleitung in seinem Block für Disziplin, Ordnung und Durchführung aller Befehle verantwortlich\footnote{Vgl. Langbein: Widerstand in den nationalsozialistischen Konzentrationslagern, S. 31.} (siehe Tabelle auf S.~\pageref{blockaeltesten}). Auch die Zuteilung von Essen und sauberer Kleidung, sowie andere hygienische Maßnahmen oblagen ihm\footnote{Ebenda und vgl. Sofsky: Die Ordnung des Terrors. Das Konzentrationslager, S. 154.}. Wer diesen Anforderungen nicht mit ausreichender Härte und Brutalität nachkam, wurde durch einen neuen Blockältesten ersetzt. So erging es Samuel Kessler\index{p}{Kessler, Samuel}, den man im Dezember 1944 seiner Funktion als Blockältester enthob und daraufhin als Fuhrmann einsetzte\footnote{Aussage von Samuel Kessler. LArchB: B Rep 058 Bd. 1.}.
~\newline
Bei Ankunft eines Häftlingstransportes fragte der Lagerälteste\index{p}{Czech, Hermann} [Hermann] Czech\index{p}{Czech, Hermann} während eines Appells, wer gut schlagen könne. Laut Aussagen mehrerer ehemaliger Insassen meldete sich unter anderen der rumänische Jude Adolf Eichner\index{p}{Eichner, Adolf}\footnote{Aussage von Heniek Zimberknopf, Israel Braun und Gedalia Pilo. LArchB: B Rep 058 Bd. 1.}. Eichner selbst behauptete später, dass ihn Czech\index{p}{Czech, Hermann} dabei ertappte, wie er eine Flasche Wein ins Lager schmuggelte und er ihn nur deshalb beförderte, um zukünftig solcherlei Ware zu beschaffen\footnote{YV: TR-10/844.}. Samuel Kessler, ein deutscher Jude, wählte man aufgrund seiner Herkunft und Sprachkenntnis als Dolmetscher und Vorarbeiter bzw. Kapo für die WUMAG aus\footnote{Aussage von Samuel Kessler. LArchB: B Rep 058 Bd. 1.}. Gleichzeitig wurde er zum Blockältesten ernannt.\newline

Aus dem Frauenlager ist nur eine, durch Schlomo Graber\index{p}{Garber, Schlomo} als human beschriebene, Blockälteste namens Izsak\index{p}{Izsak} bekannt. Sie stammte aus der ungarischen Stadt Kolozsvar\index{o}{Kolozsvar}.
\newline
Nach außen hin kenntlich zeigten sich die Blockältesten durch eine schwarze Armbinde\footnote{Aussage von Gerszon Sobotka und Samuel Kessler\index{p}{Kessler, Samuel}. LArchB B Rep 058 Bd. 1.}, die Kapos trugen eine weiße Armbinde. Samuel Kessler\index{p}{Kessler, Samuel} in seiner Doppelfunktion trug jedoch nur die schwarze Armbinde des höher gestellten Ranges\footnote{Ebenda.}.
\newline
Es ist anzunehmen, dass fast alle Blockältesten im Lager Görlitz auch die Funktion eines Arbeitskapos ausübten, nicht jedoch umgedreht. Die Arbeitskapos hatten dafür Sorge zu tragen, dass die Arbeitsgruppen zur richtigen Zeit am Arbeitsplatz eintrafen\footnote{Aussage von Gerszon Sobotka. LArchB B Rep 058 Bd. 1.} und ihr vorgegebenes Soll erfüllten. Im wesentlichen bestimmten sie auch, wer welche Arbeit zu verrichten hatte. Die meisten waren nicht einmal für ihre Rolle als Vorarbeiter geschult oder kompetent genug, bestimmte Arbeiten selbst zu verrichten. Damit wird deutlich, dass die Arbeit nicht das Hauptziel der Konzentrationslager darstellte, sondern lediglich als Mittel zum Zweck der seelischen und geistigen Zermürbung der Gefangenen unter Zuhilfenahme der Funktionshäftlinge diente.

Die ermittelten Namen der Blockältesten je Block\footnote{LArchB: B Rep 058 Bd. 1-9.}:

\begin{description} %{ p{.15\linewidth}  p{.85\linewidth}}
	\item[Block 1:] Szilit oder Schilit
	\item[Block 2:] Rosenfeld, ungarischer Abstammung
	\item[Block 3:] Schönfeld
	\item[Block 4:] Gershon Rosenberg
	\item[Block 5:] Posnanski
	\item[Block 6:] Samuel Kessler\index{p}{Kessler, Samuel}, aus Köln (bis Dezember 1944); Abraham Wolkowitz, aus \L \'od\'z
	\item[Block 7:] Jehuda Widawski, aus \L \'od\'z
	\item[Block 8:] Adolf Eichner\index{p}{Eichner, Adolf}, aus Klausenburg / Rumänien
	\item[Block 9:] Manfred Rendsburg, aus Hamburg
\end{description}
\label{blockaeltesten}


Beide hier angesprochenen Gruppen von Funktionshäftlingen hatten relativ warme Kleidung, ihre eigene abgetrennte Ecke in den Blöcken und auch genug zu essen. Es wird da\-rüber hinaus deutlich, welch enorme Macht diese Häftlinge gegenüber andern hatten. Sie konnten entscheiden, Schwache und Kranke bei der Arbeit zu schonen oder ihnen größere Essensrationen zuzuteilen. Leider sind uns nur gegenteilige Fälle aus Überlieferungen von Überlebenden bekannt, wenngleich es auch positive Einflussnahmen seitens der Kapos und Blockältesten gegeben haben mag.
\newline
Der Blockälteste\label{eichner} Adolf Eichner\index{p}{Eichner, Adolf} wurde am 10. Mai 1922 in Bukarest\index{o}{Bukarest} als Sohn eines jüdischen Kaufmanns geboren, ging nach seiner Schulzeit in Klausenburg\index{o}{Klausenburg} zu seinem Onkel nach Ungarn, um dort das Schlosserhandwerk zu erlernen. Nach eigenem Bekunden hat man ihn bereits im November 1939 von dort aus nach Auschwitz\index{o}{Auschwitz} deportiert (vgl. S.~\pageref{ungarn}). Die langjährige Haft in den KZ-Außenlagern Otmuth-Krappitz (Krapkowice, Polen)\index{o}{Otmuth-Krappitz}, Gross-Same\footnote{Der Ort konnte bisweilen noch nicht genau lokalisiert werden. Fest steht, dass die Häftlinge für die Firma Weiss \& Freitag arbeiteten.},\index{o}{Gross-Same}~Hermannsdorf (Męcinka, Polen)\index{o}{Hermannsdorf}, Bunzlau (Boles\l awiec, Polen)\index{o}{Bunzlau} und schließlich Fünfteichen (Mi\l oszyce, Polen)\index{o}{Fünfteichen}\footnote{Dort errichteten KZ-Häftlinge die sogenannten \glqq Bertha-Werke\grqq, ein Auslagerungsbetrieb der Kruppwerke.} hatte ihn bereits stark demoralisiert und zum bereitwilligen Kollaborateur der SS geformt. Es ist nicht bekannt, was ihm genau widerfahren ist, so dass er sich in Görlitz in der Funktion des Blockältesten als \glqq Schwerster Schläger\grqq~einen Namen machte (siehe auch Ermittlungen gegen Eichner, S. ~\pageref{eichner_ahndung}). Die folgenden Schilderungen zeugen von seiner unmenschlichen Grausamkeit. Israel Braun\index{p}{Braun, Israel} schildert:\label{turnen}
\begin{leftbar}
Anja \footnote{\glqq Anja\grqq lautete Adolf Eichner\index{p}{Eichner, Adolf}s Spitzname unter den Häftlingen. Angeblich rührt dieser Name von Eichners sexuellen Neigungen.} trat und schlug Häftlinge vornehmlich ins Gesicht. Er zeigte dabei einen Ausdruck von Zufriedenheit und war immer lustig.\footnote{Aussage von Israel Braun. LArchB B Rep 058 Bd. 1.}
\end{leftbar}

Leib Wyszegrodzki\index{p}{Wyszegrodzki, Leib} erinnert sich:
\begin{leftbar}
[Ein Häftlings namens] Hochmitz\index{p}{Hochmitz} wollte den Sabbat verrichten, doch Eichner\index{p}{Eichner, Adolf} sagte, dies hier sei keine Synagoge und versetzte ihm mehrere harte Schläge. Während der fünf Monate, wo ich im Block 8 gewesen bin, gab es 40 Tote. Wenn wir auch schlechtes Essen hatten, Eichner war mitschuldig wegen seines grundlosen Schlagens und seiner Sportübungen: egal ob erkältet oder mit Fieber, trieb er uns aus den Betten in den Schnee und ließ uns schwere Übungen machen. So lange es ihm passte, eine viertel oder halbe Stunde, länger hätten wir sowieso nicht ausgehalten.\footnote{Aussage von Leib Wyszegrodzki. LArchB B Rep 058 Bd. 1.}
\end{leftbar}


Heniek Zimberknopf\index{p}{Zimberknopf, Heniek} bekundet:
\begin{leftbar}
Die Häftlinge waren an diesem Abend [nach ihrer Ankunft] so ausgehungert, dass sie bei der ersten Essensausgabe versuchten zwei mal anzustellen, worauf Gerschowitz\index{p}{Gerschowitz}, Tennebaum\index{p}{Tannenbaum, Jacob} und Eichner\index{p}{Eichner, Adolf} mit Bettpfosten auf sie einschlugen. [...] Meistens schlug er mit einer Kabelisolierung. Seine Spezialität war es, mit dem Ellenbogen Häftlinge auf die Erde zu stoßen.\footnote{Aussage von Heniek Zimberknopf. LArchB B Rep 058 Bd. 1.}
\end{leftbar}

Gerszon Sobotka\index{p}{Sobotka, Gerszon} berichtet:
\begin{leftbar}
Eichner zwang Gefangene auf einen Stuhl zu treten und einen anderen Stuhl zu halten bis sie bewusstlos wurden. Jeden Sonntag nahm die SS eine Gruppe Gefangener zum Block 1, wo sie sich nackt auf den Boden legen mussten, um ausgepeitscht zu werden. Wenn die SS nicht mehr konnte, fuhr Eichner\index{p}{Eichner, Adolf} fort. [...]

Kleinmann\index{p}{Kleinmann} und Herzberg\index{p}{Herzberg} hatten ein Brot gestohlen, woraufhin sie sich auf den Boden legen mussten und Eichner\index{p}{Eichner, Adolf} mit seinen benagelten Stiefeln und Absatzeisen so lange auf ihnen herumsprang bis ihnen der Schaum vor dem Munde stand und sie besinnungslos wurden. Dann gab er ihnen noch Fußtritte, schrie nach Wasser und übergoss beide, bevor sie am selben Abend oder am darauf folgenden Tag ins Krankenrevier gebracht wurden. Nach zwei bis drei Tagen verstarb Kleinmann\index{p}{Kleinmann}. [...] \newline
Mißhandlungen machten ihm nichts aus. Es waren stets vorsätzliche Handlungen, keine Selbstverteidigung!\footnote{Aussage von Gerszon Sobotka. LArchB B Rep 058 Bd. 1.}
\end{leftbar}

Icek Brandt\index{p}{Brandt, Icek} sagte:
\begin{leftbar}
Eichner\index{p}{Eichner, Adolf} tat mehr als die Lagerleitung verlangte.\footnote{Aussage von Icek Brandt. LArchB B Rep 058 Bd. 1.}
\end{leftbar}


 \subparagraph{Küchenkapo}
Der aus Polen stammende Gustav Muskatenblut\index{p}{Muskatenblut, Gustav} führte als verantwortlicher Kapo die Häftlingsküche\footnote{Aussage von David Nechushtan. LArchB B Rep 058 Bd. 1.}. Schlomo Graber\index{p}{Graber, Schlomo} erinnert sich seiner mit folgenden Worten:
\begin{leftbar}
Beim Essenausteilen stand er mit seiner Kelle vorn und teilte jedem einen Schlag trübe Brühe aus, ein Gebräu aus Unkräutern, Steinchen und Sand, das als Suppe bezeichnet wurde. wehe dem, der es wagte, um einen Zuschlag zu bitten oder sich vor zu drängeln. Sofort bekam er eins mit der Kelle auf den Kopf und wurde als \glqq Chajess\grqq~(\glqq Biest\grqq) beschimpft.\footnote{Schlomo Graber: Schlajme, S. 82.}
\end{leftbar}

%%%%%%
\paragraph{Schreiber}
\label{schreiber}
Im Lager Görlitz unterschied man die Schreiber in Lager-, Arbeits- und Ver\-pfleg\-ungs\-schreiber, wobei alle drei sowohl für das Männerlager als auch für das Frauenlager zuständig waren. Neben den im Folgenden erwähnten Personen führten auch andere, wie etwa Henryk Vogler\footnote{Vgl. Henryk Vogler: Autoportret z Pamięci.} oder Alice Burger\footnote{Protokoll von Alice Burger, 14.08.1945, YV 015E-2476.}, Schreibtätigkeiten aus - sowohl im Lager, als auch bei der WUMAG.
In dieser Position verfügten sie über enorme Macht, die sie im Gegensatz zu manch anderen Funktionshäftlingen zugunsten der Gefangenen gebrauchten.

\subparagraph{Lagerschreiber\index{p}{Schiffer, Emmrich}}
Dem in der Tschechoslowakei geborenen \mbox{Emmrich} Schiffer\index{p}{Schiffer, Emmrich} wurde bereits im Hauptlager Groß-Rosen\index{o}{Groß-Rosen} der Posten als Lagerschreiber\index{p}{Schiffer, Emmrich} zugewiesen. Zu seinen Aufgaben gehörte im Wesentlichen die Verwaltung der Häftlingskarteien.
\begin{leftbar}
Bei jedem Tötungsfall bekam ich zuletzt den Befehl von Zunker\index{p}{Zunker, Winfried}, schriftlich Meldung nach Groß-Rosen\index{o}{Groß-Rosen} zu machen, doch immer mit einer erdachten Todesursache -- wie Herzschwäche, allgemeine Körperschwäche und ähnlichem. Die Wahl der Diagnose oblag dem Schreiber.\footnote{Aussage von Emmrich Schiffer. LArchB B Rep 058 Bd. 5. Hierbei besteht offenbar ein Unterschied zwischen der Meldung an das Hauptlager in Groß-Rosen\index{o}{Groß-Rosen} und dem Totenschein, den der Arbeitsschreiber Strauß \index{p}{Strauß, Fritz} auszufüllen hatte.}
\end{leftbar}
Emmrich Schiffer\index{p}{Schiffer, Emmrich} mußte auch Transportlisten erstellen, wenn es darum ging, arbeitsunfähige Gefangene in andere Lager zu  überstellen. Ihm war es dabei möglich, bestimmte Personen von den Listen zu streichen; nicht jedoch ohne einen Ersatz einzutragen.


\subparagraph{Arbeitsschreiber\index{p}{Strauß, Fritz}}
Fritz Strauß\index{p}{Strauß, Fritz}, am 22.11.1902 in Buchen\index{o}{Buchen / Odenwald} (Odenwald) geboren, wurde ebenfalls in Groß-Rosen\index{o}{Groß-Rosen} als Schreiber bestimmt und kam im August 1944 als einer der ersten Häftlinge ins Görlitzer KZ-Außenlager. Perfekte Deutschkenntnisse und Erfahrungen auf dem Gebiet der Buchhaltung waren für ihn hinreichende Voraussetzungen, um in eine solche Position zu gelangen.
In den Aufgabenbereich des Arbeitsschreibers\index{p}{Strauß, Fritz} fiel die Büroarbeit des Lagerführers\index{p}{Zunker, Winfried}, welche vorwiegend darin bestand, Totenscheine auszustellen. Die Todesursache setzten die Ärzte ein und sie unterschrieben die Totenscheine. Des Weiteren protokollierte Fritz Strauß\index{p}{Strauß, Fritz} die Häftlingszahlen und bereitete die Appelle vor.\newline
Der Arbeitsschreiber\index{p}{Strauß, Fritz} und der Verpflegungsschreiber\index{p}{Jeremias} teilten sich ein Zimmer -- ihnen blieben somit die unbeheizten und \glqq verlausten\grqq~Baracken erspart\footnote{Aussage von Fritz Strauß. LArchB B Rep 058 Bd. 1.}.

\subparagraph{Verpflegungsschreiber\index{p}{Jeremias}}
Ein gewisser Jeremias\index{p}{Jeremias} aus Klausenburg (Cluj-Napoca)\index{o}{Klausenburg} in Rumänien war Verpflegungsschreiber im Lager Görlitz\footnote{Aussage von Emmrich Schiffer. LArchB B Rep 058 Bd. 5.}.
Über ihn und seinen Aufgabenbereich ist bisher nichts Genaueres bekannt.

%%%%%%
\paragraph{Ärzte}
\label{arzt}
Insgesamt waren zeitweise bis zu acht Ärzte im Krankenrevier tätig. Der Lagerschreiber Emmrich Schiffer\index{p}{Schiffer, Emmrich} erinnerte sich an den Chefarzt Dr. Schein\index{p}{Schein, Dr.}, den ungarischen Arzt Dr. Seres\index{p}{Seres, Dr.} sowie Dr.Schwarz\index{p}{Schwarz, Dr.}, Dr. Kulka\index{p}{Kulka, Dr.} und Dr. Jawszon\index{p}{Jawszon, Dr.}\footnote{Emmrich Schiffer. LArchB B Rep 058 Bd. 5. Die Namen Dr. Schwarz und Dr. Schein wurden ebenfalls von Samuel Kessler in Bd. 1 und 2 genannt. Dr. Kulka ist durch Josef Gleitmann in Bd. 2 bestätigt.}.
Der aus \L \'od\'z stammende Dr. Jakob Kinrus\index{p}{Kinrus, Dr. Jakob} zeigte damals an, dass er Zahnarzt sei und wurde daraufhin auch im Krankenrevier eingesetzt. Ein weiterer Arzt aus \L \'od\'z war der Chirurg Dr. Mieczyslaw Jakobson\index{p}{Jakobson, Dr. Mieczyslaw}\footnote{Aussage von Jakob Kinrus: Yad Vashem.}.
Die Beschäftigung eines Arztes im Krankenrevir bestand nur so lange, wie es zu keinen Befehlsverweigerungen und übermäßigen Hilfeleistungen kam.
Dr. Schwarz\index{p}{Schwarz, Dr.} wurde vom Lagerältesten\index{p}{Czech, Hermann} zum Arbeitseinsatz geschickt, da er sich weigerte, die Kranken im Revier nach drei Tagen \glqq abzuspritzen\grqq\footnote{Samuel Kessler. LArchB B Rep 058 Bd. 1.}.
Jakob Kinrus\index{p}{Kinrus, Dr. Jakob} erging es ähnlich, als er die Kranken vor einem Abtransport warnte, indem er sie zur Arbeit schickte und somit ihr Leben rettete\footnote{Aussgabe von Jakob Kinrus. YV 3556634.}.
\newline
Inwiefern sich der allen Ärzten übergeordnete Sanitätsdienstgrad der SS\footnote{Dr. Mieczyslaw Jakobson. LArchB B Rep 058 Bd. 1.} in die Arbeit im Krankenrevier einmischte, ist ungewiss. Weitere Schilderungen über das Krankenrevier finden sich auf Seite~\pageref{krank}.

%%%%%%
\paragraph{Andere Funktionshäftlinge}
Im eigentlichen Sinne führte für die SS jeder Häftling eine Funktion aus. Neben den eben genannten, gab es jedoch noch weitere Gefangene, die mit einer ganz bestimmten Aufgabe betraut wurden und sich durch ihre Tätigkeit von der Mehrheit der Fabrikarbeiter unterschied. Diese Tätigkeiten verlangten zumeist weniger körperliche Anstrengungen und konnten teilweise auch selbständig ohne permanentes Beisein der SS oder der Kapos ausgeführt werden. Die täglichen Appelle blieben ihnen ebenfalls erspart. Die Besetzung der Stellen waren aber auch durch eine erhöhte Fluktuation geprägt.\newline
Als Beispiel seien hier nur Schusterei, Wäscherei, Tischler, Maler, Schildmaler\index{p}{Mącznik, Ben}\footnote{Ben Mącznik hatte laut eigener Aussage diese Position inne.}, Fuhrpark, Küchendienst und Hofdienst genannt (siehe auch S.~\pageref{lagerarbeit}).

%%%%%%%%%%%%%%%%%%%%%%%%%%%%%%%%%%%%%%%%%%%%%%%%
\subsection{Existenzbedingungen im Lager}
Unter Existenzbedingungen verstehen sich in erster Linie jene maßgeblichen Faktoren, die das Überleben sichern. Dazu zählen Ernährung, Bekleidung, Unterbringung und medizinische Versorgung.
Um es vorwegzunehmen, alle vier Bedingungen wurden absolut unzureichend erfüllt.
Die Verantwortung dafür trug die Lagerleitung, in deren Ermessen es lag, vorhandene Ressourcen gerecht zu verteilen und bei Bedarf anzufordern sowie sonstige Unzulänglichkeiten gegenüber der Kommandantur Groß-Rosen\index{o}{Groß-Rosen} oder der WUMAG anzuzeigen.
Ob es solche Bestrebungen seitens des Lagerführers\index{p}{Zunker, Winfried} gab, ist aufgrund der dürftigen Quellenlage nicht bekannt\footnote{Ein Fall ist bekannt, bei dem die Lagerleitung Nahrungsmittel bei der WUMAG anforderte, jedoch nicht für die gefangenen Menschen, sondern für die gefangenen Schweine, die sich die SS im Lager hielt (siehe S.~\pageref{schweine}).}, so dass an dieser Stelle einzig auf die Erinnerungen der ehemaligen Gefangenen und Zeitzeugen zurückgegriffen werden kann.
~
\newline~Der ehemalige Häftling Janusch Oborowicz\index{p}{Oborowicz, Janusch} beschreibt die Nahrungsmittelversorgung:
\begin{leftbar}
In den ersten vier Wochen unseres Aufenthaltes im Lager erhielten wir an Verpflegung pro Tag 400\,Gramm Brot (5 Mann ein Brot), 3/4\,Liter Gemüsesuppe, 1/2\,Liter Kaffee und 1/2\,Liter Abendsuppe. Je länger wir uns im Lager aufhielten, desto mehr wurden die Rationen gekürzt, zuletzt bekamen dann jeweils 10 Häftlinge ein Brot zugeteilt. [...] Infolge dieser geringen Rationen und aufgrund der Tatsache, dass auch die sehr schlechten Wassersuppen den Forderungen des Körpers nicht gerecht wurden, stieg die Sterbeziffer mit fortschreitender Zeit enorm.\footnote{BStU MfS ASt 13/48 Bd. 2 / 392.}\end{leftbar}
\newpage
Der Görlitzer Karl Kruner\index{p}{Kruner, Karl} führt fort:
\begin{leftbar}
Ich war im gleichen Werk (Abt. Maschinenbau) zu dieser Zeit als technischer Angestellter tätig.
Das den Häftlingen in der WUMAG verabreichte Mittagessen, das jeweils durch die Kapos gebracht wurde, bestand durchweg aus dünner Wassersuppe, so dass die Häftlinge im Laufe der Zeit körperlich nicht mehr in der Lage waren, die in der Fabrik anfallenden schwersten Arbeiten, zu denen sie herangezogen wurden, zu verrichten. Ich war wiederholt Augenzeuge, wie die Häftlinge über die entleerten Essenskübel herfielen, um sich die spärlichen Reste der in diesen Essenbehältern befindlichen Suppe zu erkämpfen. Es kam dabei wiederholt zu Schlägereien der Häftlinge, die sich auf diese Kübel stürzten.
Alle Häftlinge waren äußerst unterernährt. Im Werk war allerdings bekannt, dass unter ihnen laufend Ausfälle dadurch zu ver-zeichnen waren, dass sie des Hungertodes starben.\footnote{Karl Kruner, am 20.01.1912 in Weißwasser geboren. BStU MfS ASt 13/48 Bd. 2 / 386.}
\end{leftbar}
Ergänzend dazu muss gesagt werden, dass jene, die noch bei Kräften waren und mehr als das verlangte Arbeitssoll erfüllen konnten, von der WUMAG Prämien\label{pramien} erhielten.

Diese Prämien wurden jedoch nur in Form von Prämienscheinen ausgeben, welche die Häftlinge nur innerhlab des Lagers gebrauchen konnten. Jeden Samstag öffnete unter großem Gedränge für eine halbe Stunde die Lagerkantine, wo Zigaretten und Essen bezogen werden konnten\footnote{Ebenda. Der Andrang auf die Kantine war so groß, dass es mehrmals zu Ausschreitungen seitens der Kapos kam.}. Die angebotenen Waren galten als besonders begehrt, denn sie eröffneten Möglichkeiten, Tauschgeschäfte zu tätigen oder Funktionshäftlinge zu bestechen. Demzufolge war auch klar, was es bedeutete, keine Prämie zu erhalten.


Die Funktionshäftlinge selbst benötigten keine Prämienscheine, da sie sich vielfach die von den Gefangenen erworbenen Waren aneigneten und darüber hinaus Zugang zu den Lebensmittelvorräten des Lagers hatten. Um ihre eigene Position zu festigen, bestachen sie damit die SS-Leute. Besonders der Lagerälteste\index{p}{Czech, Hermann} Czech\index{p}{Czech, Hermann} versorgte die Wachleute und Lagerführung mit Kleidern, Lebensmitteln und Zigaretten, die eigentlich den Häftlingen zustanden\footnote{PMGR 70/4593/MF, 4702/14/DP.}. Es gibt auch Anzeichen, wonach Czech\index{p}{Czech, Hermann} und andere mehrfach Lebensmittel ins Frauenlager schafften. Das könnte einer der Gründe sein, warum bis Februar 1945 fast 140 Männer und keine einzige Frau ums Leben kam\footnote{Einäscherungsbücher der Friedhofsverwaltung Görlitz.}.
~\newline
Samuel Reifer\index{p}{Reifer, Samuel} liefert durch seine Aussage ein Gesamtbild der Zustände innerhalb des Lagers:
\begin{leftbar}
Die Zustände im Lager waren unbeschreiblich. Da war eine riesige Menschenmasse. Als ein Transport mit Juden aus \L \'od\'z\index{o}{Litzmannstadt} eintraf, bekamen wir keine neuen Baracken. Das Wasser lief durch die Löcher im Dach in die Baracken hinein. Im Winter waren die Baracken unbeheizt. Einige Häftlinge hatten keine Decken. Dort war eine ungeheure Anzahl an Läusen. Die meisten Häftlinge hatten keine Unterwäsche. Weil die ganze Fabrik von Läusen befallen war, wurde ein Entlausungssystem geschaffen, was aber wenig nützte, da unser Lager voller Läuse war. Wenn wir uns auf das Bretterbett legten, befiehlen uns sofort tausende von Läusen.
[...]
Im Lager war es furchbar schmutzig. Die Läuse fraßen uns. Es gab kein heißes Wasser. Nicht einmal während unseres Aufenthalts in Görlitz bekamen wir Gelegenheit, unsere Unterwäsche zu wechseln. Die Gefangenen verrotteten in ihren miesen Lumpen.
[...] \newline
Unser Alltag im Lager beginnt mit dem Weckruf um 4 Uhr früh. Nicht selten stehen wir in Leichtbekleidung 2 Stunden bis zum Appellende im klirrenden Frost. Zum Frühstück bekommen wir 25\,dkg [250\,g] Brot und Kaffee. Das Mittagessen, dass heißt die Wassersuppe, essen wir bei der Arbeit.
Zum Abendbrot gibt es wieder Kaffee.\footnote{Samuel Reifer, ZIH 301/2311.}
\end{leftbar}


%%%%%%%%%%%%%%%%%%%%%%%%%%%%%%%%%%%%%%%%%%%%%%%%
\subsection{Das Häftlingsrevier}
\label{krank}
Als Revier bezeichnet man eine Krankenstation innerhalb des Lagers. Sie war auf zwei Baracken aufgeteilt, in denen einmal 30 und einmal 15 Betten untergebracht waren\footnote{Jakob Kinrus. RAG Sammlungsgut Biesnitzer Grund.}. Außerdem fand sich dort ein kleines Zahnarztzimmer\footnote{Ebendieser.}. Über die Ausstattung des Krankenreviers ist nichts genaues bekannt. Doch werden dort, wie in anderen Konzentrationslagern auch\footnote{Vgl. J. Tuchel: Die Inspektion der Konzentrationslager 1938--1945, S. 152. Aus anderen Lagern ist bekannt, dass die Ärzte Papier als Verbandsmaterial einsetzten und auf vielerlei Weisen gezwungen waren zu improvisieren. Vgl. Isabell Sprenger: Groß-Rosen, S. 146.}, kaum die nötigsten Verbandsmaterialien und Medikamente vorhanden gewesen sein. Es ist insofern fraglich, inwieweit eine medizinische Grundversorgung gewährleistet werden konnte.
\newline
Das Revier war im Grunde nicht mehr als eine Unterkunft für nicht mehr arbeitsfähige Menschen. Obwohl sie nun vom Arbeitsdienst und den stundenlangen Appellen verschont blieben, konnten die wenigsten geheilt und auskuriert das Krankenrevier wieder verlassen. Die Häftlinge nannte das Krankenrevier deshalb auch \glqq Leichenkammer\grqq\footnote{Schlomo Graber: Schlajme, S. 69 und 72.}.
Die erhöhte Zahl der Todesopfer im Dezember 1944 gibt Grund zu der Annahme, dass die Betten im Krankenrevier nicht ausreichten und eine Abschiebung von Kranken deshalb in die Wege geleitet wurde. Nachweislich wurden Gefangene in das Außenlager Zittau verlegt\footnote{Aussage von Samuel Reifer. ZIH 301/2311.}.\newline
Die Häftlinge hatten häufig Phlegmone\footnote{Phlegmone sind Zellgewebsentzündungen, die an verschiedenen Körperteilen auftreten können. Schlomo Graber litt selbst unter einem Phlegmon bis lange Zeit nach der Befreiung.}, vor allem litten die meisten, besonders ab den Wintermonaten, unter allgemeiner Erschöpfung und Unterernährung, sowie aufgrund des unzureichenden und unausgewogenen Essens unter Durchfallerkrankungen. Im September/Oktober 1944, nur wenige Tage nach der Ankunft von Häftlingen aus dem Außenlager Fünfteichen\index{o}{Fünfteichen}, brach die Ruhr epidemieartig aus\footnote{Am fünften Tag nach der Ankunft gab man den Gefangen zum ersten mal 330\,g Brot, am darauffolgenden Tag jedoch nicht mehr, da die Ruhr bereits ausgebrochen war, berichtete Samuel Reifer, ZIH 301/2311. Ruhr ist eine Krankheit, die durch unzureichende hygienische Bedingungen oder unreines Trinkwasser und der dadurch hervorgerufenen Verbreitung einer bestimmen Amöbenform auftritt. Die Symptome sind in erster Linie Durchfall, aber auch Vereiterungen innerer Organe, die zum Tode führen können.}.
\newline
Abgesehen davon gab es sehr viele Verletzungen durch Folter und Arbeitsunfälle in den Fabriken der WUMAG. Georg Pilc\index{p}{Pilc, Georg} berichtet von einem tragischen Fall, wo sich Helmann Meiyer\index{p}{Meiyer, Helmann} im \glqq Panzerbau\grqq~der WUMAG akut am Bein verletzte, so dass es durch einen anwesenden Arzt amputiert werden musste\footnote{Nachtrag von Georg Pilc. LArchB B Rep 058 Bd. 3.}.
Für solche Fälle hatte ein Dienstarzt einsatzbereit zu sein.
Zu seinen Aufgaben gehörte es, mit den Arbeitern in die Fabrik zu gehen und notfalls Erste Hilfe zu leisten -- auch wenn einem Deutschen etwas an einer Maschine passierte -- erinnert sich Dr. Jakob Kinrus\index{p}{Kinrus, Dr. Jakob}\footnote{Jakob Kinrus war selbst eine Zeit lang Dienstarzt und wurde auch im Lager bei Notfällen gerufen. YV 3556634, vgl. auch Ratsarchiv Görlitz: Sammlungsgut KZ Biesnitzer Grund.}.\newline
Von den bis zu acht Häftlingsärzten galt Dr. Schein\index{p}{Schein, Dr.} als Chefarzt, der jedoch, wie alle anderen Ärzte auch, dem SS-Sanitätsdienstgrad unterstellt war. Die Arbeit teilte sich in mehre Bereiche, wie zum Beispiel Chirurgie, Notdienst oder Krankenhilfe. Neben den Ärzten arbeiteten aber auch andere Häftlinge im Revier\footnote{Zum Beispiel der erst 14-jährige Tuvia Altmann. LArchB B Rep 058 Bd. 5.}.
\newline
Den Ärzten und ihren Helfern im Krankenrevier gilt es nichts vorzuwerfen, im Gegenteil, denn nur sie hatten Zeit und Möglichkeit, sich auf die Hilfe anderen gegenüber zu konzentrieren. Ihr Widerstand gegen die Vernichtung von Leben schien angesichts der alltäglichen Folter und der schweren Arbeit in den Fabriken aussichtslos und stieß fortwährend an seine Grenzen. Eine gelungene Rettungsaktion von Dr. Kinrus\index{p}{Kinrus, Dr. Jakob} stellt eine Ausnahme dar und verdient deshalb an dieser Stelle eine besondere Erwähnung:

\label{widerstand_kinrus}
\begin{leftbar}
Jetzt war Monat März [1945]. Das Datum habe ich nicht vergessen -- es war der 15. März. Dies war auch mein größtes Erlebnis. \newline
Mein Sprechzimmer war neben dem [Raum des] Lagerältesten\index{p}{Czech, Hermann}. [...] An einem Nachmittag, ich befand mich in meinem Sprechzimmer, es war keiner da. Ich habe etwas sortiert und plötzlich ist ein Auto vorgefahren. Aus dem Auto sind vier Gestapoleute ausgestiegen und sind beim Lager\-ältesten\index{p}{Czech, Hermann} rein gegangen. Ich habe Bruchteile ihres Gesprächs gehört. Ich habe erfahren, dass am nächsten Tag früh um 7 Uhr die SS kommt mit dem Befehl [...] die Kranken mitzunehmen. Sie haben nicht gesagt, was sie mit ihnen machen, nur dass sie so einen Befehl haben. Der Lager\-älteste\index{p}{Czech, Hermann} hat das entgegengenommen. Ich habe abgewartet, bis sie abgefahren sind, und bin dann raus gegangen, [das] hat keiner gesehen. An dem Abend habe ich mit mir gekämpft, was soll ich machen? Informieren? Es sagen? Nichts sagen? [...] [Da] waren Vorfälle, wo Menschen für ein Brot verschiedene Geheimnisse verraten haben; dass jemand klaut u.s.w. Das ist beim Lager\-ältesten\index{p}{Czech, Hermann} angekommen und der hat sie dann bestraft -- [das ging] soweit, dass [...] er sie erschossen hat. Als ich abends am Ende meine Angst überwunden habe, bin ich ins Krankenhäuschen gegangen und dort habe ich gesagt: \glqq Jungs, hört mal zu! Ihr müsst alle morgen früh zum Appell aufstehen und arbeiten gehen -- fragt mich nicht warum -- so ist es.\grqq~Sie zweifelten: \glqq Ich bin schwer krank! Ich habe Durchfall, wie kannst du sagen, dass ich krank arbeiten soll?\grqq~[...] Ich war so empfindlich und aufgeregt, da ich wusste, was sie erwartet. Die Wahrheit konnte ich nicht sagen. Am Ende sagte ich: \glqq Es geht um euer Leben, ihr müsst morgen weggehen.\grqq~
[...] Zu diesem Zeitpunkt waren im Krankenhäuschen 17 Personen. 13 sind arbeiten gegangen und vier sind geblieben. Ich habe nicht gewusst, was im Lager geschieht, weil ich mit allen anderen arbeiten gegangen bin. Bei der Rückkehr hörte ich schon, dass früh die SS hier war und schrecklicher Krawall herrschte. Sie haben Leute aus dem Krankenhäuschen mitgenommen, es war keiner mehr da. Die Kranken hatten sich nun umgesehen. Einer von ihnen ist zu mir gekommen, hat mir die Hand gegeben und nichts gesagt. Der eine hat mir einen Kuss gegeben. Jedenfalls habe ich Dankbarkeit von ihrer Seite gespürt, dass ich einfach ihr Leben gerettet habe.\footnote{Beim abendlichen Appell versuchten der Lager\-führer und der Lager\-älteste mit Brot diejenigen hervor zu locken, die am Vortag noch im Krankenrevier waren. Es meldete sich keiner. Ein später verstorbener Häftling verriet dennoch Jakob Kinrus in seiner Not an den Lager\-ältesten, weshalb der Arzt Prügel erhielt und fortan zum Arbeiten in der Fabrik gezwungen wurde. Außerdem nahm man ihm seine Latschen weg und sagte, dass er auch barfuß im Schnee laufen kann. Aussage von Jakob Kinrus. YV 3556634. Ratsarchiv Görlitz: Akte zum KZ Biesnitzer Grund.}
\end{leftbar}

Dennoch taten die Lagerleitung und der Lager\-älteste\index{p}{Czech, Hermann} Erhebliches, um den Tod der Kranken zu beschleunigen. Czech\index{p}{Czech, Hermann} war es, der die Kranken aus ihren Betten jagte, ihnen befahl, sich in kaltes Wasser und Dreck zu legen, und sie teilweise zu Tode prügelte\footnote{Aus der Urteilsbegründung des zum Tode verurteilen Lagerältesten Hermann Czech. LArchB: B Rep 058 Bd. 2.}. Auch vom \glqq Abspritzen\grqq~ist in Überlebendenberichten die Rede. Dr. Schwarz\index{p}{Schwarz, Dr.} hatte sich dagegen verwehrt, woraufhin Czech\index{p}{Czech, Hermann}\linebreak\newpage die Injektionen selbst vornahm\footnote{Samuel Kessler. LArchB B Rep 058 Bd. 1. Hermann Czech, der von Beruf Apotheker war, kann man eine gewisse Kenntnis der Anwendung und Wirkungsweise bestimmter Injektionsmittel unterstellen.}. Ungewiss ist, was den Kranken injiziert wurde. Die Historikerin Isabell Sprenger schreibt über diese im KZ Groß-Rosen\index{o}{Groß-Rosen} gängige Praxis von Phenol und Blausäure als Injektionsmittel\footnote{Auch mit Benzin und Luftinjektionen wurde in Groß-Rosen zuvor experimentiert. Vgl. Isabell Sprenger: Groß-Rosen, S. 149f.}.
~\newline
\label{krankenhaus}
Während der letzten Tage und Wochen vor der Befreiung erhöhte sich der Krankenstand abermals. Dies lag vor allem an den schweren Schanzarbeiten, bei der die Menschen mit ihren Holzschuhen knöcheltief im Schlamm standen und der rauen Witterung ausgesetzt waren. Dr. Schwarz\index{p}{Schwarz, Dr.} hat in diesem Moment Mut bewiesen und bei der Lagerleitung ein Wort für die Kranken eingelegt, die daraufhin in ein öffentliches Krankenhaus aufgenommen werden konnten\footnote{Dies bezeugt insbesondere Abram Rajchbart, der zusammen mit anderen am 8. Mai im Krankenhaus von den Sowjets befreit wurde. ZIH 301/715. Es konnte nicht geklärt werden, in welches der beiden Görlitzer Krankenhäuser Klinikum oder St. Carolus) man die Häftling einlieferte. Archivanfragen in beiden Häusern waren erfolglos, da alle Krankenakten mit Verstreichen einer 30-Jahre-Frist vernichtet wurden.}.

%%%%%%%%%%%%%%%%%%%%%%%%%%%%%%%%%%%%%%%%%%%%%%%%
\subsection{Verbindungen zur Außenwelt}

Berichte über Konzentrationslager waren seit den ersten Jahren der Diktatur bis Kriegsbeginn in den Medien präsent und somit auch begrifflich im öffentlichen Bewusstsein verankert\footnote{Vgl. Robert Gellately: Hingeschaut und weggesehen -- Hitler und sein Volk, S. 283.}. Wenn es etwa in Zeitungen hieß \glqq ins Konzentrationslager\grqq, wußten die Menschen, um was es ging\footnote{Ebenda.}. Jedoch erlebten die meisten Leute diese grausame Seite des Nationalsozialismus erst im Verlauf des Krieges, als in Fabriken und auf den Straßen der Großstädte Lagerhäftlinge und Zwangsarbeiter den Alltag prägten und die Auswirkungen ihrer Haft nicht mehr zu übersehen waren. So auch in Görlitz und Umgebung, wo seit 1939 Zwangsarbeiter, insbesondere auch Kriegsgefangene aus allen besetzten Gebieten in den Fabriken, Handwerksbetrieben und in der Landwirtschaft der umliegenden Dörfer arbeiteten\footnote{Vgl. Ernst Kretzschmar: Görlitz unter dem Hakenkreuz, S. 55. Vgl. Rober Heinze: Ortschronik Rennersdorf. Vorzügliche Informationen über die Kriegsgefangenen in der Stadt Görlitz findet man in: Hannelore Lauerwald: In fremdem Land -- Kriegsgefangenlager Stalag VIII A. Vgl. auch Thomas Warkus: Kriegsgefangene und Fremdarbeiter im nationalsozialistischen Deutschland 1939--1945. Das Beispiel Görlitz. Neben den zahlreichen Zwangsarbeitern, die bei bei den Bauern untergebracht waren, gab es in der Oberlausitz auch ein Vielzahl von kleineren Lagern für Zwangsarbeiter in der Landwirtschaft. Beispielsweise in Kunnersdorf, Kemnitz und Rennersdorf.}. Grund für die verstärkte Präsenz der KZ-Häftlinge war\linebreak\newpage nicht zuletzt die von Oswald Pohl\index{p}{Pohl, Oswald}, dem Chef des SS-Wirtschaftsverwaltungshauptamtes, priorisierte Mobilisierung aller Häftlingsarbeitskräfte für Kriegsaufgaben\footnote{Pohl an Himmler vom 30. April 1942. Vgl. Robert Gellately: Hingeschaut und weggesehen -- Hitler und sein Volk, S.287.}.
Seit der Jahresmitte 1944 tauchten auch in der Oberlausitz die äußerlich erkennbaren KZ-Häftlinge in der Öffentlichkeit auf\footnote{Andernorts gewährte die SS-Wirtschaftsverwaltung bereits im März/April 1942 eigene KZ-Lager an Werksstandorten (Volkswagenwerk). Vgl. Hermann Kaienburg (Hrsg.): Wie konnte es soweit kommen?, S. 267, in Konzentrationslager und deutsche Wirtschaft. Es bildete sich um jedes Hauptlager ein Netz von Nebenlagern. Deren Zahl steigt gegen Kriegsende auf 634 an. Vgl. Robert Gellately: Hingeschaut und weggesehen -- Hitler und sein Volk, S. 285f.}.\newline

Besonders jene Bewohner in unmittelbarer Nähe des Lagers konnten sich deren Anblick nur schwer verwehren (siehe Bild ~\mypicsref{lagernah}). Trotz eines teilweisen Sichtschutzes um das Lager herum\footnote{Der Sichtschutz wurde laut Angaben von Hildegard Ecke im Sommer 1944 errichtet. Während der Inhaftierung von Juden durch die Organisation Schmelt und in der Zeit, als das Lager als Kriegsgefangenenlager diente, konnte man das Gelände uneingeschränkt einsehen.} war es beispielsweise vielen Bewohnern der Pestalozzistraße gut möglich, die stundenlangen Appelle in klirrender Kälte und all die Folterungen der Gefangenen von ihren Fenstern aus zu beobachten. Besonders die in direkter Nachbarschaft wohnende Familie des Bauern Ecke war gut im Bilde\footnote{Das Gut des Bauern Ecke stand auf einer Erhöhung in unmittelbarer Nähe des Lagers.}. Doch auch die Kleingärtner an der Südseite des Lagers sollten sich am beklagenswerten Zustand ihrer Nachbarn erschrocken haben, sofern man ihnen das Gärtnern währenddessen nicht untersagte\footnote{Einige Kleingärtner wurden Zeuge der Geschehnisse im Lager, andere gegen eine geringe Entschädigung vor dem Bau des Lagers vertrieben. Den Kontakt mit Kleingärtnern belegt ein Videointerview (Samson Munn, 1985) mit Ben Mącznik\index{p}{Mącznik, Ben}. Vgl. auch: Görlitzer Kulturspiegel September 1955: Mit Schweiß und Blut gedüngter Boden im Biesnitzer Grund. Vgl. auch: Sächsische Zeitung vom 8. Juli 1955.}.

Den Kontakt mit Kleingärtnern belegt ein Videointerview mit dem ehemaligen Häftling Ben Mącznik:
\begin{leftbar}
Das Zimmer in dem wir arbeiteten grenzte direkt an den Stacheldrahtzaun. Zwischen unserem Fenster und dem Zaun lief immer ein Wachmann mit einem Gewehr. Auf der anderen Seite des Zauns befanden sich Grundstücke von Deutschen, wo sie Gemüse anbauten.
Eines Tages sah ich aus dem Fenster, wie eine Frau versuchte mit uns zu sprechen. Sie arbeitete auf dem Beeten. Ich konnte sie nicht verstehen
und so gingen wir alle näher ans Fenster heran. Sie sagte: {}``Was habt ihr gemacht, dass man euch ins Gefängnis steckt?'' Wir erklärten ihr, dass wir nichts gemacht hätten und nur wegen unserer jüdischen Abstammung eingesperrt sind. {[}...{]} Am nächsten Tag kam\linebreak\newpage die Frau
wieder. Am übernächsten auch. Über drei Wochen hinweg kam sie immer wieder. Dann begann sie Essen zu uns hinein zu werfen. Einen Apfel, von dem sie schon ein Stück abgebissen hatte, warf sie hinein.\footnote{Shervi Basso: Video Interview of Ben Mącznik, 1996.}
\end{leftbar}

\myfigure[lagernah]{lager1}{BStU MFs Ast StKs 13/48 Bd.2.}{Aufnahme des Lagers mit dem Kühlturm des Maschinenbaukomplexes sowie den Häusern der Pästalozzi-Straße im Hintergrund}{Lager im Biesnitzer Grund}{0}

In besonderem Maße erfuhren die Anwohner entlang der Marschstrecke in Richtung Maschinen- und Waggonbau vom Leid der Häftlinge. Beide Arbeitskommandos liefen entlang der heutigen Walter-Rathenau-, Melanchthon- und Lutherstraße. Jene, die man ins Waggonbauwerk der WUMAG kommandierte, gingen weiter bis zum Brautwiesenplatz und die Cottbuser Straße bis hin zu den Werkstoren auf der Christoph-Lüders-Straße. Viermal täglich\footnote{Nachtschicht und Tagschicht.}, außer sonntags, hörten die Anwohner das lautstarke Klappern der Holzschuhe, bevor die erschöpften Kreaturen schließlich an ihren Fenstern vorüberzogen\footnote{Unter anderem durch Elfriede Terp, einer Bewohnerin der Melanchthonstraße, bestätigt.}.

\mymapfigure[map_goerlitz]{map_goerlitz}{}{Stadtplan von Görlitz mit Wegstrecke zur WUMAG}{Stadtplan Görlitz 1943/44}{0}{}

~\newline
Artur Berndt bekundet:
\begin{leftbar}
Kläglich anzusehen war nach geraumer Zeit das Arbeitskommando auf dem Marsch nach Arbeitschluss von der WUMAG in das Lager. Kraftlos schlichen sie die Lutherstraße und die Melanchthonstraße entlang. In ihrer Mitte führten sie Häftlinge, die sich nur mit Mühe auf den Beinen halten konnten. Die ganz entkräfteten saßen auf einem vierrädrigen Handwagen, auf dem noch die leeren Essenkübel standen und der von Häftlingen vor dem Kommando hergeschoben und gezogen wurde.\footnote{Erinnerungen von Artur Bernd. In: Ernst Kretzschmar. Görlitz unter dem Hakenkreuz, S. 56.}
\end{leftbar}
Beim größten Arbeitgeber der Stadt, der Waggon- und Maschinenbau AG, wo die Deutschen seit dem 15. Mai 1944 eine Minderheit darstellten, wussten die deutschen Arbeiter folglich auch über die Lage der Zwangsarbeiter und KZ-Häftlinge Bescheid\footnote{StArchD 11693 / 1258 Bild-Chronik III.}. Die Deutschen teilten sich zwar kaum einen Arbeitsplatz mit den Gefangenen, doch ist eine Warenfertigung ohne ein gewisses Maß an Kommunikation untereinander nicht vorstellbar. Ein Zusammenspiel kann sich nicht allein auf die Weisungen der Meister und Obermeister beschränkt haben, sondern erforderte auch Fähigkeiten, auftretende Probleme während der Produktion gemeinsam anzugehen. Das gilt in ähnlicher Weise auch für die kleineren Häftlingskommandos im Straßenbau, in der Kunnerwitzer\index{o}{Kunnerwitz} Landwirtschaft und in kleineren privaten Handwerksbetrieben, wie etwa der Firma Michler\index{p}{Michler, Gustav}\footnote{Vgl. S.~\pageref{michel}}.
~\newline~
Einzelnen privilegierten Gefangenen ergaben sich Möglichkeiten, bestimmte Lebensmittel in der Stadt zu besorgen, also im Grunde mit Einheimischen geschäftliche Beziehungen aufzunehmen\footnote{Paul Lewi brachte Gemüse aus der Stadt mit, um es Bekannten im Krankenrevier zu geben. Paul Lewi war von Beruf Elektromeister. Man kann annehmen, dass ihn diese Qualifikation zu etwas mehr Freiheit verhalf. LArchB B Rep 058 Bd. 1.}.
\newline
Es ist also nicht von der Hand zu weisen, dass sich die Beziehung vieler Görlitzer zu den Gefangenen des Konzentrationslagers nicht nur auf die bloße Wahrnehmung beschränkt haben kann. Ohne damit die Art der Beziehung zu bewerten, zeigt sich auf Seiten der deutschen Bevölkerung mehrheitlich ein unsicheres und distanziertes Verhalten genauso wie zwei entgegengesetzte Extreme in Sachen Menschlichkeit\footnote{Vgl. Robert Gellately: Hingeschaut und weggesehen -- Hitler und sein Volk, S.300ff.}.
Ein jedes dieser drei Verhaltensmuster kann einer bestimmten Gruppe von Personen zugeordnet werden, zu deren Verhalten sich jeweils Beispiele und Begründungen finden lassen.

%Für das Verhalten dieser drei Gruppen lassen sich auch im Fall Görlitz jeweils Beispiele und %Begründungen finden.
~\newline
Die erste Gruppe, die die Mehrheit garstellte, hat \glqq hingeschaut und weggesehen\grqq\footnote{Angelehnt an das gleichnamige Buch von Robert Gellately: Hingeschaut und weggesehen -- Hitler und sein Volk.}, denn dies war der sicherste Weg, um allen Ärger in der ohnehin schon schwierigen Zeit zu vermeiden. Gestapo, Lager-SS und örtliche Parteifanatiker, aber auch wachsame Nachbarn, bedrohten durch ihre verleumderische Strenge einen jeden, der auch nur minimales Mitleid gegenüber den Gefangenen zum Vorschein brachte. Die Warnung an die mitleidigen Deutschen bestand darin, selbst den Weg anzutreten, den die KZ-Häftlinge ihnen vor Augen führten. Mitte Juni 1944 ergriff die Gestapo eine Betriebsangehörige der WUMAG: \glqq Sie wird wegen wiederholten Verkehrs mit Tschechen in ein Konzentrationslager gebracht und kommt auf keinen Fall zu uns zurück.\grqq, heißt es daraufhin in einer Aktennotiz der Personalabteilung der WUMAG\footnote{Ein gewisses Fräulein Thiele (Direktion des Gehaltsbüros) vermerkte dies am 15. Juni 1944. StArchD 11693 WUMAG, 1258 (Bild-Chronik III).}. Im Sommer 1944 brachte man noch weitere Görlitzer wegen ihrer politischen Einstellung ins KZ Groß-Rosen\index{o}{Groß-Rosen}\footnote{Ernst Kretzschmar: Görlitz unter dem Hakenkreuz, S. 61.}. Bei solch drastischen Maßnahmen ist es kaum verwunderlich, dass einige sich nicht einmal mehr getrauten, nach den Häftlingskolonnen zu schauen, wenn sie am Haus vorbei liefen\footnote{Elfriede Terp wurde als kleines Mädchen von ihrem Großvater mit einer Ohrfeige gewarnt, als sie am Fenster schaute, woher der Lärm und das Geschrei kam. Es war ein Gefangenenzug, der die Melanchtonstraße entlang kam.}.

Henryk Vogler:

\begin{leftbar}
Aber die Mehrheit verhielt sich gegenüber den jüdischen Arbeitern ein bisschen so, als wenn sie nicht existieren. Sie wichen ihren Blicken aus, sie schauten nach oben und zur Seite, es ließ sie kalt. Außerdem unterließen sie Kontakte während der Arbeitszeit -- ihre Existenz und ihre Probleme interessierten sie nicht. Man hätte denken können, dass sie von der Existenz des Lagers, aus dem sie täglich extra Arbeitergruppen unter Aufsicht eskortierten, nichts wussten.\footnote{Henryk Vogle, PMGR 5758 / 734 / DP.}
\end{leftbar}

Andere, wie etwa Artur Bernd, haben \glqq hingeschaut und nicht weggesehen\grqq. Diese wenigen Leute gehörten zur zweiten Gruppe. Schlomo Graber\index{p}{Graber, Schlomo} schreibt eindrucksvoll von seinem Meister im Waggonbau:

\begin{leftbar}
Er schimpfte und brüllte uns zwar an, so dass wir alle vor ihm zitterten. Aber forderte er mich einmal -- natürlich wieder schreiend -- auf, ihm eine bestimmte Schraube aus dem Lager zu holen und erklärte mir genau, in welcher Schublade ich sie finden würde. Als ich die betreffende Schublade aufzog, lag dort ein belegtes Brot in Packpapier zwischen den Schrauben. Ich aß es schnell und kehrte an den Arbeitsplatz zurück. [...] Bei einer anderen Gelegenheit waren der Meister und ich allein im Lager. Er bat mich, ihm die Hände zu zeigen, zählte meine Finger und sagte: \glqq Du hast doch genauso fünf Finger wie ich. Warum bist du dann hier?\grqq~Das würde ich mich auch fragen, antwortete ich. Der Meister warnte mich und auch die anderen Häftlinge zur Vorsicht. Sobald wir an unseren Arbeitsplatz zurückkehrten, setzte er sein gespieltes Schimpfen und Brüllen fort. Obwohl er sich der Gefährlichkeit seines Tuns bewusst war, brachte er uns ab und zu Essen.\footnote{Zitiert aus Schlomo Graber: Schlajme, S. 74f.}
\end{leftbar}
Dass Meister der WUMAG auch ein absolut gegenteiliges Verhalten an den Tag legten, wird in Samuel Reifers\index{p}{Reifer, Samuel} Bericht über die Folterungen durch Müller\index{p}{Müller} deutlich (siehe S.~\pageref{muller})\footnote{ZIH 301/2311.}.
Müller\index{p}{Müller} gehört zu der nicht so kleinen Gruppe von Deutschen, von der man sagen kann, sie hat \glqq hingesehen und mitgemacht\grqq. An einer Reihe von  Berührungspunkten zwischen Lager und Außenwelt gab es Personen, deren Verhalten jenem der ersten Gruppe ähnelt, jedoch nicht als passiv bezeichnet werden kann. Zu ihnen gehörten sehr wohl die Funktionäre in den Chefetagen der WUMAG, aber auch bestimmte Personen in der Stadtverwaltung, der NSDAP und zahlreichen Ämtern. Jedoch auch unbedeutendere Leute schließt diese Gruppe mit ein. So zum Beispiel Frau Klara Brandenburg\index{p}{Brandenburg, Klara}, eine Verwandte des Lagerführers\index{p}{Zunker, Winfried} Zunker\index{p}{Zunker, Winfried}. Ihr machte es offenbar nichts aus, dass Häftlinge ihr regelmäßig mit einem Leiterwagen Lebensmittel vorbei brachten, die eigentlich für die hungernden Gefangenen bestimmt waren\footnote{Samuel Mandelbaum musste diese Aufgabe im Auftrag des Lagerführers\index{p}{Zunker, Winfried} erledigen. Die betreffende Frau wohnte damals in der Hospitalstraße 19. LArchB B Rep 058 Bd. 2.}. Oder jene Angestellten des Transportunternehmens Schubert\index{p}{Schubert, Ernst}, die insgesamt 140 Leichen innerhalb von sieben Monaten stets von Häftlingen verladen ließen und ins Krematorium brachten\footnote{Schubert selbst gehörte zu den wenigen Zivilpersonen, die Zutritt und Einblick ins Lager hatten und die sich immer mehr abzeichnenden Spuren des Terrors beobachten konnten. In dem Zusammenhang müßte man auch die Mitarbeiter des Görlitzer Krematoriums erwähnen. Drei Häftlinge aus den KZ-Außenlagern Niesky starben am Tag der Einäscherung nämlich in Görlitz und nicht in Niesky. Einäscherungsbücher der Görlitzer Friedhofsverwaltung. Fragt sich, ob sie während der Fahrt oder in erst im Feuer des Krematoriums starben.}.
~\newline
Zu beantworten bleibt noch die Frage, inwiefern die Gefangenen des KZ-Außenlagers bei ihrem Erscheinen überhaupt als Häftlinge eines Konzentrationslagers beziehungsweise als Juden erkannt wurden. Vielfach wird in Vernehmungsprotokollen die Haftstätte von Görlitzer Bürgern, ehemaligen Wachleuten und ehemaligen Gefangenen, dem SS-Jargon entsprechend, als \glqq Arbeitslager im Biesnitzer Grund\grqq~bezeichnet. Besonders jene unmittelbar involvierten Personen, wie etwa der WUMAG Fuhrparkleiter, der NSDAP Kreisleiter und die Wachleute mieden die Bezeichnung Konzentrationslager\footnote{BStU MfS ASt 13/48 Bd. 2.}. Die gestreifte Kleidung und die kahl geschorenen Köpfe sollten die Häftlinge als Verbrecher dastehen lassen\footnote{Hermann Czech befahl den Häftlingen, ihre Mützen abzunehmen, damit sie durch ihre kurzgeschorenen Köpfe wie Verbrecher aussahen. Samuel Reifer: ZIH 301/2311.} und nichts, abgesehen von einem kleinen gelben Dreieck auf der Kleidung, deutete an, dass sie aus rassistisch-antisemitischen Gründen gefangen gehalten wurden\footnote{Samuel Reifer erwähnt gar, dass man ihnen anfangs in der WUMAG gruppenweise ein Bad genehmigte und sie gut behandelte, bis zu dem Tag, als sich herausstellte, dass sie Juden waren. Die Meister in der WUMAG kannten sehr wohl die Symbolik des gelben Dreiecks, doch hatte Reifers, aus Fünfteichen kommende Transport von Häftlingen noch am Abfahrtsort seine Sträflingskleidung gegen eingefärbte Zivilkleidung eintauschen müssen. Samuel Reifer, ZIH 301/2311.}. Dennoch konnte diese von der SS erdachte Tarnung nicht verleugnen, dass es sich bei den Gefangenen um Menschen handelte, denen die chronische Unterernährung und Kraftlosigkeit ins Gesicht geschrieben stand. Der Beweis, dass sich die Görlitzer Bevölkerung bis Kriegsende in einer vergleichbaren oder zumindest ähnlichen Hungersituation befand, konnte nicht erbracht werden\footnote{Vgl. Franz Scholz: \glqq Görlitzer Tagebuch\grqq; Sammlungsgut Biesnitzer Grund, RAG; Prozessakten Malitz-Meinshausen, BStU Mfs ASt 13/48.}, wenn man einmal von den Vertriebenen 1945 absieht. In einer Zeugenaussage von Ursula Taube heißt es:
\begin{leftbar}
Während meiner Tätigkeit in der Melanchthonschule hatte ich Gelegenheit, wiederholt festzustellen, dass die Häftlinge gierig in dem Abfallhaufen auf der Klenke-Wiese nach Abfällen suchten, um sie zu essen. Besonders die Männer machten einen sehr schwachen Eindruck und es bestand kein Zweifel darüber, dass sie sehr hungrig und unterernährt waren. Die Wahrnehmung habe nicht nur ich gemacht, sondern alle Einwohner der Südstadt, in der ich wohne, war die Unterernährung bekannt.\footnote{Ursula Taube, eine Görlitzerin, die den Evakuierungsmarsch der KZ-Häftlinge begleitete. BStU MfS ASt 13/48 Bd. 2 / 404.}\end{leftbar}
Der Bestattungsunternehmer Ernst Schubert\index{p}{Schubert, Ernst} sagte aus:
\begin{leftbar}
Sobald wir mit unserem Leichenwagen in die Melanchthonstraße fuhren, sagte die Bevölkerung, dass wieder der jüdische Leichentransport käme. So war die ungeheure Sterblichkeit bereits Stadtgespräch.\footnote{Ernst Schubert, BStU MfS ASt 13/48 Bd. 2 / 546.}
\end{leftbar}


Was die Görlitzer Bürger über die Häftlinge des KZ-Außenlagers wirklich dachten, lässt sich schwer verallgemeinern.
Angesichts der Größenordnung und der sich verstärkenden Gerüchte über die mörderischer Brutalität der Nationalsozialisten kann man eigentlich nicht glauben, dass die Görlitzer überhaupt keine Ahnung davon hatten, was im Lager vorging, und alles glaubten, was ihnen die Propaganda über die \glqq gefährlichen Kriminellen\grqq~im \glqq Arbeitslager\grqq~erzählte und zu suggerieren versuchte\footnote{Vgl. Robert Gellately: Hingeschaut und weggesehen -- Hitler und sein Volk, S. 305.}. Berichte aus der Görlitzer Bevölkerung belegen, dass die Situation der Gefangen durch die Augenzeugen als Notfall erkannt wurde, ein Eingreifen jedoch mit unabsehbaren Gefahren und Kosten verbunden gewesen ist\footnote{Der namhafte Psychologe Philip Zymbardo schreibt dazu, dass die Bereitschaft Hilfe zu leisten davon abhängt, ob (1) der Zuschauer den Notfall bemerkt, (2) das Ereignis als Notfall eingestuft wird, (3) eine Verantwortlichkeit erkannt wird und (4) die Kosten des Helfens nicht zu hoch sind.}.
In Anbetracht der oftmals gemiedenen Gefahren, die ein Einschreiten bewirkt hätte, stellt sich umgekehrt die Frage, wie hoch die Schuld des Einzelnen an dem Vergehen ist. Hannah Arendt\footnote{Die deutsch-amerikanische Politologin Hannah Arendt (1906--1975) wurde unter anderem durch ihre Totalitarismusforschungen (\glqq The orgins of Totalitarianism\grqq) bekannt.} sagte dazu einmal:
 \glqq Die persönliche Schuld, die uns als Mitglied einer Gesellschaft an deren Vergehen trifft, ist im Allgemeinen viel kleiner, als die Schuld, die uns Außenstehende hinterher zumessen, aber unsere persönliche Schuld ist andererseits viel größer als die Schuld, die wir uns selbst eingestehen.\grqq\footnote{Hannah Arendt zitiert in Hans-Peter Dürr: \glqq Für eine zivile Gesellschaft\grqq.}

Die hier dargestellten Berührungspunkte mit der \glqq zivilen\grqq~Außenwelt, verdeutlichen exemplarisch den Einfluss, der in der Nähe von Wohngegenden errichteten KZ-Außenlager auf den Alltag der deutschen Bevölkerung.
Wenngleich viele Leute eine Teilschuld tragen oder zumindest einen kleinen Beitrag zur Aufrechterhaltung des Systems leisteten, gilt es mit aller Deutlichkeit festzuhalten, dass die Hauptverantwortung für die Errichtung von Konzentrationslagern, und damit auch für die Behandlung der KZ-Häftlinge, Hitlers\index{p}{Hitler, Adolf} Ernennung zum Reichskanzler in Jahren 1933 bis 1945 trägt\footnote{Hermann Kaienburg: Wie konnte das geschehen?, S. 267.}. Alle grundlegenden Befehle gingen von Hitler\index{p}{Hitler, Adolf} aus und Himmler\index{p}{Himmler, Heinrich} sorgte für die Organisation des Terrors in den Konzentrationslagern\footnote{Ebenda.}. Auf das KZ-Außenlager Görlitz bezogen, verteilt sich die Verantwortlichkeit auf verschiedene Personengruppen (siehe S.~\pageref{ns-verbrechen}), wie etwa der Lagerleitung, Wachmannschaften, WUMAG-Mitarbeiter und auch Funktionshäftlinge.

%%%%%%%%%%%%%%%%%%%%%%%%%%%%%%%%%%%%%%%%%%%%%%%%
\section{Lagerleitung und Wachmannschaften}
Die Lagerleitung bildete in erster Linie der Lagerkommandant\index{p}{Rechenberg, Erich}, welcher gleichzeitig mehrere Außenlager befehligte. Sein ständiger Vertreter vor Ort war der Lagerführer\index{p}{Zunker, Winfried}, welcher eng mit dem so genannten Rapportführer zusammenarbeitete und dem Hauptlager stets Meldung zu erstatten hatte.
~\newline
Während man in den Vorkriegsjahren die seit 1937 sogenannten Totenkopfstandarten in den jeweiligen KZ-Wachtruppen einsetzte, bestanden die Wachmannschaften später aus Mitgliedern der Polizeiverstärkung, meist ältere Männer der Allgemeinen SS, oder aus Wehrmachtsangehörigen, die man für \glqq frontuntauglich\grqq~erklärt hatte\footnote{Vgl. Isabell Sprenger: Groß-Rosen, S. 21. und vgl. Karin Orth: \glqq Experten des Terrors. Die Konzentrationslager-SS und die Shoah\grqq, S. 99 in Paul (Hgs.).}. Vielfach mußten sich Reservisten in den letzten Kriegsjahren in den Landesschützenbataillonen oder beim Volkssturm melden. Ihr Einsatz in Konzentrationslagern bewahrte sie vor der Front, so dass sich viele jener Weltkriegsveteranen gegenüber dem Leid der Menschen verschlossen haben. Im Nachhinein schämten sich viele ihrer Feigheit und bedauerten ihre Tatenlosigkeit\footnote{Vgl. S.~\pageref{kotter}.}. Eine Entschuldigung kann dies jedoch nicht sein.\newline

\paragraph{Lagerkommandant}
Erich Willi Fritz Rechenberg\index{p}{Rechenberg, Erich} (siehe Bild ~\mypicsref{rechenbergfoto}) wurde am 01.03.1901 als Sohn einer gutbürgerlichen Familie in Berlin\index{o}{Berlin} geboren. Er machte sein Abitur und besuchte eine Reihe weiterführender Schulen, die ihn später befähigten, als Registrator im Berliner\index{o}{Berlin} Landgericht zu arbeiten\footnote{Rechenberg besuchte unter anderem auch eine Landwirtschaftsschule. Laut Aussage seines Sohnes Hans-Joachim Rechenberg.}. Später heiratete er und hatte drei Kinder\footnote{Ebendieser.}.

\myfigure[rechenbergfoto]{wp1}{Foto aus den 1960er Jahren. LArchB B Rep 058.}{Lagerkommandant\index{p}{Rechenberg, Erich} Erich Rechenberg\index{p}{Rechenberg, Erich}}{}{.4}


Bereits am 1.1.1928 trat er in die NSDAP ein, unterbrach seine Mitgliedschaft jedoch am 9.5.1928, bis  er zwei Jahre später am 1.10.1930 erneut in die Partei eintrat. Zwischen 1929 und 1934 war er Mitglied der SA, und nach deren Zerschlagung 1934 wechselte er zur SS\footnote{Aussage von Frau Rechenberg. LArchB B Rep 058 Bd. 2.}.

Während des zweiten Weltkrieges diente er bis zum 14.10.1941 im 1. Infanterieregiment 301 und kämpfte in Polen, Frankreich und der Sowjetunion\footnote{Angaben aus Rechenbergs Antrag auf Entschädigung für die Kriegsgefangenschaft. LArchB B Rep 058 Bd. 2.}. Neben den damals üblichen Auszeichnungen beförderte man ihn im Mai 1941 zum Leutnant
\footnote{Er erhielt das Eiserne Kreuz zweiter Klassen sowie das Sturmabzeichen. Belege über Rechenbergs Kriegsgefangenschaft, erfragt in einem Rechtshilfeersuch im Zusammenhang mit den Vorermittlungen gegen Erich Rechenberg im Jahre 1971. LArchB B Rep 058 Bd. 3.}.
Laut Aussage seines Sohnes Hans-Joachim wurde er dreimal verwundet und erlitt mehre Trümmerbrüche. Eine besonders schwere Verletzung erlitt er in Weißrussland durch einen Durchschuss, der die Wirbelsäule nur knapp verfehlte\footnote{Telefonisches Interview mit Hans-Joachim Rechenberg.}. Bei der Bewilligung von späteren Entschädigungen behauptete Rechenberg später, seit Juni 1942 bis Kriegsende im Landesschützenbataillon 238, welches unter der Kommandantur des rückwärtigen Armeegebietes (Korück) 580 stand, gedient zu haben.
Es gibt Hinweise für eine Beförderung zum Oberleutnant der Wehrmacht, jedoch auch für eine Versetzung zur SS nach Auschwitz\index{o}{Auschwitz} im Juni 1944\footnote{Aussage von Frau Rechenberg. LArchB B Rep 058 Bd. 2.}. Dies war, angesichts des Personalmangels in der SS, durchaus nichts Ungewöhnliches\footnote{Vgl. Hermann Kaienburg: Die Wirtschaft der SS, S. 391f.}.
Entsprechend seinem Rang in der Wehrmacht blieb er im gehobenen Dienst, nun jedoch unter der Bezeichnung Obersturmführer\footnote{Vgl. Ämter, Abkürzungen, Aktionen des NS-Staates, S. 46ff.}.
Dies befähigte ihn anscheinend dem Kommandanturstab des KZ Groß-Rosen anzugehören
\footnote{Vgl. Isabell Sprenger: Groß-Rosen, S. 97f.}. Rechenberg\index{p}{Rechenberg, Erich} war als Lagerkommandant\index{p}{Rechenberg, Erich} laut Angaben in den Ermittlungakten der Berliner Staatsanwaltschaft auch für die Wachmannschaften der Groß-Rosener Außenlager Bautzen, Görlitz, Kamenz\index{o}{Kamenz} (Herrental), Kratzau (Chrastava, Tschechien)\index{o}{Kratzau}, Niesky\index{o}{Niesky} (Wiesengrund) und Zittau\index{o}{Zittau} (heute: Sieniawka, Polen) verantwortlich\footnote{LArchB B Rep 058 Bd. 10. Detailierte Quellen, die das bestätigen könnten, gibt es nicht.}.
Er pendelte zwischen den Nebenlagern und dem Hauptlager, weshalb ihn viele Häftlingen nicht namentlich kannten\footnote{Samuel Mandelbaum sagte, der Lagerkommandant sei nur wenig in Erscheinung getreten. LArchB B Rep 058 Bd. 2.}. Rechenberg legte Wert da\-rauf mit \glqq Herr Oberleutnant\grqq~angesprochen zu werden und verzichtete da\-rauf, die SS-Uniform zu tragen. Mit seiner Frau und seinen drei Kindern lebte er in Sichtweite des KZ Görlitz in einer Baracke\footnote{Laut Aussage von Hildegard Ecke, der Tochter des Bauern Ecke, welcher das Bauerngut im Biesnitzer Grund von der Stadt Görlitz gepachtet hatte.} auf dem \glqq Lehmberg\grqq\footnote{\glqq Lehmberg\grqq~nannte man umgangssprachlich den kleinen Hügel im Biesnitzer Grund, welcher um die Jahrhundertwende aus Abraum vom Bau der Gleisanbindung der Görlitzer Machinenbauwerke dort angehäuft wurde.}. Sein Heim ließ er durch Häftlinge aufräumen und renovieren\footnote{Aussage von Samuel Sally Kessler in LArchB B Rep 058 Bd. 3}. Seine täglichen Mahlzeiten nahm Rechenberg\index{p}{Rechenberg, Erich} im KZ-Außenlager Görlitz ein
\footnote{Aussage von Isaac Weintraub. LArchB B Rep 058 Bd. 3.}. Weiterhin profitierte Rechenberg\index{p}{Rechenberg, Erich} wie auch seine Wachleute von den Lebensmitteln, die aus der Häftlingsküche entwendet wurden, um sich und seine Familie besser zu versorgen. Darüber hinaus ließ er mehrere Kaninchen innerhalb des Lagers von Häftlingen füttern \footnote{Aussage von Samuel Kessler und Isaac Weintraub. LArchB B Rep 058 Bd. 3. Auch der Lagerschreiber bestätigt dies in Band 5. Schlomo Graber bestätigte dies während eines Gesprächs.}. Näheres zu den Ermittlungen gegen Erich Rechenberg ist auf Seite ~\pageref{rechenberg_ahndung} zu lesen.


\paragraph{Lagerführer}
Wilfried Zunker\index{p}{Zunker, Winfried} (Bild ~\mypicsref{zunkerfoto}) wurde am 18.09.1917 in Fürstenwalde\index{o}{Fürstenwalde / Spree} / Spree geboren. Zwischen 1932 und 1935 war er in der Hitlerjugend aktiv und verdiente sich das HJ-Ehren\-ab\-zei\-chen, das SA Wehrabzeichen und den Grundschein der Deutschen Lebensrettungsgesellschaft. Später erlernte er den Beruf eines Gärtners und schloss sich am 1.1.1936 der \glqq nicht lebensrettenden SS\grqq\footnote{\glqq Nicht lebensrettenden\grqq meint, dass Zunker keinen Sanitätsdienst verrichtete. SS-Nr: 281 940, Einheit: Fp.Nr. 41 516.} und NSDAP\footnote{NSDAP Nr: 4636734.} an.
\newline
Am 25.09.1939 wurde er während des Polenfeldzuges durch einen Streckschuss am Fuß verwundet.
Im Jahre 1940 diente Zunker\index{p}{Zunker, Winfried} in der ersten Einheit der \glqq Leibstandarte SS Adolf Hitler\grqq, welche gemeinhin europaweit für eine ganze Reihe von Massakern verantwortlich ist. Zunker\index{p}{Zunker, Winfried} läßt sich 1943 an die SS-Junkerschule in Braunschweig\index{o}{Braunschweig} versetzen und wird noch im selben Jahr in den Rang eines Oberscharführers erhoben. In der Folgezeit arbeitete er als Büroangestellter bei der Inspektion der Sicherheitspolizei und dem Sicherheitsdienst in Breslau\index{o}{Breslau}\footnote{LArchB B Rep 058 Bd. 1.}.
\newline
Wenngleich über Zunkers\index{p}{Zunker, Winfried} Tätigkeit in Breslau\index{o}{Breslau} und bei der Leibstandarte nichts Genaueres bekannt ist, steht fest, dass er seinem Lebenslauf entsprechend ideologisch außerordentlich gefestigt gewesen sein muss. Ab dem 8. August 1944 wird er als Lagerführer\index{p}{Zunker, Winfried} im KZ-Außenlager Görlitz eingesetzt. Näheres zun Ermittlungen gegen Winfried Zunker ist auf Seite~\pageref{zunker_ahndung} zu lesen.
~\newline

\label{zunkerfoto}
\begin{minipage}[b]{.47\linewidth}
	\myfigure{wp2}{}{}{}{0}
\end{minipage}
\hspace{20pt}
\begin{minipage}[b]{.47\linewidth}
	\myfigure{wp3}{}{}{}{0}
\end{minipage}
\vspace{-25pt}
\mypics[LArchB B Rep 058.]{Lagerführer\index{p}{Zunker, Winfried} Winfried Zunker\index{p}{Zunker, Winfried}}{Lagerführer Winfried Zunker}






\paragraph{Rapportführer}
Der für die Lagerschreibstube verantwortliche Rapportführer hatte tägliche Veränderungen in der Lagerstärke zu melden. Zeugenaussagen zufolge nahm diesen Posten ein gewisser Herr Lange oder Langer ein. Bis auf seine Herkunft aus dem Raum Chemnitz und sein ungefähres Alter zwischen 50 und 60 Jahren ist nichts Näheres über diese Person bekannt. Mehrfach fällt sein Name auch im Zusammenhang mit Erschießungen während des Todesmarsches.


\paragraph{Wachmannschaften der SS} Die Aufgabe der SS-Wachmannschaft bestand in der Absicherung des Lagerkomplexes sowie der sicheren Verwahrung der Insassen im Lager sowie auf dem Weg zu ihren Arbeitsorten.
Der Berliner Staatsanwaltschaft zufolge gehörten die Görlitzer Wachleute dem SS-Totenkopfsturmbann Abteilung 9 des KZ Groß-Rosen an. Anfang November waren im Stammlager insgesamt 2430 Wachmänner und 813 Aufseherinen eingesetzt; Mitte Januar 1945 erhöhte sich die Zahl auf 3.222 bzw. 906\footnote{Vgl. Alfred Konieczny: Das KZ Groß-Rosen in Niederschlesien, S. 316, in: Ulrich Herbert, Karin Orth, Christoph Dieckmann (Hrsg.): Die nationalsozialistischen Konzentrationslager in Entwicklung und Struktur, Band. 1, Wallstein Verlag, Göttingen 1998.}. Diese Aufstockung wurde durch Soldaten des Volkssturms und der Landesschützenbataillone zur Bewachung der KZ-Gefangenen erzielt, da die SS nicht über ausreichend Kapazitäten verfügt und keine Soldaten von der Front abgezogen werden konnten\footnote{Vgl. Hermann Kaienburg: Die Wirtschaft der SS, S. 391f.}.
Es handelte sich größtenteils um ältere Männer. Die meisten der Volkssturmleute kamen noch vor dem Evakuierungsmarsch im Februar 1945 ins Lager\footnote{Samuel Kessler meint, dass die Männer vom Volkssturm während des Evakuierungsmarsches keinen einzigen Häftling erschossen hätten. Ähnliches sagte auch Maximilian Brandt aus. LArchB B Rep 058 Bd. 3.}. Bei einem Verhör durch die Staatsanwaltschaft sagte der ehemalige Wachmann Arno Günzel:

\begin{leftbar}
In Schulungen ist uns gesagt worden, wie wir uns gegenüber den Häftlingen zu verhalten haben. Die Ausbildung erfolgte außerhalb des Häftlingslagers. Wie hatten Gelegenheit die Häftlinge von weitem zu sehen. Wir hatten demzufolge keinen unmittelbaren Kontakt zu den Häftlingen des KZ Groß-Rosen\index{o}{Groß-Rosen}. [...] Ich muß aber einräumen, dass unser Transport der erste war, der nach Görlitz ging. [...] Bei der Ankunft im Lager erfolgte die Diensteinteilung.\footnote{Arno Günzel, BStU MfS ASt 13/48 Bd. 2.}
\end{leftbar}

In Überlieferungen von ehemaligen Gefangenen des Lagers finden sich keine negativen Äußerungen über diese Männer. Ignatz Weinryb\index{p}{Weinryb, Ignatz}~sagt es mit einem Satz: \glqq Diese Leute waren sehr nett.\grqq~\footnote{Ignatz Weinryb änderte später seinen Vornamen in Irving, als er nach Amerika auswanderte. LArchB B Rep 058 Bd. 6.}. Dennoch kann dies nur als teilweise Entlastung gesehen
In wie weit sie jedoch eine Mitschuld an den unmenschlichen Verhältnissen im Lager trugen, in dem sie etwa den Terror einiger Funktionshäftlinge duldeten, bleibt offen.
\newpage
Die späte Einsicht des ehemaligen Wachmanns Erich Kotter\index{p}{Kotter, Erich}:
\label{kotter}
\begin{leftbar}
Den Häftlingen wurde ein derart schlechtes Essen verabreicht, was die Schweine nicht fressen würden. Ich habe derartige Missstände, wie im Lager Biesnitzer Grund festzustellen waren, vorher niemals erlebt und würde es für unmöglich gehalten haben, wenn ich mich nicht mit eigenen Augen davon überzeugt hätte. [...] Diejenigen Personen, denen die Betreuung des Lagers oblag, haben in jeder Hinsicht versagt und sich eines Verbrechens schuldig gemacht, dass nicht wieder gut zu machen ist. Man muss sich schämen Wachmann eines derartigen Lagers gewesen zu sein.\footnote{Erich Kotter, BStU MfS ASt 13/48 Bd. 2 / 409.}
\end{leftbar}

Folgende Personen konnten bislang als Teil der Wachmannschaft ermittelt werden:
\begin{itemize}
\item Max Derbeck\index{p}{Derbeck, Max}; SS-Mann

\item Arno Günzel\index{p}{Günzel, Arno}; SS-Stabsscharführer (* 1886 in Gränitz bei Dresden)%14.03.1886

\item Homilius\index{p}{Homilius}; SS-Oberscharführer

\item Erich Kotter\index{p}{Kotter, Erich} (* 1903 in Girbigsdorf)
%landschaftsgärtner, landkauf:obstbau %14.03.

\item Karl Kruner\index{p}{Kruner, Karl} (* 1912 in Weißwasser)%20.01.

\item Max Lange\index{p}{Lange, Max}; SS-Rottenführer (* 1885)%09.05.

\item Willi Neumann\index{p}{Neumann, Willi}; SS-Sturmscharführer

\item Kurt Richter\index{p}{Richter, Kurt}; SS-Unterscharführer

\item Hand Rockstroh\index{p}{Rockstroh, Hans}\footnote{Hans Rockstroh war ein Bekannter der in Nachbarschaft zum Lager wohnenden Familie Ecke. Laut Hildegard Ecke war er damals 55 Jahre alt.}

\item Schuster\index{p}{Schuster}; SS-Unterscharführer (aus dem Sudentenland)

\item Erich Wauter\index{p}{Wauter, Erich}; Oberscharführer (* 1890)\footnote{Erich Wauter war seit 1937 Mitglied in der NSDAP und ist 1935 der Allgemeinen SS beigetreten (Unterscharführer). Nach einer Beförderung zum Oberscharführer war er in der Waffen-SS und wurde in verschiedenen Konzentrationslagern eingesetzt: September 1939 bis April 1940 KZ Saschsenhausen, anschließend bis Januar 1945 im KZ Ravensbrück und schließlich bis Februar 1945 im KZ Groß-Rosen bevor er in Görlitz zum Einsatz kam. Nach Kriegsende verblieb er bis 1946 im amerikanischen Internierungslager Dachau. LArchB B Rep 058 Bd. 5.}

\end{itemize}

Zur Evakuierung im Februar 1945 wurden zusätzlich ukrainische und/oder lettische SS-Angehörige niederen Dienstgrades herangezogen\footnote{Aussage von Emmrich Schiffer. LArchB B Rep 058 Bd. 5.}.

\subparagraph{Küchenchef und Magazinverwalter}
Der von Hause aus gewesene Gastwirt Arno Günzel\index{p}{Günzel, Arno} arbeitete in der Küche für die Wachmanschaften als Küchenchef und Magazinverwalter\footnote{Aussage des Lagerschreibers Emmrich Schiffer. PMGR und LArchB B Rep 058 Bd. 5.}. Er bekleidete den Rang eines Stabsscharführers.
Als Stabsfeldwebel wurde Arno Günzel\index{p}{Günzel, Arno} zunächst 1944 zum Militärdienst zu einem in der Nähe von Dresden stationierten Landesschützenbataillon eingezogen\footnote{PMGR 154/DS 5/68-16/MF.}. Im Sommer desselben Jahres schickte man ihn und 35 bis 40 andere Soldaten zur Ausbildung nach Groß-Rosen, wo er zum Stabsscharführer befördert wurde, und anschließend zusammen mit einem Häftlingstransport nach Görlitz.


\subparagraph{Sanitätsdienstgrad}
Zwischen Herbst 1944 und März 1945 wurde ein unbekannter, etwa 40 Jahre alter SS-Mann als SS-Sanitätsdienstgrad eingesetzt. Später dann eine etwas jüngere, gleichfalls unbekannte Person\footnote{Aussage von Dr. Jakobson. LArchB B Rep 058 Bd. 1.}.

\paragraph{Wachpersonal im Frauenlager}
Über das weibliche Personal der SS im Außenlager Görlitz gibt es fast gar keine Aufzeichnungen. Es ist nicht einmal bekannt, welchem Wachbataillon jene Frauen angehörten. Belegt ist bislang nur die Tätigkeit von Liesbeth Frieda Barth\index{p}{Barth, Liesbeth Frieda}, am 11.07.1920 in Kunnersdorf / Krs. Oels\index{o}{Kunnersdorf / Krs. Oels} geboren. Sie gehörte zwischen dem 1. Juli 1944 und Mai 1945 den SS-Wachmannschaften in den Groß-Rosener\index{o}{Groß-Rosen} Außenlagern Breslau-Hundsfeld (Psie Pole k Wroc\l awia)\index{o}{Breslau-Hundsfeld}\footnote{Das Lager existierte zwischen dem 01.07.1944 und dem 25.01.1945.} und Görlitz an\footnote{LArchB B Rep 058 Bd. 4.}.

%%%%%%%%%%%%%%%%%%%%%%%%%%%%%%%%%%%%%%%%%%%%%%%%
\section{Vernichtung durch Arbeit}
In den ersten Monaten des Jahres 1942 zeichnete sich ab, dass das Unternehmen Barbarossa keinesfalls siegreich enden würde und man sich militärisch und wirtschaftlich auf einen langfristigen Krieg einstellen musste. Für die Konzentrationslager hatte diese Umstellung zunächst organisatorische Konsequenzen, in dem die \emph{Inspektion der Konzentrationslager} als Amtsgruppe D in das Wirtschaftverwaltungs-Hauptamt (WVHA) eingegliedert wurde. Es galt also neben der politischen, repressiven und vernichtenden Funktion der Konzentrationslager zunehmend deren ökonomisches Potential in Form von Arbeitskräften effizienter zu nutzen.
Oswald Pohl\index{p}{Pohl, Oswald}, der Chef des WVHA, beriet sich im Frühjahr 1942 mit dem Rüstungsministerium und der Rüstungsindustrie über den Einsatz von jüdischen Zwangsarbeitern, woraufhin im Herbst desselben Jahres Vereinbarungen getroffen wurden, KZ-Häftlinge künftig an die Rüstungsindustrie zu vermieten. Eines der ersten Außenlager  entstand daraufhin als Nebenstelle des KZ Groß-Rosen in Breslau-Lissa\index{o}{Breslau-Lissa}. Fortan wuchs die Zahl der KZ-Außenlager, die man in der Nähe der Industriebetriebe, also auch innerhalb der Reichsgrenzen errichtete\footnote{Vgl. Karin Orth in: \glqq Die Täter der Shoah\grqq, Paul (Hgs.), S. 93: \glqq Experten des Terrors. Die Konzentrationslager-SS und die Shoah\grqq.}.
~\newline
Der von Goebbels\index{p}{Göbbels, Josef} ausgesprochene Gedanke Hitlers\index{p}{Hitler, Adolf} von der \glqq Vernichtung durch Arbeit\grqq~erfoderte seitens der Lagerkommandanten ein Umdenken in Bezug auf den Umgang mit Häftlingen.
Der Leiter der Inspektion der Konzentrationslager befahl im Januar 1943 deshalb in einem Schreiben an alle KZ-Kommandanten \glqq für die Erschöpfung jeder Möglichkeit zur Erhaltung der Arbeitskraft der Häftlinge persönlich Verantwortung zu tragen\grqq. Was folgte waren Auseinandersetzungen zwischen den Kommandanten, die sich, im Bestreben die Sterblichkeit in ihrem Lager zu senken, gegenseitig kranke Häftlinge zuschoben. Die so genannter Liquidierung unnützer Esser, die doch den arbeitsfähigen Häftlingen die Lebensmittel wegaßen und Platz wegnahmen, stand im Gegensatz zur Verringerung der Sterblichkeit.
Dies bedeute: \glqq Häftlingsarbeitskraft einfach mit Leben und Tod kalkulieren, als ob es um Maschinen ginge, die entweder funktionieren oder auf den Schrott gehören\grqq~\footnote{Miroslaw K\'arn\'y: \glqq Vernichtung durch Arbeit\grqq, S. 150 in: Sozialpolitik und Judenvernichtung.}. Neue Arbeitskräfte waren schwierig und zeitaufwendig zu beschaffen. Sicherheitsverwahrte, Juden, Zigeuner, Russen und Ukrainer, auch Polen mit Haftstrafen über drei Jahren, Tschechen oder Deutsche mit Haftstrafen über acht Jahre sollten ohne Rücksicht auf andere staatliche Instanzen restlos ausgeliefert werden\footnote{Miroslaw K\'arn\'y: \glqq Vernichtung durch Arbeit\grqq, S. 135f, in: Sozialpolitik und Judenverfolgung.}. Der Kampf und das Gerangel von Wirtschaft und SS, um ausreichend gute Arbeitskräfte hatte zu dem Zeitpunkt längst begonnen\footnote{Vgl. Hermann Kaienburg (Hrsg.): \glqq Wie konnte es soweit kommen?\grqq~, S. 267 in: Konzentrationslager und deutsche Wirtschaft.}.
~\newline
Zwar machten die KZ-Häftlinge in der Arbeitskräftebilanz 1944 lediglich ein Prozent der Belegschaft\footnote{Nicht mit eingerechnet ist die Zahl der Fremdarbeiter, so dass sich bei Berücksichtigung derselben der Prozentsatz der KZ-Häftlinge verringern würde. Vgl.: Miroslaw K\'arn\'y: \glqq Vernichtung durch Arbeit\grqq, S. 133f in: Sozialpolitik und Judenvernichtung.} aus, doch kam ihnen in mehrfacher Sicht eine Schlüsselposition zu. Zum einen gab es eine wachsende Zahl von Arbeiten, die aufgrund ihrer Schwere und Gefährlichkeit ausschließlich von \glqq KZ-mäßig\grqq disziplinierten Arbeitern verrichtet werden konnten, und zum anderen verübten die KZ-Häftlinge ein abschreckende Wirkung auf die Bevölkerung und trugen somit dazu bei, die Kriegswirtschaft aufrecht zu erhalten.\newline Ungeachtet des Anteils an der gesamten Arbeitskräftebilanz betrug der Anteil der Häftlinge an der Gesamtbelegschaft eines Unternehmens oftmals mehr als nur ein Prozent (in der Görlitzer WUMAG sogar 12,5\%\footnote{Angabe vom am 1.11.1944. StArchD 11693 / 1258 Bild-Chronik III.}), entsprechend bedeutsam war deshalb ihre Arbeitsleistung.
~\newline
Die Arbeit der Häftlinge war jedoch grundsätzlich verschieden geartet. Zum einen waren es Leistungen für die SS (vgl. Lagerschreiber auf S. ~\pageref{schreiber}) oder innerhalb des Lagers, als Dienst für die Gefangenen (vgl. z.B. Häftlingsärtze S.~\pageref{arzt}). Der größte Teil der Häftlinge, zumindest jener in den KZ-Außenlagern, schuftete in sogenannten Fachkommandos innerhalb der Rüstungsindustrie. Die direkt zur Vernichtung führende und sinnlose Arbeit sowie die Ausprägungen zum Widerstand gegen Arbeit, soll erst im nächsten Abschnitt behandelt werden.


%%%%%%%%%%%%%%%%%%%%%%%%%%%%%%%%%%%%%%%%%%%%%%%%%%%%%%%%%%%%%%%%%%%%%%%
%%%%%%%%%%%%%%%%%%%%%%%%%%%%%%%%%%%%%%%%%%%%%%%%%%%%%%%%%%%%%%%%%%%%%%%%%%
%%%%%%%%%%%%%%%%%%%%%%%%%%%%%%%%%%%%%%%%%%%%%%%%%%%%%%%%%%%%%%%%%%%%%%%%%%%%
\subsection{Die Häftlingsarbeit in der Waggon- und Maschinenbau AG}

%%%%%%%%%%%%%%%%%%%%%%%%%%%%%%%%%%%%%%%%
% Aktiengesellschaft, Aufsichtsrat
Die Geschichte der Waggon- und Maschinenbau Aktiengesellschaft Görlitz, kurz WUMAG\footnote{Nicht zu verwechseln mit anderen gleichnamigen Unternehmen, wie etwa der WUMAG Bautzen\index{o}{Bautzen}. Die Görlitzer WUMAG hatte keine größeren Produktionsstätten außerhalb der Stadt.}, geht ursprünglich auf die Christoph-Lüders-Werke zurück. Im Jahre 1849 baute Christoph Lüders in Görlitz die ersten Schienenfahrzeuge für Holztransporte der Stadt. 1861 entstand aus der Firma eine Aktiengesellschaft für die Fabrikation von Eisenbahnmaterial. 1916--1918 erfolgte eine völlige Reorganisation der bis dahin verstreuten Betriebe. Durch die Fusion mit der \emph{Cottbuser Maschinen-Anstalt und Eisengießerei AG} und der \emph{Görlitzer Maschinenbau AG} entstand Anfang 1921 die Waggon- und Maschinenbau Aktiengesellschaft Görlitz\footnote{Vgl. Wolfgang Theurich: 160 Jahre Waggonbau in Görlitz. S. 96. Vgl. auch: Jürgen Schröder: 50 Jahre WUMAG 1948--1998.}. Weitere Firmenfusionen folgten in den darauffolgenden Jahren.
 Die Produktpalette der Görlitzer Abteilungen reichte in Friedenszeiten von Dampfturbinen, Dieselmotoren und Kraftwagenaufbauten bis hin zu den damals weithin bekannten Eisenbahn- und Straßenbahnwagen\footnote{Ernst Kretzschmar: Görlitz unter dem Hakenkreuz, S. 30.}. \newline

%Um gegenüber der Reichsbahn, dem einzigen Großkunden in der Waggonbausparte, eine bessere Verhandlungsposition einnehmen zu können, gründete man 1922 gemeinsam mit sieben weiteren Waggonbauunternehmen die Eisenbahn-Liefergemeinschaft (\glqq Eislieg\grqq)\footnote{Gründungsmitglied der Eislieg war neben der WUMAG Görlitz auch die Uerdinger und Düsseldorfer Waggonbaufabrik unter der Leitung von Ernst Schröder. Jürgen Schröder: 50 Jahre WUMAG. 1948--1998.}.

Geschäftliche Beziehungen bestanden mit zahlreichen europäischen und russischen Kunden. Die Aktienmehrheit am Unternehmen besaß die Deutsche Bank in Berlin\index{o}{Berlin} (siehe Tabelle~\ref{anteilseigner}) sowie Görlitz\footnote{StArchD 11693 / 1258 Bild-Chronik III. Die Namen der sonstigen Anteilseigner lauten wie folgt: Direktor Hartmut Reimann (Berlin), Karl Börner (Berlin), Conrad Pätzold (Berlin), Dresdner Bank (Dresden), Allgemeine Deutsche Kreditanstalt (Dresden), Creditanstalt Bankverein (Wien), Pommersche Bank (Stettin), Reichs-Kredit-Gesellschaft A.G. (Berlin), Baurat Karl Fausel (Berlin), Delbrück, Schickler und Co. (Berlin), Berliner Handelsgesellschaft (Berlin), Niederlausitzer Bank (Cottbus), G. Hasslinger Söhne (Berlin) u.a. Jede Aktie hatte einen Nennwert von 100 Reichsmark.}. Beide Banken bestimmte maßgeblich die Zusammensetzung des Vorstands und konnte somit entscheidenden Einfluss auf die unternehmerischen Aktivitäten nehmen.
~\newline
Eine Beteiligung des Flick-Konzerns, wie die Historiker Gräfe und Töpfer behaupten, kann ausgeschlossen werden\footnote{Vgl. Gräfe / Töpfer: Ausgesondert und fast vergessen. Die beiden Historiker schließen ausgehend von der Zugehörigkeit der Bautzner WUMAG zum Flick-Konzern auch auf die Görlitzer WUMAG. Dieser Schluss ist jedoch eine Verwechslung, die offenbar auf der Gleichnamigkeit der beiden Unternehmen beruht.}. Im Vorfeld der Machtergreifung der Nationalsozialisten flüchtete der bisherige Vorstandsvorsitzende Hanns Tillmanns -- er war sog. Halbjude -- nach Shanghai\index{p}{Tillmanns, Hanns}\footnote{Vgl. Wolfgang Theurich: 160 Jahre Waggonbau in Görlitz. S. 100.}. Dem Vorstand gehörten seit 1932 der Vorsitzende Conrad  Geerling\index{p}{Geerling, Conrad}, sein Stellvertreter Dr. Hans Kuba\index{p}{Kuba, Dr. Hans} sowie Ottokar Dietrich\index{p}{Dietrich, Ottokar} (später: Johannes H. Meyer\index{p}{Meyer, Johannes H.}), Hans Gräfe\index{p}{Gräfe, Hans} (später ausgeschieden) und Walter Wolf\index{p}{Wolf, Walter} (später ausgeschieden) an\footnote{Vgl. StArchD 11693/1258.}. Bemerkenswert ist auch die Besetzung des Aufsichtsrats, dem bis zu seinem Dienstausscheiden am 19. Juli 1944 auch der Görlitzer Oberbürgermeister Ernst Leichtenstern\index{p}{Leichtenstern, Ernst} angehörte\footnote{Ernst Leichtenstern gehörte bis zum 19.07.1944 dem Aufsichtsrat an. Seit 7. April 1941 hielt er das Amt des Görlitzer Oberbürgermeisters. Während seiner Amtszeit wurden alle in der Stadt verbliebenen Juden ins Zwangsarbeiterlager Tormersdorf bei Rothenburg (Sachsen) deportiert und das KZ-Außenlager im Biesnitzer Grund errichtet. Ernst Leichtenstern ist am 7. April 1945 in Breslau gefallen.}. Weitere Mitglieder des Aufsichtsrates der WUMAG waren: Max Grunow, Vors. (Direktor), Dr. Paul Mojert (stellvertretender Vorsitzender), Dr. Ing. Conrad Herrmann (aus Berlin), Dr. Hans Körner (aus Berlin), Heinrich Otto (Görlitz), Gustav Pilster (am 1.8.1944 verstorben) und Theodor Viebeg (aus Berlin)\footnote{StArchD 11693/1258.}.
%%%%%%%%%%%%%%%%%%%%%%%
\addtocounter{footnote}{1}
\begin{table}
\centering
\begin{tabularx}{.8\textwidth}{Xrr}
\hline
\textbf{Anteilseigner}
& \textbf{Aktienwert / RM}
& \textbf{Anteil / \%}\\
\hline
Deutsche Bank Berlin\index{o}{Berlin} & 2.940.000 & 45,6 \\
Commerzbank Görlitz & 1.225.000 & 19,0\\
Dresdner Bank Berlin & 800.900 & 12,4 \\
Deutsche Bank Görlitz & 715.300 & 11,1 \\
sonstige	& 759.000 & 11,9 \\
\hline
\end{tabularx}
\label{anteilseigner}
\addtocounter{footnote}{1}
\caption{Anteilseigner der der Görlitz \mbox{WUMAG} zur Hauptversammlung am 15.12.1943}
\end{table}




%%%%%%%%%%%%%%%%%%%%%%%%
Bereits zu Beginn der 1930er Jahre produzierte die WUMAG Militärfahrzeuge\footnote{Vgl. Wolfgang Theurich: 160 Jahre Waggonbau in Görlitz. S. 100.}. In der Waggonbauabteilung zählten dazu u.a. \glqq Schallmess-, Funk- und Schützenpanzerwagen, MG-Wagen, Aufbauten für Krankenkraftwagen, Panzeraufbauten, Sonderanhänger, Ersatzfeldwagen, geländegängige Lastkraftwagen, Sanitätsschlitten, Straßenfahrzeuge und Überladebrücken\grqq\footnote{ebenda.}. Mit dem Ausbau der Ausbau der Kesselschmiede, der Errichtung eines Dieselmotorenprüfstands und einer Fließstraße unternahm die WUMAG seit Kriegsbeginn erhebliche Anstrengungen, die Produktion auf Kriegsgüter umzustellen (siehe Tabelle ~\ref{kriegsproduktion})\footnote{StArchD WUMAG 1120. Vgl. Ernst Kretzschmer: Görlitz unter dem Hakenkreuz.}. Zuschusszahlungen und günstige Kredite ermöglichten fristgerechte Lieferung an die Wehrmacht, dem größten Auftraggeber neben der Reichsbahn\footnote{Am 04. Dezember 1939 erfolgte beispielsweise eine Zuschusszahlung des Oberkommando der Marine von 875537 RM, um die Lieferungen an selbige zu beschleunigen.}.
%%%%%%%%%%%%%%%%%%%%%%%
\begin{table}
\begin{tabularx}{\textwidth}{lLL}
\hline
& \textbf{Maschinenbau}
& \textbf{Waggonbau}\\
\hline

\textbf{Marine}
& Dieselmotoren, Dampfmaschinen, Turbinen
& Bojen, WPD Gefäße\\

\textbf{Heer}
& Granaten, Oberlafetten, Pak 1938, Pack\-kist\-en, Tur\-bin\-en, Prüfpressen
& Spezialaufbauten, Sonderanhänger, pfer\-de\-be\-spannte Fahrzeuge, Tragegestelle für Tiere\\

\textbf{Luftwaffe}
& Instandsetzung von Flugmotoren, Fahr\-ge\-stel\-le für Flugzeuge
& Spezialaufbauten, Kesselwagen, fahrbare Anschlussgleise, Sonderanhänger\\
\hline
\end{tabularx}
\caption{Während des Krieges produzierte Güter%
}
\label{kriegsproduktion}
\end{table}


%%%%%%%%%%%%%%%%%%%%%%%
\subsubsection{Der Einsatz von KZ-Häftlingen}
Die ausgesprochen gute Konjunktur der Kriegsjahre ging einher mit dem Mangel an Arbeitskräften, die man wegen des Kriegsdienstes entbehren musste. Kriegswirtschaftsgesetze und \mbox{-verordnungen} konnten die Lücken in den Reihen der Belegschaft nicht schließen und belasteten stattdessen die Arbeiter und Angestellten. Arbeitszeitbeschränkungen und Arbeitsschutzbestimmungen wurden bei gleichzeitigem Einfrieren der Löhne und Gehälter aufgehoben\footnote{Erlass einer Lohnstopp-Anordnung vom Treuhänder Schlesien. Verbot von Lohn- und Gehaltserhöhungen nach dem 4.September 1939. Fortfall der Zahlungen von Zuschlägen für Überstunden, Nacht- und Schichtarbeit. StArchD 11693 WUMAG / 1258. Diese Reglementierungen wurden 1941 wieder aufgehoben. Vgl. Hermann Kaienburg: SS-Wirtschaft, S. 387.}. Vergeblich versuchte die Geschäftsleitung ihre Auftraggeber bei der Wehrmacht davon zu überzeugen, die dringend benötigten deutschen Fachkräfte vorerst nicht für den Fronteinsatz zu mobilisieren. Solch einem Anliegen konnte die Wehrmacht nur entgegenkommen, indem sie \glqq fremdländische\grqq~Arbeitskräfte aus den besetzten Gebieten nach Görlitz deportierte. Der Preis dafür bestand in der Errichtung von Baracken und in der Versorgung der Gefangenen. Trotzdem armortisierten sich die Kosten durch die günstigen Tarife, die die WUMAG für Kriegsgefangene und Zwangsarbeiter zu entrichten hatte. Diese Aufgaben- und Kostenteilung ging eindeutig zulasten der betroffenen Kriegsgefangenen und Zivilarbeiter, die unter der mangelnden Versorgungslage zu leiden hatten.
\newline
Bei einer Wirtschaftsprüfung im Herbst 1940 erhob man die WUMAG in den Stand eines wichtigen Kriegsbetriebes\footnote{StArchD 11693 WUMAG / 1120.}, infolge dessen auch erstmalig Juden in Kooperation mit der Organisation Schmelt zur Zwangsarbeit herangezogen werden konnten.
Seit dem Scheitern der militärischen Offensive gegen die Sowjetunion 1941 musste sich die deutsche Wirtschaft auf einen langen Abnutzungskrieg einstellen und gleichzeitig das größte Problem, den Arbeitskräftemangel, bewältigen. Gegenüber dem Ministerium für Rüstung und Kriegsproduktion hatten die Unternehmen diejenigen Arbeitskräfte anzufordern, die das ihnen auferlegte Produktionsprogramm erforderte\footnote{Zeugenaussage von Albert Speer 1947 im Prozess gegen den Großindustriellen Friedrich Flick. Vgl. Herbert Obenaus: Die Außenkommandos des Konzentrationslagers Neuengamme in Hannover, S. 215.}. Dies führte dazu, dass Mitte Mai 1944 nicht mal mehr jeder zweite der über 10.000 Mitarbeiter der WUMAG ein Deutscher war. Die Frage, warum es in der Folgezeit zum Einsatz von KZ-Häftlingen in der WUMAG (Tabelle ~\ref{wumag_pers} auf S. ~\pageref{wumag_pers}\footnote{StArchD 11693 / 1258 Bild-Chronik III.}) kam, lässt sich nur teilweise erörtern. Einerseits führte Albert Speer\index{p}{Speer, Albert}, der Reichsminister für Rüstung und Kriegsproduktion, 1947 als Zeuge im Prozess gegen den Großindustriellen Friedrich Flick\index{p}{Flick, Friedrich} an, dass ein Unternehmer \glqq grundsätzlich auch keinen Einfluss darauf hatte, ob er Zwangsarbeiter oder Häftlinge zugeteilt erhielt\grqq\footnote{Ebenda.}. Andererseits muss auch die WUMAG eine Bedarfsmeldung an den Jüdischen Ältestenrat der Organisation Schmelt in Sosnowiec gerichtet haben, anhand derer schriftliche Aufforderungen zum Arbeitseinsatz versandt wurde\footnote{Andrea Rudorff: Das Lagersystem der Organisation Schmelt in Oberschlesien. S. 156.}. Angesichts dieser Erfahrung im Einsatz von jüdischen Zwangsarbeitern konnte die Görlitzer WUMAG im Zuge der Auflösung der Organisation Schmelt das existierende Lager mit etwas Nachdruck in ein KZ-Außenlager umwandeln und die Genfangenenzahl erhöhen.
\\
Auf Drängen Albert Speers\index{p}{Speer, Albert} ging die SS bereits 1942 dazu über, KZ-Außenlager bei Rüstungsbetrieben einzurichten\footnote{Hermann Kaienburg: KZ-Haft und Wirtschaftsinteresse, S. 51.}. Die Zahl der externen Lager stieg ab 1942 zunächst langsam, ab 1944 jedoch explosionsartig an\footnote{Ebenda.}. Die vollständige Auflösung der Organisation Schmelt bedingte am 09. Juni 1944 die Übernahme des Arbeitslager Görlitz durch das KZ Groß-Rosen\index{o}{Groß-Rosen}.
~\newline
Wie diese Übernahme genau vonstatten ging, lässt sich heute nicht mehr vollständig nachvollziehen und dokumentarisch belegen. Dem SS-Wirtschaftsverwaltungshauptamt (WVHA), genauer der Amtsgruppe D II, oblag die Entscheidung zur \glqq Gestellung\grqq von KZ-Gefangenen für Wirtschaftseinsätze, die Einsatzplanung erfolgte jedoch hauptsächlich in Speers\index{p}{Speer, Albert} Ministerium für Rüstung und Kriegsproduktion\footnote{Ab Oktober 1944 mussten die Anträge auf die Häftlingsgestellung unmittelbar im Ministerium für Rüstung und Kriegsproduktion eingereicht werden. Vgl. Hermann Kaienburg: KZ-Haft und Wirtschaftsinteresse, S. 58.}. Obwohl dieses Projekt staatliche Instanzen planten, bedeutet dies nicht, die WUMAG von jeglicher Verantwortung freizusprechen. Ohne ihr Mitwirken wäre ein Arbeitseinsatz nicht vorstellbar gewesen. Es bedurfte ihrer Zustimmung und nicht zuletzt der Einreichung eines Antrages an das SS-WVHA\footnote{Vgl. Hermann Kaienburg: Wie konnte es soweit kommen?, S. 269.}. Es scheint insofern unglaubwürdig, dass die Initiative zum Einsatz von KZ-Häftlingen nicht von der WUMAG ausging. Aus ökonomischer Sicht war sie nämlich mit Vorteilen verbunden. KZ-Häftlinge waren günstiger als sowjetische Zwangsarbeiter (\glqq Ostarbeiter\grqq) und zudem in großen Gruppen verfügbar\footnote{Ostarbeiterentgelt und Ostarbeiterabgabe beliefen sich bei Hilfskräften zusammen auf 4,- RM bis 6,- RM, bei Facharbeitern auf 6,- RM bis maximal 8,- RM pro Arbeitstag. Vgl. Hermann Kaienburg: KZ-Haft und Wirtschaftsinteresse, S. 60.}. Der Historiker Hermann Kaienburg schrieb dazu: \glqq Sie waren durch geringfügige Maßnahmen zu disziplinieren. Die hohe Mobilität und die flexible Arbeitszeit gestattete einen sehr beweglichen Einsatz, und Geheimhaltung wurde erleichtert. Die Wirtschaftsunternehmen konnten dem KZ-Gefangenen ohne Rücksicht auf Leben und Gesundheit schwierige und gefährliche Arbeit übertragen.\grqq\footnote{Vgl. Hermann Kaienburg: KZ-Haft und Wirtschaftsinteresse, S. 59.}. Die Vorteile der KZ-Arbeit waren für das Rüstungsministerium und die Wirtschaft unübersehbar und grundsätzlich im Sinne \glqq des Interesses des Landes als einer kriegsführenden Nation\grqq\footnote{Vgl. Hermann Obenaus: Die Außenkommandos des Konzentrationslagers Neuengamme, S. 214.}.
\newline
Zur Bewilligung eines Antrages auf Häftlingsgestellung mussten die Anforderungen des WVHA erfüllt sein, bevor das Rüstungsministerium seine Zustimmung geben konnte. Die WUMAG hatte Unterkünfte zu stellen, die den Sicherheitsvorschriften der SS genügten. Das Barackenlager im Biesnitzer Grund entsprach weitestgehend den Erfordernissen\footnote{Die Errichtung der Baracken erfolgte nach Maßgaben des Reichsarbeitsministers am 30. Oktober 1941, wonach bauliche und Brand -- und Luftschutzmaßnahmen entsprechend umgesetzt werden mussten. LArchD 11693 WUMAG / 1302.} und konnte durch wenige bauliche Veränderungen, wie etwa der Anbringung eines Sichtschutzes und der Elektrisierung des Stacheldrahtzauns\footnote{Laut Aussage von Alfred und Hildegard Ecke, die als unmittelbare Anwohner durch diese Maßnahmen betroffen waren.}, den Vorschriften gerecht werden. \label{vorauskommando2}Die WUMAG bediente sich zur Erfüllung der sicherheitsvorschriften erstmals einiger KZ-Häftlinge aus Groß-Rosen\index{o}{Groß-Rosen}\footnote{Witbold Ziblinski, der erste Tote des KZ-Außenlagers Görlitz, starb laut Friedhofsverwaltung bereits am 3. August 1944, d.h. etwa 5 Tage vor Ankunft des ersten Gefangenentransportes. Er gehörte zum Vorauskommando.}. Des Weiteren hatte die WUMAG für Unterhaltung, insbesondere auch für die Beheizung der Räumlichkeiten Sorge zu tragen.
\newline
Die WUMAG selbst stellte auch Bedingungen, dabei ging es nicht allein um die Anzahl der Häftlinge, die in ihren Fabriken eingesetzt werden sollten, sondern auch um deren fachliche Qualifikation. Der ehemalige Häftling Paul Levi\index{p}{Levi, Paul} erinnert sich der Kommission in Auschwitz\index{o}{Auschwitz}, die arbeitsfähige Fachkräfte für den Transport nach Görlitz auswählte. \footnote{Paul Levi wurde als Elektromeister für den Abtransport nach Görlitz bestimmt. LArchB B Rep 058 Bd. 1.}. Es ist nicht unwahrscheinlich, dass dieser Auswahlkommission auch Vertreter der Görlitzer WUMAG angehörten oder zumindest deren Anforderungen für die Selektion umgesetzt wurden.

%%
% Ben Munn transcript about Auschwitz
%%


\begin{table}
\centering
\begin{tabularx}{.8\textwidth}{Lrrr}
\hline
Datum & Waggonbau & Maschinenbau & gesamt\\
\hline
1. September 1944  & 200		& 450 	& 650 \\
1. Oktober 1944  & 399 	& 1050 	& 1449\\
1. November 1944  & 399 	& 1050  & 1449\\
1. Dezember 1944  & 396 	& 1006  & 1402\\
1. Januar 1945  & 376 	& 973  	& 1349\\
1. Februar 1945  & 350 	& 928 	& 1278\\
\hline
\end{tabularx}
\label{wumag_pers}
\caption{
Die Anzahl der unter \glqq Sonstige\grqq~geführten Juden in der Gefolgschaftsaufteilung der WUMAG}
\end{table}

Im Sinne der Aufgaben- und Kostenteilung oblag es der SS, die Häftlingstransporte zu organisieren und deren Sicherheit zu gewährleisten. Ebenso die Bewachung innerhalb der Lager sowie auf dem Weg zum Arbeitseinsatz gehörte zu den Aufgaben des dafür bereitgestellten SS-Bataillons. Während der Arbeit übernahm der Werksschutz der WUMAG zusammen mit den Kapos die Überwachung der Häftlinge.
~\newline
Wie auch in anderen KZ-Außenlagern, ist im Fall Görlitz davon auszugehen, dass sich die WUMAG schriftlich gegenüber dem KZ Groß-Rosen\index{o}{Groß-Rosen} verpflichtete, für Licht, Wasser, Beheizung und Desinfektion (\glqq Entwesung\grqq)\footnote{Laut Aussage von Samuel Reifer, ZIH 301/2311.} zu sorgen, die Einrichtung von Häftlingsküche und -krankenrevier bereitzustellen, und die Kleidung der Häftlinge zu reinigen\footnote{Vgl. Hermann Kaienburg: Wie konnte es soweit kommen?, S 273.}.


\myfigure[maschinenbaufoto]{wumag1}{Luftbilddatenbank, Ing.-Büro Dr. H.G. Carls.}{Luftaufnahme der Maschinenbau-Werke der WUMAG, 30. Mai 1944}{Luftbild Maschinenbau, 30. Mai 1944}{0}

\myfigure[waggonbaufoto]{wumag2}{Luftbilddatenbank, Ing.-Büro Dr. H.G. Carls.}{Luftaufnahme der Waggonbau-Werke der WUMAG, 30. Mai 1944}{Luftbild Waggonbau, 30. Mai 1944}{0}


%%%%%%%%%%%%%%%%%%%%%%%%%%%%%%%%%%%%%%%%
\subsubsection{Die Arbeit der KZ-Häftlinge in der WUMAG}
Bei Ankunft neuer Häftlinge im Görlitzer Lager erfolgte eine Selektion durch die Meister der Waggon- und Maschinenfabrik\footnote{Aussage von Samuel Reifer, ZIH 301/2311.}. Dabei suchte man nach Facharbeitern, die gezielt in den entsprechenden Abteilungen eingesetzt werden konnten. So zum Beispiel erhielt der Holzfachmann Abraham Rajchbart eine Stelle in Tischler-, Schlosser- und Drechslerwerkstätten. Ungelernte wurden in der Regel als Hilfsarbeiter eingestellt.
\newline
Der Einsatz der maximal 1449 KZ-Gefangenen erfolgte an vielerlei Orten innerhalb der Werke, zumeist jedoch im Maschinenbau (Bild~\mypicsref{maschinenbaufoto} und Karte~\mymapsref{maschinenbauskizze} ) sowie, etwas weniger zahlreich, auch im Waggonbau (Bild ~\mypicsref{waggonbaufoto}). Ferner auch beim Abriss des werksgehörigen \glqq Reichertlager\grqq\footnote{Das sogenannte Reichertlager befand sich in der Reichertstraße unweit der Maschinenbauabteilung der WUMAG. Dieses Lager diente der Unterbringung von Zwangsarbeitern verschiedener Nationalitäten. Aussage von Artur Berndt in: Roland Otto: Die Verfolgung der Juden in Görlitz unter der faschistischen Diktatur, S. 67. Vgl. auch: Aussage von Gedalia Pilo. LArchB B Rep 058 Bd. 1.}.

Die Art der Arbeit hatte weitreichende Folgen für die Existenzbedingungen und Überlebenschancen der Häftlinge. Die meisten Arbeiten innerhalb der WUMAG waren mit großen körperlichen Anstrengungen verbunden. Hinzu kamen gesundheitliche Belastungen und Gefahren, wie große Hitze am Glühofen oder an der Stanze\footnote{Aus den Erinnerungen von Artur Berndt. Vgl. Ernst Kretzschmar: Görlitz unter dem Hakenkreuz, S. 56.}, schadstoffbelastete Luft bei Schweißarbeiten, und chemische Gifte bei der Panzerlackierung mit Aceton\footnote{Aussage von Jakob Kinrus. RAG Sammlungsgut KZ Biesnitzer Grund.}. Ungenügend gesicherte Maschinen bedingten Unfälle, wie einige Aussagen belegen\footnote{Unter anderem die Beinamputation von Helmann Meiyer, die von George Pilc bekundet wird. LArchB B Rep 058 Bd. 3.}. Arbeitsschutzmaßnahmen galten für die KZ-Gefangenen offenbar nicht:

\begin{leftbar}
Vorher habe ich bei den Panzern gearbeitet, die wurden mit Acetonspray bespritzt, um sie zu tarnen. Die Deutschen, die dort gearbeitet haben, trugen Masken, uns haben sie nichts gegeben. Das war Gift. Ja, der Mensch hat einfach so das Gift eingeatmet. Ich habe dort zwei Tage gearbeitet und schon gesehen, dass wenn ich noch den dritten Tag arbeite, bin ich erledigt.\footnote{Jakob Kinrus bat den Abteilungsleiter um Versetzung. Dieser willigte aus Mitleid ein und versetzte Jakob Kinrus an einen anderen Arbeitsplatz. Vgl. RAG Sammlungsgut KZ Biesnitzer Grund.}
\end{leftbar}

\begin{leftbar}
Goerlitz wasn't to kill, it was only to work, and to die.\footnote{\glqq In Görlitz ging es nicht ums Töten, sondern nur um das Arbeiten und sterben.\grqq, Zitat von Leon Hostig, In: David van Biema: \glqq Poisoned Lives\grqq.}
\end{leftbar}

\subparagraph{Arbeit bei privaten Handwerksbetrieben}
Die WUMAG beschäftigte\label{michel} während des Krieges eine Reihe kleinerer Handwerksbetriebe innerhalb der Produktionsstätten. Der Arbeitskräftemangel bereitete den Betrieben Schwierigkeiten bei der Erfüllung von Aufträgen, weshalb die WUMAG kleinere Häftlingskommandos zur Unterstützung der Arbeiten anbot. Nachweislich wurden 1944 einige Gefangene der Baufirma Vogt zugeteilt, als selbige mit dem Bau von Unterständen im Maschinenbau (Seiteneingang Melanchtonstraße / Ecke Reichertstraße) beauftragt war\footnote{Nach Überlieferungen von Herrn Scholz aus Königshain, dessen Vater bei der Firma Vogt angestellt war.}. Auch der Elektrotechniker Gustav Michler\index{p}{Michler, Gustav} verfügte über Arbeitskräfte aus dem KZ-Außenlager\footnote{Paul Levi arbeitete anfangs bei der WUMAG und später bei der Firma Michler als Elektrotechniker. LArchB B Rep 058 Bd. 5. Die Firma Michler hatte damals auf der Jochmannstraße 11 ihren Sitz. Vgl. RAG Görlitzer Adressbuch 1941--1942.}.

\begin{landscape}
\mymapfigure[maschinenbauskizze]{map_maschinenbau}{}{Skizze der Maschinenbauwerke der WUMAG 1944/45}{Skizze der Maschinenbauwerke der WUMAG 1944/45}{0.519}{}
\end{landscape}

%%%%%%%%%%%%%%%%%%%%%%%%
\paragraph{Ausnutzung der Arbeitskraft}
In einem von Oswald Pohl\index{p}{Pohl, Oswald}, dem Chef des SS-WVHA, an die KZ-Kommandanten gerichteten Schreiben vom 22.11.1943 heißt es, dass \glqq die für Häftlinge befohlene Arbeitszeit von täglich 11 Stunden [...] eingehalten werden muss\grqq\footnote{Vgl. Hermann Kaienburg: SS-Wirtschaft, S. 114.}. Die KZ-Gefangenen in der WUMAG arbeiteten an sechs Tagen in der Woche in zwei Schichten, tags und nachts. Die Tagschicht begann um 5:30 und endete um 16:30\footnote{Die Arbeitszeiten unterschieden sich vermutlich je nach Arbeitskommando. Vielfach wird auch von 12 Stunden Arbeit pro Tag berichtet, wobei ungewiss ist, ob die Wegstrecke zeitlich mit einbezogen ist.}. Im Anschluss daran begann die Nachtschicht\footnote{Aussage von Israel Braun. LArchB B Rep 058 Bd. 6}. Am Sonntag waren die Häftlinge zumindest von den Arbeiten in der Fabrik befreit. Israel Braun sagte aus:

\begin{leftbar}
Sonntags sollte Ruhetag sein, doch man hielt uns auch dann beschäftigt: so mussten wir zum Beispiel Steine von einem Haufen zum anderen schleppen, hier hatten meist die Kolonnenältesten, die uns wochentags zur Arbeit begleiteten, das Sagen.\footnote{LArchB B Rep 058 Bd. 6.}
\end{leftbar}

Gemessen an den Arbeitszeiten der freien Arbeiter, die im Zuge des \glqq totalen Kriegseinsatzes\grqq~überlange Arbeitszeiten von bis zu 60 Stunden in der Woche hinnehmen mussten, mag die Häftlingsarbeitszeit von über 66 Stunden nicht so diskriminierend erscheinen. Dabei gilt es aber die bis zu vier Kilometer lange Marschstrecke zum Arbeitsplatz und die stundenlangen Appelle vor und nach Arbeitsbeginn zu berücksichtigen. Man kann davon ausgehen, dass die SS darum bemüht war, die Ruhezeiten auf ein Minimum zu reduzieren. Israel Braun sagte aus:

\begin{leftbar}
Nachtschichtler sollten tagsüber eigentlich Ruhe haben, wurden jedoch oft noch zu anderen Arbeiten wie Kohlen ausladen, Saubermachen, Kartoffeln schleppen herangezogen.\footnote{Aussage von Israel Braun. LArchB B Rep 058 Bd. 1.}
\end{leftbar}

Unter Einbeziehung der unzureichenden Verpflegung und der durchweg schlechten körperlichen Verfassung der Häftlinge verbietet deren Situation jeglichen Vergleich mit den freien Arbeitern. Die Auswirkungen dessen führten zu Entkräftung und nachlassender Arbeitsleistung, wenn nicht sogar zum Tode. Verschiedene Schätzungen kamen bereits in der ersten Kriegshälfte zu dem Schluß, dass die Arbeitsleistung von KZ-Häftlingen im Vergleich zu freien Arbeitern, je nach Arbeitsbedingungen, zwischen 5 und 50\% beträgt \footnote{Vgl. Hermann Kaienburg: SS-Wirtschaft, S. 115.}. Folglich stieg der Bedarf an Arbeitskräften, um die Produktionsziele verwirklichen zu können.

Dieser Tatsachen muss sich die Geschäftsleitung der WUMAG bewußt gewesen sein, denn nicht ohne Grund zahlte sie Prämien\label{pramie} im Wert von 1 bis 3 Reichsmark, wenn mehr als das vorgeschriebene Arbeitssoll erfüllt wurde (siehe S.~\pageref{pramien}).

Nicht sicher nachgewiesen werden konnte, ob sich die WUMAG entgegen der sonst üblichen Aufgabenteilung zwischen SS und Wirtschaft, dazu bereit erklärte, einen Teil der Essensrationen zu übernehmen. In Berichten ehemaliger Häftlinge ist immer wieder die Rede von Essenszuteilung seitens der WUMAG\footnote{Israel Braun sagte aus: \glqq Samstag erhielten sie Essen direkt aus der Lagerküche, weshalb die Schüssel mit auf Arbeit genommen werden mußte\grqq. LArchB B Rep 058 Bd. 1.}.

%%%%%%%%%%%%%%%%%%%%%%%%%%%%%%%%%%%%%%%%
\label{muller}
\paragraph{Umgang mit den Gefangenen}
Die Arbeitskräfte aus den Konzentrationslagern bildeten auch im Waggon- und Maschinenbau die unterste Stufe in der nach rassenideologischen Kriterien aufgebauten Sozialordnung.
Die Haltung der Aufsichtskräfte gegenüber den KZ-Häftlingen war unterschiedlich und durchaus entscheidend für die Existenzbedingungen der Gefangenen\footnote{ZIH 301/715.}. Ein Meister namens Müller\index{p}{Müller} erstellte täglich einen Bericht über Leute, die ungenügend arbeiteten, was eine Bestrafung auf dem Appellplatz des Lagers zur Folge hatte. Das Weisungsrecht der Werksleitung und der ihr unterstellten Aufsichts- und Führungskräfte, also auch der Meister, beschränkte sich auf die Zuweisung von Arbeit und auf fachliche Anordnungen und Aufsicht\footnote{Vgl. Hermann Kaienburg: SS-Wirtschaft, S. 108.}. Die Mißhandlung eines Häftlings durch die Meister war demnach untersagt, wenngleich es Ausnahmen gab, in denen die Meister darüber hinaus, ihre Untergebenen auf brutalste Art und Weise folterten\footnote{ZIH 301/2311, 301/715 und 2765.}. Oftmals wurden Arbeitsverstöße durch die Kapos mit 25 Peitschenhieben bestraft, die Meister tolerierten dieses Vorgehen\footnote{ZIH 301/715.}. Abgesehen von der Denunziation und Menschenverachtung dieser Vorgesetzten, hinderte ihr Handeln gleichzeitig die Erfüllung der Produktionsvorgaben. Man könnte diese Vorgehensweise auch als Sabotage an Arbeitskräften bezeichnen -- paradoxer Weise wurden die betroffenen Häftlinge oftmals selbst der Sabotage bezichtigt.
\begin{leftbar}
Einmal, während der Nachtschicht, wurden ein paar durchbohrte Scharniere neben meiner Maschine gefunden. Müller\index{p}{Müller} verlangte, dass ich ihm sage, wer dies getan hat. Ich wusste nichts davon, doch Müller wollte eine Antwort aus mir herausprügeln, er trat mich und schlug mir ein paar mal ins Gesicht.\newline
Am folgenden Tag befahl mir Müller\index{p}{Müller}, gleich nachdem ich zur Arbeit kam, in den Keller zu gehen, wo er mich in die Hände von Kapo Spalter\index{p}{Spalter}, aus Chrzanow [Krenau (Chrzanow, Polen)]\index{o}{Krenau}, fallen ließ und meinte, dass ich meine Schuld nicht zugeben würde. Spalter sollte eine Antwort aus mir herausprügeln. Ich wiederholte, dass ich von nichts wusste. Dann legten sie mich über einen Stuhl, zwei jüdische Vorarbeiter hielten mich fest. Einer hielt meinen Kopf, der andere meine Beine. Spalter schlug mich mit einem Gummiknüppel. Ich bekam 20 Schläge auf meinen nackten Rücken. Müller nahm ihm dann den Gummiknüppel ab und gab ihm einen dicken Holzknüppel. Spalte hielt das Knüppel mit beiden Händen und schlug so heftig auf mich ein, dass das Knüppel an mir zerbrach. Ich war blutüberströmt und konnte mich nicht von der Stelle bewegen. Halb lebendig brachte man mich in die Fabrik, wo ich gekrümmt neben meiner Maschine stand und vor Schmerzen weinte.\newline
Um ein Uhr konnte man im Keller Mittagessen gehen, aber ich konnte nicht runter gehen, so blieb ich am Arbeitsplatz. Als mich ein SS-Mann fragte, warum ich nicht zum Mittagessen gehen würde, zeigte ich ihm meine Wunden. Er rief Spalter\index{p}{Spalter} und brüllte ihn an, dass es ihm nicht gestattet ist, einen Häftling in solch einer Weise zu schlagen. In diesem Moment kam Müller mit ein paar Vorarbeitern hinzu. Ich widersprach dem und erzählte, was Müller Spalter gesagt hat. Ich lies meine Hosen herunter und zeigte den an der Arbeitsstelle anwesenden Personen öffentlich meine Wunden von den Schlägen. Müller errötete und sagte: \glqq Weg mit dem Schwein\grqq. Sie brachten mich in den Keller, dort lag ich bis zum Ende des Arbeitstages auf dem Bauch. Müller\index{p}{Müller} kam dann in den Keller und brüllte, dass ich simuliere. Er begann mich wieder zu treten und zu schlagen, schließlich gingen wir zu einem Lastwagen. Man brachte mich ins Lager. Müller\index{p}{Müller} befahl, mich an den Lastwagen zu binden und ins Lager zu laufen. Als er weg war, setzten mich Kollegen auf die Ladefläche. Halb lebendig fuhren sie mich ins Dispensatorium [Krankenrevier]. [...]
Müller\index{p}{Müller} konnte lange Zeit nicht vergessen, wie ich mit runter gelassenen Hosen vor all den Fremden stand. Nach dieser unmenschlichen Prügelei musste ich sechs Wochen auf meinem Bauch schlafen. Es ist bis heut noch nicht wieder gut.\footnote{Samuel Reifer: Jüdisch Historisches Institut Warschau, 301/2311.}
\end{leftbar}

\paragraph{Kriegsprofiteure?}
Das Geschäft mit Rüstungsgütern sowie die durch Kriegswirtschaftsgesetze und den Einsatz von Zwangsarbeitern gesunkenen Personalkosten wirkten sich positiv auf die Geschäftsbilanzen der Kriegsjahre aus (siehe Tabelle~\ref{umsaetze} auf S.~\pageref{umsaetze})\footnote{StArchD 11693 / 1258 Bild-Chronik III. Vgl. Ernst Kretzschmar: Görlitz unter dem Hakenkreuz, S. 57.}. Trotz erheblicher Umsatzsteigerungen konnten jedoch keine beträchtlichen Gewinn zunahmen verzeichnet werden. Dies mag zum einen an den erheblichen Investitionen im Produktionsbereich\footnote{Umstellung auf Kriegsproduktion und Erweiterung von Produktionslagern für Rüstungsgüter.} und der geringen Gewinnspanne der Kriegsgüter liegen, zum anderen aber auch am Einsatz von weniger zuverlässigen Zwangsarbeitern. Abgesehen von der Errichtung der Barackenlager\footnote{In einem Geschäftsbericht vom 15. März 1944 werden 23 Barackenlager erwähnt. StArchD 11693 WUMAG / 1027.} und den resultierenden Unterhaltskosten konnte der schlechte Gesundheitszustand der Gefangenen, das von ihnen eingeschleppte Ungeziefer\footnote{Samuel Reifer, ZIH 301/2311.}, sowie die rigorosen Sicherheitsvorkehrungen\footnote{Zum Beispiel Isolierung gegenüber anderen Zwangsarbeitergruppen bzw. deutschen Arbeitern. Vgl. Schlomo Graber: Schlajme, S. 74.} den Arbeitsablauf negativ beeinflussen. Transporte in andere Lager und die vielen Todesfälle führten zu einer hohen Fluktuation unter den sogenannte fremdvölkischen Arbeitskräften\footnote{Der ehemalige SS-Wachmann Karl Kruner sagte aus: \glqq~Im Werk war allerdings bekannt, dass unter ihnen laufend Ausfälle dadurch zu verzeichnen waren, dass sie des Hungertodes starben.\grqq (BStU MfS ASt 13/48 Bd. 2 / 386).}. Die Arbeitsproduktivität der KZ-Gefangenen war alles andere als überdurchschnittlich\footnote{Die ähnlich schlecht gestellten sowjetischen Zwangsarbeiter galten in einem Monatsbericht der WUMAG aufgrund ihres schlechten Gesundheitszustandes als nur 50\% leistungsfähige Kräfte. StArchD 11693 WUMAG / 1027.}. Hinzu kamen Betriebsstörungen und schwierig zu kontrollierende Sabotageakte (siehe S.~\pageref{loetzinn}).

Bei der Hauptversammlung am 15. Dezember 1943 entschieden sich die Aktionäre für eine nahezu vollständige Gewinnausschüttung; wahrscheinlich ohne auch nur einen Gedanken an die missliche Lage jener Zwangsarbeiter zu verschwenden. Von den 863.184,29 Reichsmark Reingewinn wurden 700.000 RM den Stammaktionären (zuzüglich 50.000 RM Dividendenabgabe an den Staat) und etwa 63.432 RM dem Aufsichtsrat vergütet\footnote{StArchD 11693 WUMAG / 1258.}.

%%%%%%%%%%%%%%%%%%%%

\begin{table}
\centering
\begin{tabularx}{.6\textwidth}{Xrr}
\hline
\textbf{Geschäftsjahr} & \textbf{Umsatz / RM} & \textbf{Gewinn / RM}\\
\hline
1934 / 1935 & & 259.783,34\\
1935 / 1936 & & 341.408,56\\
1936 / 1937 & & 478.545,34\\
1937 / 1938 & & 480.414,68\\
1938 / 1939 & 29.439.814,53 & 629.098,76\\
1939 / 1940 & 37.428.773,42 & 631.219,19\\
1940 / 1941 & 50.965.594,40 & 752.734,11\\
1941 / 1942 & 61.925.061,00 & 829.736,29\\
1942 / 1943 & 57.913.714,61 & 863.184,29\\
1943 / 1944 & 110.150.000,00 & 819.586,77\\
\hline
\end{tabularx}
\caption{Auszug aus den Bilanzen der WUMAG\label{umsaetze}}
\end{table}



Zumindest dem Vorstandsvorsitzenden\index{p}{Geerling, Conrad} gelang es während der Kriegsjahre, genügend Kapital anzuhäufen, um sich kurz vor Kriegsende\footnote{Vgl. Wolfgang Theurich: 160 Jahre Waggonbau in Görlitz. S. 155.} rechtzeitig in die westlichen Besatzungszonen abzusetzen und das Geschäft mit dem Waggon- und Maschinenbau wieder aufleben zu lassen.
\newline
Bei einer Zusammenkunft des ehemaligen WUMAG Generaldirektors\footnote{Entspricht der Funktion des Vorstandsvorsitzenden.} Conrad Geerling mit Ernst Schröder\index{p}{Schröder, Ernst}, seinerseits Generaldirektor der Uerdinger und Düsseldorfer Waggonfabrik, berichtete Geerling\index{p}{Geerling, Conrad}, dass er im April 1952 mit einer Gruppe von Mitarbeitern aus Görlitz, insbesondere Monteure, einen neuen Betrieb in Hamburg\index{o}{Hamburg} aufgebaut hatte, der sich zunächst mit der Reparatur und Instandhaltung Görlitzer Produkte befasste\footnote{Vgl. Jürgen Schröder: 50 Jahre WUMAG, 1948--1998. Vgl. auch MTZ52, S. 107.}.
Geerling\index{p}{Geerling, Conrad} wagte es 1952 nicht, den Namen WUMAG anzunehmen, wenngleich er plante, deren Produkte zu fertigen. Das Unternehmen nannte sich, entsprechend einer ehemaligen Tochtergesellschaft, \glqq Gewerkschaft Union Naphta, Abteilung Waggon- und Maschinenbau\grqq\footnote{Ebenda.} und hatte annähernd 2000 Mitarbeitern.
\newline
Den eben erwähnten Ernst Schröder\index{p}{Schröder, Ernst} verband eine Geschäftsfreundschaft mit Conrad Geerling\index{p}{Geerling, Conrad}, die auf die Gründung der Eisenbahn-Liefergemeinschaft (\glqq Eislieg\grqq) 1922 zurückgeht. Der Sohn Ernst Schröders\index{p}{Schröder, Ernst}, Günther Schröder,\index{p}{Schröder, Günther} stieg in das Hamburger\index{o}{Hamburg} Geschäft ein, was schon bald wieder den Namen WUMAG (Waggon- und Maschinenbau GmbH) trug. Die WUMAG produzierte Dieselmotoren in Generallizenz der Krupp-Germaniawerft Kiel\index{o}{Kiel} und übernahm somit den Motorenbau der Krupp-Germaniawerft komplett\footnote{Vgl. MTZ52, S. 107 und MTZ53, S. 220}.
Sechs Jahre nach Conrad Geerlings\index{p}{Geerling, Conrad} Tot ging das Unternehmen 1953 in Konkurs\footnote{Ebenda.}. Die aus der WUMAG Hamburg\index{o}{Hamburg} hervorgegangene Tochterfirma in Krefeld\index{o}{Krefeld} hatte sich im Zuge weitreichender Expansion in \glqq WUMAG Niederrhein, Waggon- und Maschinenbau GmbH\grqq~umbenannt. Das Unternehmen produziert heute Hubarbeitsbühnen sowie Walzen und Maschinen, sieht sich jedoch nicht als Rechtsnachfolger der WUMAG Hamburg\index{o}{Hamburg}. Viel schlimmer noch ist die Leugnung jeglicher Schuld ihres Mitbegründers\index{p}{Geerling, Conrad}, des \glqq durch den Krieg vertriebenen ehemaligen Generaldirektors [...] Conrad Geerling\index{p}{Geerling, Conrad}\grqq\footnote{Ebenda.}. Nicht der Krieg hatte ihn vertrieben, sondern die Verantwortung für die Versklavung tausender Zwangsarbeiter zur Kriegsproduktion im Sinne der Nationalsozialisten.


%%%%%%%%%%%%%%%%%%%%%%%%%%%%%%%%%%%%%%%%%%%%%%%%
\subsection{Andere Arbeitskommandos}
\label{lagerarbeit}
Von den im Oktober 1944 in Görlitz inhaftierten 1500 KZ-Gefangenen arbeiteten im selben Zeitraum 1449 für die WUMAG. Der Einsatz der übrigen 51 erfolgte zum Teil innerhalb des Lagers in den Küchen für Häftlinge und Wachmannschaften sowie der Schusterei\footnote{Aussage von Iser (früher Isaak) Feigenblat. LArchB B Rep 058 Bd. 6.}, Schneiderei und Wäscherei\footnote{Aussage von Aron Greenfield. LArchB B Rep 058 Bd. 6.}. Darüber hinaus gab es ein Arbeitskommando außerhalb der Stadt.

%%%%%%%%%%%%%%%%%%%
\subsubsection{Das Arbeitskommando in der Landwirtschaft und das KZ-Außenlager Kunnerwitz}
\label{alkunnerwizt}
In der im Südwesten an Görlitz angrenzenden Gemeinde Kunnerwitz\index{o}{Kunnerwitz} gab es ein Arbeitskommando, welches im KZ-Groß-Rosen\index{o}{Groß-Rosen} auch als ein solches geführt und von manchen Historikern als Außenlager von Groß-Rosen\index{o}{Groß-Rosen} bezeichnet wird\footnote{Als Männerlager erwähnt in: Isabell Sprenger: Groß-Rosen, S. 232.}. Die Quellenlage ist diesbezüglich äußerst dürftig, so dass nicht einmal bekannt ist, wann und wie lange dieses Kommando existierte. Das Lager befand sich vermutlich an der Straße Richtung Weinhübel\index{o}{Weinhübel}. Die unmittelbare Nähe zu den Leichenfunden in der Sandgrube (siehe S. \pageref{kunnerwitz}) stützen diese Vermutung.\newline
Fritz Wolff\index{p}{Wolff, Fritz} aus Osterfelde wurde 1939 aus politischen Gründen in verschiedenen Polizeigefängnissen festgehalten und später ins KZ Groß-Rosen\index{o}{Groß-Rosen} deportiert.
\begin{leftbar}
Von dort wurde ich mit 24 anderen Häftlingen nach Kunnersdorf [Namensverwechslung mit Kunnerwitz ?]\index{o}{Kunnerwitz} bei Görlitz zu einem Bauern abkommandiert. Es ist möglich, dass ich von Oktober bis Dezember 1942 wieder vorübergehend im KZ Groß-Rosen\index{o}{Groß-Rosen} war.\footnote{Aussage von Fritz Wolff. LArchB B Rep 058 Bd. 4 / 2231.}
\end{leftbar}

Unbestätigt blieben bisweilen Annahmen, wonach das Arbeitskommando Kunnerwitz im Sommer 1944 dem KZ-Außenlager Görlitz angegliedert wurde und im Zuge dessen eine Neubelegung erfolgte. Aus diesen Gründen kann das Lager in Kunnerwitz dennoch als KZ-Außenlager bezeichnet werden.

%% Karte Kunnerwitz / Luftbild


%%%%%%%%%%%%%%%%%%%%%%%%%%%%%%%%%%%%%%%%%%%%%%%%
\section{Terror und Vernichtung}
\begin{leftbar}
Die Gefangenen stammten aus allen besetzten Gebieten Europas. Doch eint sie nicht einmal ein gemeinsamer Widerstandsgeist gegenüber ihren Peinigern. Der einzige Gedanke dieser Menschen galt dem Überleben, selbst wenn sie dafür ihre Mitgefangenen opfern oder sogar morden mussten.\grqq, schreibt Simon Schweitzer\index{p}{Schweitzer, Simon}, der einige Tage im Görlitzer Lager verbrachte\footnote{Simon Schweitzer: Simons langer Weg, S. 151.}.
\end{leftbar}


%%%%%%%%%%%%%%%%%%%%%%%%%%%%%%%%%%%%%%%%%%%%%%%%
\subsection{Mord und Folter an Häftlingen}

Die mangelhafte Verpflegung und Hygiene der Häftlinge sowie die Arbeit in den Fabriken führten zu einer allmählichen Schwächung und Verelendung; Misshandlung hingegen bedeutete sofortiges Verderben oder den sicheren Tod. \newline
Mord und Folter bildeten den Gipfel der Unmenschlichkeit und verdeutlichen erneut die maßlosen Ungerechtigkeiten im System des Konzentrationslagers. Es gab weder eine Gleichbehandlung von Häftlingsgruppen, noch eine Verhältnismäßigkeit im Strafmaß. Den Funktionshäftlingen blieb nicht nur Hunger und Schwerstarbeit erspart, sondern auch die alltäglichen Mißhandlungen. Besonders den Lagerältesten\index{p}{Czech, Hermann} und Kapos war daran gelegen, die von ihnen mitgetragene Ordnung im Lager zu erhalten und aufs Schärfste zu kontrollieren. Dabei korrumpierten sie andere Häftlinge, die ihrerseits bereit waren einen Leidensgenossen für ein Stückchen Brot zu denunzieren. Eine zerschlissene Decke, ein paar Kartoffelschalen oder Krankheit gaben Anlass genug zur Folter, ebenso Absprachen unter Gefangenen, die sich gegen die Lagerleitung richteten. Oftmals wurde versucht, durch Foltermaßnahmen ein Exempel zu statuieren und Nachahmer abzuschrecken. Die gewaltsamen Übergriffe durch SS und Funktionshäftlinge waren jedoch zumeist willkürlich und entsprachen nur selten einer Bestrafung im eigentlichen Sinne.
 \newline
Zu den üblichen Methoden gehörten Peitschen- und Stockhiebe ebenso wie das zu Tode Treten oder Prügeln. In fast allen Erinnerungsberichten werden Gewaltexzesse dieser Art geschildert, bei denen Menschen ums Leben kamen. Mehrfach hetzten SS oder Kapos Hunde auf die Gefangenen, so dass ihnen ganze Fleischteile ausgerissen wurden. In den Wintermonaten übergoss man Menschen mit kaltem Wasser und zwang sie in unbeheizten Räumen\footnote{\glqq Etwa im März [19]45 wurde ein Lagerinsasse von Görlitz im Waschraum angebunden und mit kaltem Wasser begossen. Er wurde so naß im Waschraum belassen und erfror bis zum nächsten Morgen. Über den Fall hörte ich von Lagerinsassen, am darauf folgenden Abend bekam ich -- wie üblich -- von Zunker den Befehl, über den Abgang Meldung zu machen.\grqq~Laut Emmrich Schiffer. LArchB B Rep 058 Bd. 5.} oder gar im Freien zu schlafen. Auch die alltäglichen Appelle vor und nach der Arbeit sind bei eisigen Temperaturen mit Folter gleichzusetzen. Gleiches gilt für die qualvollen \glqq Turnübungen\grqq, die die Blockältesten\footnote{Vgl. Adolf Eichner\index{p}{Eichner, Adolf} auf S.~\pageref{turnen}.} nach der Arbeit anordneten, und jene sinnlosen Beschäftigungen, wie das Schleppen von Steinen\footnote{Aussage von Israel Braun. LArchB B Rep 058 Bd. 6.}, mit denen man die Gefangenen vor allem am Wochenende schikanierte und ihnen jegliche Erholung verwehrte. Die SS setzte viel daran, die Freizeit der Häftlinge auf die Zeit der Essensausgabe und die kurze Nachtruhe zu begrenzen\footnote{Schreiben der Kommandantur des Konzentrationslager Groß-Rosen\index{o}{Groß-Rosen} an die Lagerführer vom 14.12.1944 betreffs Arbeitszeit der Häftlinge an den Weihnachtsfeiertagen: An den Tagen, an denen keine Appelle erfolgen, sind Appelle abzuhalten. PMGR, unbekannte Signatur.}. Jene, die nachts in der Fabrik arbeiten mussten, sollten tagsüber eigentlich ruhen, doch zog man sie oftmals zu Reinigungsarbeiten heran oder befahl ihnen, Kohlen oder Kartoffeln auszuladen\footnote{Aussage von Israel Braun. LArchB B Rep 058 Bd. 6.}. Auch von Gefangenen, die nicht\pagebreak\newpage imstande waren zu arbeiten, ließ man nicht ab. Als Simulanten und Saboteure beschimpft, waren die Menschen im Krankenrevier den Misshandlungen des Lagerältesten\index{p}{Czech, Hermann} Czech\index{p}{Czech, Hermann} hilflos ausgeliefert\footnote{Emmrich Schiffer berichtet von einem Studenten namens Freund, den Czech der Sabotage bezichtigte, weil er wegen Fussphlegmone arbeitsunfähig war. Freund schaffte man nach Groß-Rosen\index{o}{Groß-Rosen}. PMGR 4702/14/DP. Siehe auch: Urteilsbegründung Hermann Czech, LArchB: B Rep 058 Bd. 2.}.
\newline
In Berichten ehemaliger Görlitzer Häftlinge fällt auf, dass es so etwas wie eine kollektive Erinnerung an besonders tragische Fälle von Misshandlungen gibt.

\paragraph{Ausbinden}
Das Ausbinden oder auch Pfahlhängen war eine der grausamsten Foltermethoden in vielen Konzentrationslagern\footnote{Vgl. Isabell Sprenger: Groß-Rosen, S. 174.}. Als Pfahl diente ein Lichtmast, an dem der Häftling mit den Händen auf dem Rücken angebunden und, anfangs noch auf einem Schemel stehend, aufgehangen wurde. Beim Wegtreten des Schemels kugelten sich die Arme aus, so dass er den Boden nach einiger Zeit mit den Zehnspitzen erreichen konnte. Diese Tortur zögerte die SS oftmals über mehrere Stunden oder gar bis zum Tod des Häftlings hinaus.
\newline
In Görlitz ist dazu der Fall des Häftlings \mbox{Szaja} \mbox{Ensel}\index{p}{Ensel, Szaja}\footnote{Szaja Ensel, Häftlingsnummer: 24276. PMGR 4702/14/DP.} bekannt. Szaja Ensel\index{p}{Ensel, Szaja} hatte im Frühjahr 1945 aus Verzweiflung und Hunger das Mutterkaninchen aus dem Hasenstall des Lagerkommandanten\index{p}{Rechenberg, Erich} gestohlen und es gegen Brot beim Häftlingsschuster eingetauscht. Der Verlust des Kaninchens blieb nicht lange unbemerkt, und so kam es, dass der notleidende Dieb verraten und gestellt wurde\footnote{Aussage von Isaac Weintraub. LArchB B Rep 058 Bd. 3. In Übereinstimmung mit Emmrich Schiffer u.a. PMGR 4702/14/DP. Der Name des Opfers stützt sich auf Aufzeichnungen des Lagerschreibers, welche er bereits 1947 polnischen Behörden zukommen ließ. In späteren Aussagen anderer ehemaliger Häftlinge wird in diesem Zusammenhang auch der Name Brenner erwähnt, jedoch scheint es sich dabei um eine Verwechslung zu handeln. Moszek Brenner wurde wegen Diebstahls zweier Kartoffeln vor den Augen vieler Häftlinge durch Zunker erschossen.}.
Nach mehreren qualvollen Stunden und vergeblichen Schreien wurde er schließlich nachts vom Pfahl genommen und verstarb am darauffolgenden Morgen im Keller.
~\newline

\paragraph{Neid auf Schweine}
\label{schweine}
Innerhalb des Lagers hielt sich die SS einige Schweine, deren Futter aus Küchenabfällen der WUMAG bestand. Angesichts der unzureichenden und substanzlosen Kost im Lager nahmen viele Häftlinge die Gefahr auf sich, den Schweinen ihr Futter zu stehlen, um dem Hunger zu entgehen.

\begin{leftbar}
Wir waren neidisch auf die Schweine, die bessere Nahrung als wir erhielten\footnote{Vgl. Schlomo Graber: Schlajme, S. 72f.}.
\end{leftbar}

Als die SS zwei Gefangene auf frischer Tat ertappte, wollte man sie zur Abschreckung von Nachahmern vor den Augen der Gefangenen hinrichten. Der Lagerführer\index{p}{Zunker, Winfried} Zunker\index{p}{Zunker, Winfried} erschoss den ersten Häftling namens Schwimmer, den zweiten ließ er jedoch gehen, da seine Pistole klemmte und die Hinrichtung mehrmals fehlschlug\footnote{Iser Feigenblat, Schlomo Graber, Samuel Kessler, Josef Gleitmann, Aron Krakauer, George Pilc, u.v.m.}.

\paragraph{Kälte}
Keiner der gewöhnlichen Häftlinge hatte ausreichend warme Kleidung, die den rauen Witterungsbedingungen im Herbst und Winter entsprach. Der dünne Sträflingsanzug schien für den Sommer konzipiert, schützte jedoch weder vor Kälte noch Nässe. Anstatt warmer Schuhe hatten die meisten bestenfalls Holzlatschen. Das Tragen von Unterwäsche war strengstens untersagt. Viele versuchten sich dennoch ein paar Stoff- oder Papierfetzen um den Leib zu wickeln. Zahlreich sind die Fälle, in denen SS oder Funktionshäftlinge solche Zuwiderhandlungen bestraften; teils mit Todesfolgen\footnote{Samuel Reifer, ZIH 301/2311. Vgl. Schlomo Graber: Schlajme, S. 81.}.

\begin{leftbar}
Bei einem Appell wurde festgestellt, dass einer der Häftlinge [...] sich ein Stück von einer Decke abgeschnitten und um den Leib gebunden hatte. Die Häftlinge froren damals sehr. Kaum einer besaß noch Hemden. Dieser Häftling wurde zu einer besonderen Exekution verurteilt. Er musste sich auf einem im Essraum der Baracke 1 befindlichen Schemel legen und sollte fünfzig Schläge erhalten. Geschlagen wurde mit der Lederpeitsche des Lagerältesten\index{p}{Czech, Hermann}, die wahrscheinlich noch Draht enthielt. Den Befehl zu dieser Exekution gab der Lagerkommandant\index{p}{Rechenberg, Erich}, der während der Exekution dabei stand. Geschlagen haben Häftlinge, denen gesagt worden war, wenn sie es nicht tun würden, es ihnen genauso ergehen würde. Bei dieser Exekution wurde der ungarische Lagerarzt herbeigerufen, welcher den Tod feststellte.\footnote{Isaac Weintraub. LArchB B Rep 058 Bd. 4.}
\end{leftbar}



%%%%%%%%%%%%%
\subsection{\glqq Muselmänner\grqq}
Als \glqq Muselmann\grqq~bezeichnet man einen zerstörten, am Lagerleben zerbrochenen Mensch. Andauernder Hunger führte zu allgemeinener Körperschwäche und schrittweiser Vernichtung. Muskelabbau geht mit verminderten Vitalfunktionen einher. Die mentale und seelische Verfassung unterlagen gleichsam einem radikalen Rückgang. Der Häftling verlor das Gedächtnis und die Fähigkeit sich auf etwas anderes als Nahrung zu konzentrieren. Hungerphantasien überlagerten den quälenden Hunger\footnote{Schlomo Graber: \glqq Schlajme\grqq.}. Schläge nahm er widerstandslos hin. Im letzten Stadium spürte der Häftling schließlich auch keinen Hunger und keine Schmerzen mehr. Der \glqq Muselmann\grqq~ging im Elend zugrunde, weil er nicht mehr weiter konnte. Er war Leitfigur des Massensterbens, eines Todes durch Hunger, Seelenmord und Verlassenheit, ein Toter schon zu Lebzeiten.
\newline
\glqq Muselmänner\grqq~waren aufgrund ihrer geringen Arbeitsleistung für die Wirtschaftsunternehmen wertlos. Entgelte brauchten die Unternehmen für die kranken und schwachen Häftlinge aus diesem Grund nicht an die SS zu zahlen, trotzdem gingen ihnen die Kosten für Versorgungsleistungen weiterhin zu Lasten. Aus ökonomischer Sicht waren die Unternehmen deshalb bestrebt, diese Häftlinge gegen gesunde auszutauschen. Ende 1942 / Anfang 1943 führte die SS eine Regelung ein, wonach schwache Häftlinge generell ausgetauscht werden konnten\footnote{Vgl. Hermann Kaienburg: Konzentrationslager und deutsche Wirtschaft, S. 59.}. Der Austausch kam oftmals einem Todesurteil gleich und bedeutete zunächst immer eine Versetzung in ein anderes Lager. Die Entscheidung, welche Häftling ausgesondert wurden, oblag der Lagerleitung und wahrscheinlich auch der Gestapo.
\begin{leftbar}
Eines Abends, ungefähr im Dezember 1944, wurde in einem Block, ich lag damals im Block 6, eine Selektion durchgeführt. Etwa 10 Uniformierte -- darunter war bestimmt Zunker\index{p}{Zunker, Winfried} -- sehr wahrscheinlich auch der oben erwähnte hochrangige Offizier [Rechenberg\index{p}{Rechenberg, Erich}] -- bildeten die Kommission. Alle Blockinsassen mussten mit entblößtem Oberkörper an der Kommission vorbei. Eine nicht genauer erinnerliche Zahl wurde in Folge der Aussonderung auf LKW verladen und angeblich nach Groß-Rosen\index{o}{Groß-Rosen} versandt. Keiner von den Abtransportierten wurde nachher gesehen.\footnote{Mordechai (Mordko) Kozusarz. LArchB B Rep 058 Bd. 6.}
\end{leftbar}

Im Dezember 1944 wurden etwa 30 erschöpfte, arbeitsunfähig gewordene Lagerinsassen\label{widerstand_schiffer} \glqq selektiert\grqq, um sie nach Zittau (heute: Sieniawka, Polen)\index{o}{Zittau} zu schicken. Der Lagerschreiber\index{p}{Schiffer, Emmrich} Emmrich Schiffer\index{p}{Schiffer, Emmrich} erwähnte dazu:
\begin{leftbar}
Ich erinnere mich, dass ich eine Liste der Verschickten aufstellen musste. Offiziell hieß es, sie gehen zur Erholung. Diese Redewendung war jedoch jedem von uns verständlich. Keinen Namen der Verschickten vermag ich aufzureihen. Einen dieser Gruppe namens Friedländer -- [heute] angeblich wohnhaft in Netanya [Israel] -- konnte ich von der Liste streichen. Zunker\index{p}{Zunker, Winfried} hat dem Arbeitseinsatzschreiber die Liste der Namen gegeben, ich musste dann die Liste aufschreiben.\footnote{Emmrich Schiffer. LArchB B Rep 058 Bd. 5.}
\end{leftbar}

Im Dasein eines \glqq Muselmannes\grqq~wurde Schlomo Graber\index{p}{Graber, Schlomo} Opfer einer \glqq Selektion\grqq:
\begin{leftbar}
Gelegentlich besuchte eine Gestapo-Kommission die Fabrik, um die Leistungsfähigkeit zu überprüfen. Wer als \glqq Muselmann\grqq~eingestuft wurde, d.h. dem Tod durch Verhungern oder Entkräftung nahe war, wurde von der Werkbank weggeholt. Man stellte die Muselmänner auf eine Waage, und wer weniger als 30 Kilo wog, wurde ins Krematorium geschickt. Bei einer dieser Selektionen wurde ich zusammen mit 12 anderen ausgesondert. Wir alle brachten keine 30 Kilo auf die Waage.\footnote{Vgl. Schlomo Graber: Schlajme, S. 75.}
\end{leftbar}
Schlomo Graber\index{p}{Graber, Schlomo} schickte man ins Lager zurück, ihm gelang es später, in der Küche der Wachmannschaften unterzutauchen und dem Transport nach Groß-Rosen\index{o}{Groß-Rosen} zu entgehen.



%%%%%%%%%%%%%%%%%%%%%%%%%%%%%%%%%%%%%%%%%%%%%%%%
\subsection{Veranlasste Tötung durch die Gestapo}

In sehr vielen Aussagen von Überlebenden des KZ-Außenlagers Görlitz finden sich Berichte über Exekutionen durch die Gestapo. Den Gefangenen des Lagers sind diese Ereignisse besonders gut in Erinnerung geblieben, da diese sich auf die letzten Kriegsmonate beschränkten und eine ungewöhnliche Ausnahme im Lageralltag darstellten. Schon bei Ankunft der zur Hinrichtung bestimmten Personen wurde \glqq Blocksperre\grqq~verhängt. Die Häftlinge sollten nicht sehen, was draußen vor sich ging und wurden in ihrem Block angehalten, nicht aus dem Fenster zu schauen. Dies galt für den Lagerältesten\index{p}{Czech, Hermann} Czech\index{p}{Czech, Hermann} jedoch ebensowenig wie für einen Häftling namens Ludwig Egon Stellmach\index{p}{Stellmach, Ludwig Egon}, der sich wie folgt erinnert:
\begin{leftbar}
Sie wurden in einen Raum geführt, wo sie sich entkleiden mussten und wo ihre Personalien aufgenommen wurden. Ich musste die Personalien aufnehmen. Ich musste ihnen dann auf Polnisch oder Russisch sagen, sie müssten jetzt erst baden gehen. Auf dem Weg zum Nebenraum musste der Häftling an einem Gestapo-Beamten vorbei. Sobald er vorbei war, erschoss ihn dieser Gestapobeamte durch Genickschuss mit der Pistole. Ich habe ungefähr 10 solcher Fälle selbst beobachtet, als ich die Personalien dieser Häftlinge aufnehmen musste. Es war immer der selbe Gestapobeamte, der den Häftlingen die Genickschüsse gab. Er war in Zivil. Er war eine kleine Person.\footnote{Ludwig Stellmach, wahrscheinlich besser bekannt unter dem Namen Egon Wischalla, wurde in Oppeln (Opole, Polen) geboren. Am 2. Oktober 1958 sollte sich Stellmach vor dem Landgericht Hannover für seine Beteiligung an der Hinrichtung der sowjetischen Soldaten rechtfertigen. Aus bislang unbekannten Gründen wurde das Verfahren aufgrund einer Verfügung der Staatsanwaltschaft Aachen am 16. April 1959 eingestellt. Die Ermittlungsakten mit der Bezeichnung Az. UR 4/57 2 Js 835/58 befinden sich heute jedoch weder in einem Aachener noch in einem Hannoveraner Archiv. Die hier abgedruckte Zeugenaussage stellte die zentrale Verwaltungsstelle zur Aufklärung von NS-Verbrechen im Jahre 1970 der Staatsanwaltschaft Berlin zu. Aussage von Ludwig Stellmach LArchB B Rep 058 Bd. 2.}
\end{leftbar}
\newpage
In dieser Aussage wird bereits deutlich, dass es sich bei den Ermordeten um polnische oder sowjetische Staatsbürger handelte\footnote{Vereinzelt gibt es Aussagen von Häftlingen, die von Hinrichtungen deutscher und amerikanischer Offiziere berichten. Gerszon Sobotka. LArchB B Rep 058 Bd. 6. Anzeichen, wonach die, auf dem Jüdischen Friedhof 1948 exhumierte, Frau mit langem Haar und weißem Gewand eine Deutsche sei, konnten widerlegt werden. Sie wurde in Zeugenaussagen eindeutig als eine der wenigen im Lager verstorbenen Frauen zugeordnet (Vgl. S.~\pageref{weiss}).}. Bei Letzteren muss man zwischen Zivilisten (sogenannte Ostarbeiter) und den Angehörigen der Roten Armee unterscheiden.

Der Grund, weshalb man sie zur Hinrichtung ausgerechnet ins Lager Görlitz brachte, lässt sich nicht eindeutig bestimmen. Wahrscheinlich sah sich die Görlitzer Gestapo gezwungen, einen neuen Ort für Hinrichtungen zu finden, nachdem die Konzentrationslager Groß-Rosen\index{o}{Groß-Rosen} und Auschwitz\index{o}{Auschwitz} der Roten Armee in die Hände gefallen waren. Das Görlitzer KZ-Außenlager bot aus Sicht der Gestapo einen bequemen Ersatz. Entsprechende Absprachen mit der Lagerleitung muss es gegeben haben. Die Frage, ob sich der Lagerkommandant\index{p}{Rechenberg, Erich} hätte dagegen wehren können, kann nicht beantwortet werden\footnote{In einem ähnlichen Fall versuchte sich ein Angehöriger des Groß-Rosener Kommandanturstabes zu rechtfertigen, indem er sagte: \glqq Die Idee, sich zu weigern, sei ihm nicht gekommen, er sei sicher, dass man ihn hinter Gitter gebracht hätte.\grqq. Vgl. Isabell Sprenger: Groß-Rosen, S. 190.}. Unklar ist auch, ob sich das Lagerpersonal (Lagerführer\index{p}{Zunker, Winfried} und Lagerkommandant\index{p}{Rechenberg, Erich}) oder die Funktionshäftlinge (Lagerälteste\index{p}{Czech, Hermann}) an den Erschießungen beteiligten oder nur durch ihre Anwesenheit die Mordaktionen überwachten.
\newline

%%
\paragraph{Erschießung sowjetischer Offiziere}
Entgegen dem Völkerrecht ordnete Hitler\index{p}{Hitler, Adolf} mit dem sogenannten \glqq Kommissarbefehl\grqq~bereits am 6. Juni 1941 die \glqq Liquidierung\grqq\footnote{Das Wort Liquidierung gehört zu Victors Klemperers Lingua Tertii Imperii (LTI), der Sprache des Dritten Reiches.} aller kriegsgefangenen sowjetischen Offiziere an. Während der letzten Kriegsmonate lag Görlitz im Frontbereich und war von drei Seiten eingeschlossen. Das Hauptfeld der Kampfhandlungen lag zwar im weiteren Umkreis, doch blieb die Stadt von Fliegerbomben, Tieffliegerangriffen und Artillerietreffern nicht ganz verschont. Kreisleiter Malitz\index{p}{Malitz, Dr. Bruno} zeigte sich fest entschlossen, die Stadt mit Wehrmacht und Volkssturm zu halten\footnote{Das geht aus den Eintragungen des Tagebuchs von Josef Goebbels\index{p}{Goebbels, Joseph} hervor, der am 8. März 1945 letztmalig die Stadt Görlitz besuchte. Vgl. Ernst Kretzschmar: Görlitz unter dem Hakenkreuz, S. 63.}. Abschüsse sowjetischer Flugzeuge und die Festnahme einiger Offiziere sind demnach nicht unwahrscheinlich gewesen\footnote{Ebenda, S. 62.}. Die Beteiligung der Gestapo, die laut Maximilian Brandt die Offiziere ins Lager brachte, lässt auf ein Verhör der gefangenen Offiziere schließen. Maximilian Brandt\index{p}{Brandt, Maximilian} sagte dazu:
\begin{leftbar}
Durch die Fenster im Block sahen wir Gestapo-Leute uniformierte sowjetische Flieger ins Lager bringen. Die Russen wurden in den Waschraum geführt und dort erschossen. Am Vortag der Erschießung der Russen -- mit denen auch Zivile eingebracht und im Waschraum erschossen worden sind -- habe ich im Lager Gestapo-Leute mit dem Lagerkommandanten\index{p}{Rechenberg, Erich} -- den Obersturmführer [Rechenberg\index{p}{Rechenberg, Erich}] -- sprechen gesehen.\footnote{Maximilian Brandt. LArchB B Rep 058 Bd. 2}
\end{leftbar}

Der Zahnarzt Dr. Jakob Kinrus\index{p}{Kinrus, Dr. Jakob} erinnert sich ebenfalls der Hinrichtung sowjetischer Offiziere, die jedoch durch die SS ins Lager gebracht wurden:
\begin{leftbar}
Etwa im April 1945 nach 17 Uhr -- als alle von der Arbeit ins Lager zurückgekehrt waren -- wurde Blocksperre befohlen. Etliche 5\,m vor der Krankenstube hielt ein schwarzer PKW -- aus dem drei oder vier SS-Offiziere und drei oder vier, also insgesamt sieben Personen, sowjetische gefangene Offiziere, ausstiegen. Die Russen waren ohne Achselstücke gewesen, doch ihre Uniform bezeugte, dass es Offiziere gewesen sind. Mir auffallend wie höflich Zunker\index{p}{Zunker, Winfried} die Gruppe empfing, die Russen in das benachbarte Gebäude -- etwa 10--15\,m von der Krankenstube entfernt -- in den Waschraum im Keller lud. Ich konnte den Wortlaut gut hören. Einer der Russen verstand etwas deutsch. Nach kurzer Zeit ertönten vom Keller Schüsse -- es waren Einzelschüsse -- sie hörten sich nach Pistolenfeuer an. Nach Verlauf einiger Minuten erschien ein gedeckter Transportwagen, Zunker\index{p}{Zunker, Winfried} hat einige Häftlinge hinbefohlen und die Leichen wegschaffen lassen. Wohin die Leichen verbracht wurden, weiß ich nicht. Mein Kollege Dr. Schwarz\index{p}{Schwarz, Dr.}, der aus Ungarn stammte, musste dann Todesscheine -- mit Anführung der Todesursache -- Herzschlag -- unterschreiben.\footnote{Dr. Jakob Kinrus. LArchB B Rep 058 Bd. 2.}
\end{leftbar}

%%
\paragraph{Die Ermordung von Ostarbeitern}
In einem Schreiben Himmlers\index{p}{Himmler, Heinrich} an die Gestapo am 30. Juni 1943 wird die \glqq Verfolgung der Kriminalität unter den polnischen und sowjetrussischen Zivilarbeitern\grqq~aus dem deutschen Strafrecht ausgeschlossen und der Polizei unterstellt. Lokale Gestapo\-stellen konnten gegenüber Polen und Ostarbeitern ihre eigene \glqq Justiz\grqq~ausüben und brauchten nicht einmal eine Genehmigung von höherer Stelle einzuholen\footnote{BA: R58/1027, Runderlass des Chefs der Sipo: Betr. \glqq Vereinfachung im Schutzhaftverfahren\grqq~(4. Mai 1943 sowie deren Instruktionen: 6. August 1943).}. Bei schwerwiegenden Verstößen erfolgte die sofortige Exekution, die die SS und Gestapo verharmlosend als \glqq Sonderbehandlung\grqq, bezeichnete\footnote{Schreiben Himmlers vom 10. Februar 1944 an die höheren SS- und Polizeiführer. Vgl. Robert Gellately: Hingeschaut und weggesehen -- Hitler und sein Volk, S. 249.}. Über die Machenschaften der Görlitzer Gestapo gibt es keine\linebreak\newpage sicheren Belege, so dass sich nur anhand von Zeugenaussagen Vermutungen über einzelne Hinrichtungen anstellen lassen. Der ehemalige Häftling Juda Widawski\index{p}{Widawski, Juda} kann eine dieser Hinrichtungen bezeugen:
\begin{leftbar}
Als wir nach der Evakuierung zurück nach Görlitz gebracht worden sind, wurden, einige Wochen vor unserer Befreiung, zivile Ostarbeiter, ich erinnere mich an sechs Mann, ins Lager gebracht. Gestapoleute hatten diese Zivilarbeiter im Waschraum des Lagers erschossen. Ich musste bei dieser Gelegenheit mit einigen Leidensgefährten einen Wagen -- voll geladen mit etwa 30 Leichen -- auf den Friedhof ziehen. Die meisten Leichen gehörten verstorbenen jüdischen Häftlingen aus dem Lager an. Wir mussten am Friedhof die Leichen bestatten.\footnote{Juda Widawski. LArchB B Rep 058 Bd. 2.}
\end{leftbar}

%%%% WIDERSTAND

\subsection{Selbstbehauptung und Widerstand}

Selbstbehauptung ist die Wahrung des Menschseins und die Fähigkeit, sich seine moralischen und kulturellen Werte unter entwürdigenden Bedingungen zu bewahren. Auch wenn es bis hierhin unglaublich scheint, gibt es einige Anzeichen kulturellen Lebens, die eine gewisse Alltäglichkeit, jenseits von Drill und Terror, widerspiegeln. Nach Emir Kertesz\footnote{Ungarischer Literaturnobelpreisträger und ehemaliger KZ-Häftling in Auschwitz, Zeitz und Buchenwald. Vgl. Emir Kertesz: Roman eines Schicksalslosen.}, gibt es drei Strategien, die Zeit in einem Konzentrationslager zu überstehen: Autosuggestion, das heißt die bildhafte, verbale oder gedankliche Vorstellung eines Wunsches; der Glaube und die Wahrung des Menschseins.
Leider finden sich nur sehr wenige Aussagen darüber, wie die Menschen im KZ-Außenlager Görlitz ihre Inhaftierung mental durchstanden.\newline
Schlomo Graber\index{p}{Graber, Schlomo}:
\begin{leftbar}
Gottlieb stammte aus Munkacs\index{o}{Munkacs} in Karpatorussland  [Mukatschewe, Ukraine] und kam mit seinen Söhnen ins Lager. Er war ein gelehrter, gesetzestreuer Jude mit großem Talmudwissen und wusste als einzigster im Lager, wann die jüdischen Feiertage waren. Abends saß er mit seinen Söhnen vor dem Block und hielt ihnen mündlichen Talmudunterricht. Er war ein Mann, der stets Menschenantlitz bewahrte und auch im Konzentrationslager die ethischen Grundsätze einhielt.\footnote{Schlomo Graber: Schlajme, S. 79.}
\end{leftbar}

Rona Agnec\index{p}{Agnec, Rona}:
\begin{leftbar}
Wir haben von Büchern gesprochen und den Inhalt nacherzählt. Eine Frau hat auch Französischunterricht gehalten. Wir hatten aber nicht viel Zeit, vielleicht ein bis zwei Stunden, da waren wir froh, wenn wir uns hinlegen konnten oder in den Waschraum gehen konnten, denn auf einmal konnten wir dort nicht hineingehen. Unter uns war auch eine Opernsängerin. Sie hat uns ein paar mal ungarische Lieder gesungen. Das war für uns sehr traurig, weil dann jeder an seine Kinder, an seinen Mann oder an seine Mutter denken musste.\footnote{Interview mit Dr. med. Rona Agnec vom 10. Oktober 2006 in Budapest.}
\end{leftbar}

Henryk Vogler\index{p}{Vogler, Henryk}\footnote{Veröffentlicht unter der Herausgeberschaft von Adam W\l odek\index{p}{W\l odek, Adam}: Najcichszy Sztandar. Krakau, 1945. In dem 22 Gedichte umfasenden Band sind die Namen der Dichter nicht vermerkt, so dass keines der Gedichte Henryk Vogler zugeorndet werden konnte.} und Iren Weinberg\index{p}{Weinberg, Iren}\footnote{Iren (Steier) Weinberg: Gedichte/Lieder geschrieben im Ghetto Nagyvarad\index{o}{Nagyvarad}, Auschwitz\index{o}{Auschwitz} und Görlitz, 1944-45. Herausgegeben im ungarischen Journal \glqq Het Tukre\grqq. YV 3757268; O.76 file: 246, 17 Seiten;} veröffentlichten beide nach dem Krieg einige ihrer im Lager geschriebenen Gedichte. Im Zuge der letzten Kriegstage lockerte sich das Lagerregime gegenüber kulturellen Begeheren\footnote{Vgl. ausführliches Testimonie von Henryk Vogler auf S. ~\pageref{vogler}}:
\begin{leftbar}
[Die Lagerkapos Tannenbaum\index{p}{Tannenbaum, Jacob} und Schneebaum\index{p}{Schneebaum}] brachten mich auf die Idee einer Satire. In letzter Zeit versuchte ich mich zu bemühen eine Art Theaterstück zu schreiben und schließlich eine Lagerbühne zu montieren. Ich spielte vor Kameraden aktuelle Stücke und Monologe in deutscher Sprache, da unsere Gesellschaft multinational war und weil es die einzige Möglichkeit war sich zu verständigen. Ein berühmtes deutsches Weihnachtslied: \glqq O Tannenbaum! \dots\grqq~sang ich abwechselnd: \glqq Tennenbaum\grqq~statt \glqq Tannenbaum\grqq~--spöttisch lachte man über den Lagerkapo. Die Ähnlichkeit der Namen benutzte ich für ein Wortspiel: Wenn der Winter anbricht, Schnee die Bäume bedeckt und dann den Tannenbaum, so wird aus Tennenbaum ein Schneebaum.\footnote{Vgl. Henryk Vogler: Autoportret z Pamięci.}
\end{leftbar}

Schlomo Graber\index{p}{Graber, Schlomo} erwähnt in seinem Memoiren ein von deutschen Juden verfasstes Lied, welches zur Lagerhymne avancierte:
\begin{leftbar}
Wenn der Tag erwacht,\\
Die Sonne lacht,\\
Die Kolonnen ziehen,\\
In des Tages Mühen,\\
Im Morgengrauen,\\
Oh Zwangsarbeit,\\
Ich werde dich nie vergessen,\\
Weil du mein Schicksal bist.\footnote{Schlomo Graber: Schlajme, S. 78. Der Text des Liedes weist große Übereinstimmung mit dem \glqq Buchenwald-Marsch\grqq~auf.}
\end{leftbar}


%%
Über die Selbstbehauptung hinaus versteht sich unter Widerstand eine Tätigkeit mit weiter gesteckten Zielen. Im engeren Sinne wären dies lediglich Handlungen gegen die Wachmannschaften; etwas weiter gedeutet, schließt Widerstand alle Aktivitäten mit ein, die sich gegen die Intentionen der Lagerleitung richteten. Somit sind Selbstbehauptung und Solidarität grundlegende Leistungen des Widerstands gegen ein System, welches das Selbstbewusstsein und die Kameradschaft seiner Gegner zu zerstören versucht, um sie letztendlich soweit zu demoralisieren, dass es einer psychologischen Vernichtung gleich kommt.

In vielen Berichten der Überlebenden werden individuelle Initiativen beschrieben, wie anderen geholfen werden konnte; sei es durch ein Stückchen Brot, ein organisiertes Kleidungsstück oder durch tatkräftige Unterstützung beim Gehen und Arbeiten. Jedoch lassen sich in den vorliegenden Quellen keine Indizien für einen organisierten Widerstand im Lager Görlitz finden; ebenso wenig werden Formen des körperlichen Widerstands oder Fluchtversuche erwähnt\footnote{Fluchtversuche soll es jedoch laut Samuel Kessler während der Evakuierung gegeben haben. Samuel Kessler und Paul Levi: LArchB B Rep 058 Bd. 1}. Die Häftlinge waren ohnehin in einer bedrohlichen Situation, so dass sie jegliche Gefahren mieden. Die Konsequenzen für geringfügige Vergehen waren so verheerend, dass körperliche Gegenwehr gar nicht in Frage kommen konnte. \glqq Man wusste, dass die Front immer näher rückte\grqq\footnote{Samuel Kessler. LArchB B Rep 058 Bd. 1.} sagte Samuel Kessler, der sich in seiner relativ sicheren Funktion als Blockältester und später als Fuhrmann für die Gefangenen einsetzen konnte. Samuel Kessler\index{p}{Kessler, Samuel} errichte einen Spitzeldienst, der Mitgefangene vor Kapos und vor dem Lagerältesten\index{p}{Czech, Hermann} Hermann Czech\index{p}{Czech, Hermann} warnte\footnote{Samuel Kessler: LArchB B Rep 058 Bd. 1.}. Gleichsam durch seine Funktion begünstigt, gelang es dem Arzt Jakob Kinrus, nicht nur sein Möglichstes für die Kranken zu tun, sondern auch einen Plan der SS zu durchkreuzen und dadurch 13 Personen vor der Hinrichtung zu bewahren (siehe S.~\pageref{widerstand_kinrus}). Auch der Lagerschreiber\index{p}{Schiffer, Emmrich} Emmrich Schiffer\index{p}{Schiffer, Emmrich} schaffte es, zumindest eine einzelne Personen vor dem Abtransport in ein sogenanntes Erholungslager zu retten (siehe S.~\pageref{widerstand_schiffer}). Die bisher aufgezählten Widerstandshandlungen entstanden jeweils aus einer Situation heraus und sind ihrem Wesen nach Reaktionen auf unmittelbare Handlungen der SS und ihrer Komplizen.

Sabotageakte, wie der Folgende von Schlomo Graber\index{p}{Graber, Schlomo} beschriebene, waren nicht als eine Reaktion durch ein Ereignis motiviert und dienten auch nicht dem Selbstzweck der Ausübenden -- vielmehr bestärkten sie das Durchhaltevermögen aller Mitwissenden. Der Allmacht des Systems wurde versucht, entgegenzuwirken, sei es durch Sabotage, simulierte oder verlangsamte Arbeit oder durch das Pfuschen -- die Gewissheit, sich dem Druck nicht immer nur zu beugen, gab Hoffnung, dem Ganzen zu widerstehen.
\label{loetzinn}
\begin{leftbar}
Das Ziel war es, die Produktion soweit wie möglich zu behindern. Wenn sie hörten, dass der Lötzinn zu Ende ging, bohrten sie ein Loch in die Wand, schütteten den letzten Rest hinein und sangen den Meistern im Chor: \glqq Kein Zinn, Kein Zinn!\grqq. Eine andere Gruppe sorgte irgendwo für Kurzschluß. Einige dieser Arbeiter wurden erwischt und hingerichtet. Als wir einmal zur Arbeit kamen, sahen wir eine Gruppe tschechischer Zwangsarbeiter mit erhobenen Händen in der Eingangshalle vor dem Haupttor der Werkshalle [im Waggonbau] auf dem Boden sitzen, umringt von Gestapo-Leuten. Ich erfuhr, dass sie der Spionage verdächtigt wurden. Die Deutschen hatten ein Funkgerät entdeckt, das die Anzahl der produzierten Fahrzeuge weitergegeben hatte. Die gesamte Gruppe wurde hingerichtet.\footnote{Schlomo Graber: Schlajme, S. 74.}
\end{leftbar}
