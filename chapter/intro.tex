
\section*{Vorwort}
\addcontentsline{toc}{section}{Vorwort}%
Es war im Frühjahr 1994, als der inzwischen verstorbene Altbernsdorfer Ortschronist Siegfried Hittig uns Schüler anlässlich der Projekttage am Herrnhuter Gymnasium von einem Zug jüdischer Häftlinge erzählte, der sich zu Kriegsende, von Bernstadt kommend, am Fuße des Eichler-Berges in Rennersdorf verlor und schier spurlos verschwand. 
Niemand konnte mir eine Antwort darauf geben, was sich in diesem kleinen Dorf unweit der heutigen Grenzen zu Polen und der Tschechischen Republik im Winter 1945 abspielte. Weder meine Lehrer, noch meine um Rennersdorf ansässige Verwandtschaft hatten die leiseste Ahnung davon, was die Schülerin Katja Junge in einer Hausarbeit über das KZ Biesnitzer Grund fast 10 Jahre später zur Sprache brachte. Das Ausmaß und allein schon die Existenz nationalsozialistischer Konzentrationslager jenseits von Auschwitz und Buchenwald, sondern in der Oberlausitz, machten mich betroffen und neugierig zugleich. Rennersdorf war kein Einzelfall. Orte wie Bautzen, Großkoschen, Guben, Kamenz, Klein-Radisch, Niesky, Kunnerwitz, Niederoderwitz, Spohla, Weißwasser und Zittau reihen sich auf deutscher Seite ebenso in die Liste der ehemaliger Außenlager des KZ Groß-Rosen ein, wie das KZ-Außenlager in Görlitz, dessen Häftlinge sich 1945 auf einen Todesmarsch in das dafür gebildete Lager nach Rennersdorf begaben. Katja Junges Arbeit und die bis dato einzigen wissenschaftlichen Publikationen von Roland Otto bzw. Gräfe und Töpfer, in denen die KZ-Außenlager Görlitz und Rennersdorf ausführlicher Erwähnung fanden, ließen viele Fragen unbeantwortet und eröffneten lediglich die Auseinandersetzung mit einer Vielzahl von Aspekten. Dieses Versäumnis ist nicht den Autoren vorzuwerfen, sondern der Tatsache geschuldet, dass vorhandene Quellen bis 1990 gar nicht oder nur schwierig zugänglich waren. Beispielsweise hielt das Ministerium für Staatssicherheit sämtliche Prozessakten der Gerichtsverfahren gegen beteiligte NS-Verbrecher unter Verschluss, so dass selbige Dokumente noch heute nur mit erheblichen Auflagen eingesehen werden können. Darüber hinaus war es Historikern in der DDR nicht möglich, uneingeschränkt ausländische Archive zu kontaktieren und Informationen zu erbitten, geschweige denn, Zeitzeugen im westlichen Ausland zu besuchen. Roland Otto hatte noch dazu erhebliche Probleme, seine Diplomarbeit über die Verfolgung der Juden in Görlitz durchzusetzen, da er den so genannten Antifaschisten nicht die von ihnen allein beanspruchte Opferrolle zubilligte. Abgesehen von den damaligen politischen Blockaden war es vor dem Zeitalter des Internets schwierig, Personen, Orte und verschiedene Medien ausfindig zu machen oder in einem Zuge alle Jüdischen Gemeinden in Nordamerika auf der Suche nach Überlebenden zu kontaktieren. 

Anliegen dieses Buches ist es, die Gräuel des NS-Terrors jenseits von Buchwald und Auschwitz anhand der Erscheinung der KZ-Außenlager Görlitz und Rennersdorf exemplarisch zu vergegenwärtigen. Wer etwas über den Holocaust erfahren will, braucht nicht unbedingt in eine der bekannten KZ-Gedenkstätten fahren, wenn sich dutzende stille und unscheinbare Erinnerungs- und Gedenkorte in unmittelbarer Nachbarschaft befinden. Die in diesem Werk dokumentierten Geschehnisse leisten einen Beitrag für eine regionale und öffentliche Auseinandersetzung mit den zurückliegenden Diktaturen im Sinne einer zeitgemäßen Erinnerungsarbeit. Nicht zuletzt, um fremdenfeindlichen Tendenzen in der Oberlausitz entgegenzuwirken und den europäischen Dialog zu fördern.

\section*{Vorwort zur zweiten Auflage}
\addcontentsline{toc}{section}{Vorwort zur zweiten Auflage}%
Für die Herausgabe einer zweiten Auflage habe ich zwei Gründe. Erstens waren die 500 Exemplare der ersten Auflage im Neisse Verlag innerhalb von neun Monaten beim Verlag vergriffen und zweitens erlangte ich durch Leserbriefe, Rezensionen und nicht zuletzt durch die Auszeichnung mit dem Sächsischen Landespreis für Heimatforschung (Jugendpreis) vielfache Hinweise, das bestehende Werke zu verbessern und gewisse Abschnitte zu vertiefen.
Um die Verfügbarkeit künftig länger aufrecht zu erhalten, habe ich mich bewusst für eine digitale Publikation unter der freien Lizenz von \emph{Creative Commons} entschieden. Neue Erkenntnisse finden somit schneller zum Leser. 
Eine Übersicht über die Änderungen dieser Version sind im \emph{changelog}\footnote{\url{http://softbook.nise81.com/changelog.txt}.} verzeichnet. 

\section*{Einleitung}
\addcontentsline{toc}{section}{Einleitung}%
Ein Konzentrationslager war ursprünglich gleichbedeutend mit einem Gefangen- oder Internierungslager, in den Gruppen von Menschen festgehalten und konzentriert wurden. 
Zunächst betraf dies insbesondere Frauen und Kinder, die man angeblich aus Schutz vor Kampfhandlungen in solchen Lagern verwahrte. In Wahrheit diente die Gefangennahme als politisches Druckmittel, wie etwa im amerikanisch-spanischen Krieg oder dem Burenkrieg. Zumal diese Lager jeweils nur als vorübergehende Maßnahme gedacht waren, entsprachen sie in ihrem Wesen nicht dem systematischen Terror der nationalsozialistischen oder sowjetischen Lager\footnote{Vgl. Kotek / Rigoulot: Das Jahrhundert der Lager, S. 71.}. Gänzlich unabhängig von Kriegshandlungen wurde erstmals in Deutsch-Südwestafrika (heute Namibia) der Versuch unternommen Menschen in Konzentrationslagern zu internieren und durch Zwangsarbeit in so genannten Konzentrationslagern zu Grunde zu richten\footnote{Kotleg und Rigoulot schreiben, dass 1905 in den Konzentrationslagern mehr als die Hälfte (7862 Personen) der 10.632 Frauen und Kinder und 4137 Männer ums Leben kamen. Vgl. Kotek / Rigoulot: Das Jahrhundert der Lager, S.80.}. Der vorangegangene Versuch das Volk der Hereros vollkommen auszulöschen wurde durch Kaiser Wilhelm II\index{p}{Wilhelm II} aufgrund von Medienprotesten und wirtschaftlichen Belangen zwar abgebrochen, doch ebenso rassistisch und brutal durch Sklavenarbeit in Eisenbahnprojekten, sowie in privaten Wirtschaftsunternehmen fortgesetzt\footnote{Großen Privatunternehmen hatte man bereits eigene Lager zugesprochen, wohingegen kleine Firmen täglich die von ihnen benötigen Arbeitskräfte bezogen. Ebenda, S. 80ff. Vielfach unbekannt ist darüber hinaus, dass an den Gefangenen bereits medizinische Experimente durchgeführt wurden; u.a. von den späteren Lehrmeistern Josef Mengeles\index{p}{Mengele, Josef}, Eugen Fischer\index{p}{Fischer, Eugen} und Theodor Millisson\index{p}{Millisson, Theodor}.}.

%Das jene Lager penibel bürokratisch organisiert waren mag weniger überraschen, als die Tatsache, dass zwei Lehrmeister Josef Mengeles bereits medizinische Experimente an den Gefangenen durchführten.

Noch vor den Nationalsozialisten etablierte die Sowjetunion ein riesiges, dauerhaftes Lagersystem als administrative Maßnahme zur Isolation, Bestrafung und Versklavung politischer oder krimineller Delinquenten bzw. unschuldiger Menschen. Bereits im Jahre 1923 entwickelte die Lagerleitung auf den Solowezki-Inseln\index{o}{Solowezki-Inseln} am Weißrussischen Meer, ähnlich wie später in deutschen Konzentrationslagern, ein System der teilweisen Selbstverwaltung durch politische Häftlinge. Zu dem wurde ein Normensystem zur Maximierung der Arbeitsleistung geschaffen, dass sprichwörtlich die Vernichtung von Leben innerhalb der ersten drei Monate zum Ziel hatte\footnote{Naftaly Aronovich Frenkel\index{p}{Frenkel, Naftaly Aronovich}, ein zum Lagerkommandanten aufgestiegener Häftling prägte den Satz: \glqq Aus dem Häftling müssen wir alles in den ersten drei Monaten herausholen - danach brauchen wir ihn nicht mehr.\grqq~ Kotek / Rigoulot: Das Jahrhundert der Lager, S. 181.}. Im Hinblick auf  Organisation, wirtschaftliche Erwägungen und Massenmord gibt es weiterreichende Parallelen, aber natürlich auch Unterschiede zum System der nationalsozialistischen Konzentrationslager, auf die hier jedoch nicht weiter eingegangen werden soll.

Als Mittel staatspolizeilicher Gewalt fungierten die Konzentrationslager während der gesamten 12-jährigen Herrschaft der Nationalsozialisten. Formal legitimiert durch die \glqq Verordnung des Reichspräsidenten zum Schutz von Volk und Staate\grqq~wurden \glqq Beschränkungen der persönlichen Freiheit [...] außerhalb der sonst hierfür bestimmten gesetzlichen Grenzen [...]\grqq\footnote{RGB 1933, Teil 1, S83.} zur gängigen Praxis, um Personen für unbestimmte Zeit und ohne richterliche Verfügung in sogenannte Schutzhaft zu nehmen. Schutzhaft hatte den Charakter einer staatspolizeilichen Repressionsmaßnahme und diente durchaus nicht, wie es der Name vermuten läßt, dem Schutz der betreffenden Person.
\\
Die Entwicklung der Konzentrationslager einschließlich der Schutzhaft wurde durch die jeweilige außenpolitische Situation geprägt und lässt sich in vier Phasen einteilen. Während der Jahre 1933--1935 spricht man von der ersten Phase, in der die Schutzhaft als Instrument der Herrschaftssicherung diente. Die Isolation führender Oppositioneller brachte den Widerstand gegen die Nationalsozialisten zum Erliegen und sorgte für eine \glqq soziale Formierung\grqq~der Gesellschaft. Es entstanden eine Reihe von kleinen Konzentrationslagern, wie zum Beispiel in Niesky, Weißwasser, Leschwitz (heute: Weinhübel bei Görlitz) und Hainewalde, von denen nach einer straffen Neuorganisation im Jahr 1935 lediglich die Lager in Bad Sulza, Berlin-Tempelhof, Dachau, Hainewalde, Hamburg-Fuhlsbüttel, Kislau, Lichtenburg, Mohringen und Sachsenburg bestehen blieben.
In der zweiten Phase zwischen 1936 und 1938 waren jene Menschen von der Schutzhaft betroffen, die nicht dem nationalsozialistischen Bild entsprachen. In der damaligen Redeweise galt dies für Bibelforscher (Zeugen Jehovas), Assoziale, Berufs- und Gewohnheitsverbrecher, Arbeitsscheue, Homosexuelle und seit 1938, besonders im Gefolge der Reichsprogromnacht, auch für Juden und andere \glqq Nichtarier\grqq.
Kurz vor Kriegsbeginn setzte die dritte Phase ein. Der erwartete Anstieg der Gefangenenzahlen durch die Deportation von Juden in den besetzten Gebieten führte zu einer abermaligen Neuorganisation der Konzentrationslager. Zu den insgesamt sechs großen Lagern innerhalb des Reichsgebiets, die jeweils für mehrere tausend Gefangene ausgelegt waren, kamen nach Beginn des Krieges die Lager Auschwitz (1940), Hinzert (1940), Groß~Rosen (1941), Natzweiler (1941), Lublin (1941), Wewelsburg (1941), Arbeitsdorf (1942) und Stutthof (1942) hinzu. 
\\
Als im Winter 1941/42 die \glqq Blitzkrieg-Strategie\grqq~gegen die Sowjetunion scheiterte und eine Reihe von Niederlagen seitens der Nationalsozialisten folgte, musste man sich auf einen länger anhaltenden Abnutzungskrieg einstellen. Die kriegswirtschaftliche Umstellung auf den \glqq totalen Krieg\grqq~und die damit verbundene Mobilmachung von Arbeitskräften bestimmte maßgeblich die vierte Phase der Konzentrationslager, deren Auswirkungen hier am Beispiel der Lager in Görlitz und Rennersdorf dokumentiert werden. 
Beide Lager vermitteln exemplarisch Einblick in das Wesen der nationalsozialistischen Gewaltherrschaft, die nicht als zentrales Ungeheuer, sondern als allgegenwärtige Erscheinung, fernab von Auschwitz und Buchenwald auftrat. Die Ansiedlung von Konzentrationslagern in der unmittelbaren Nähe von Rüstungsbetrieben konfrontierte die alltägliche Lebenswelt der Deutschen mit den Ausprägungen von Terror und Gewalt im Leben der KZ-Häftlinge. Die Wechselwirkungen und Beziehungen zwischen KZ-Kosmos und zivilem Alltag, Einfluss und Zusammenspiel von Wirtschaft und Staat, sowie Wissen und Bewusstsein der Bevölkerung sind am Gegenstand der KZ-Außenlager besser zu erklären als an anderen Erscheinungsformen der nationalsozialistischen Herrschaft. Anliegen dieser Dokumentation ist die rationale Darstellung der Geschehnisse und Zusammenhänge in den KZ-Außenlagern Görlitz und Rennersdorf einschließlich ihrer Auswirkungen in der Nachkriegszeit. Anstatt einzelne Gruppen als Schuldige auszumachen, wird versucht das Wirken des Systems als Ganzes beleuchtet.

