\chapter{Zusammenfassung}

\section{Zusammenfassung}
Die Geschichte jüdischer Zwangsarbeit während des Zweiten Weltkrieges begann in Görlitz mit der Errichtung eines Zentralen Arbeitslagers der Organisation Schmelt im Frühjahr 1943. Das ZAL war eines von mindestens 23 \glqq Menschenlagern\grqq~ der \emph{Waggon und Maschinenbau AG} (WUMAG) - Görlitz' größtem Industriebetrieb und Arbeitgeber. Das ZAL wurde ursprünglich als Kriegsgefangenenlager auf dem städtischen Grundstück, unweit von Wohnsiedlungen in Form eines Barackenlagers gebaut. Mit der Auflösung der Organisation Schmelt im Frühjahr 1944 gliederte man den Görlitzer Standort dem KZ Groß-Rosen an. Die Schmelt-Häftlinge deportierte die SS in andere Arbeitslager (KZs). Das Barackenlager in Görlitz ab dem 8. August unter Leitung der SS durch Gefangene aus Groß-Rosen an die Sicherheitsstandards von KZ-Arbeitslagern baulich angepasst wurde. Bis Herbst 1944 deportierte die SS 1.500 Menschen nach Görlitz. Es waren größtenteils Juden aus Polen und Ungarn. Darunter befanden sich auch 300 Frauen, die man in einem abgetrennten Bereich des Lagers unterbrachte. 

Die Gefangenen bildeten eine heterogene Gruppe, die sich hinsichtlich ihres Alters, ihrer Nationalität und ihrer einstigen gesellschaftlichen Stellung wesentlich voneinander unterschieden, jedoch dem System und dem Terror des Konzentrationslagers, wenn überhaupt, nur schwerlich Stand halten konnten. Eine Sonderstellung galt den Funktionshäftlingen, welche die Organisation des Lagers relative autonom bestimmten und sich insbesondere durch weitgehende Privilegien von den übrigen Gefangenen unterschieden. Die Lagerleitung besetzte die Stellen der Häftlingsselbstverwaltung und duldete deren Übergriffe auf andere Gefangene sowie deren Korruptionssystem. Die meisten dokumentierten Folterungen, Misshandlungen und Vergewaltigungen werden Funktionshäftlingen zugeschrieben. Die Bewegungsfreiheit des Lagerältesten\linebreak\newpage sowie bestimmter anderer Funktionshäftlinge sorgte für einen Tauschhandel über die Grenzen des Lagers und der WUMAG hinaus. Ein Teil der unterschlagenen Lebensmittel gelangte zu den Frauen im Lager, sofern sie sich zu Gegenleistungen bereit erklärten. 

Neben der chaotischen Lagerorganisation bedingte die mangelnde medizinische Versorgung, widrige hygienische Bedingungen, schlechte Kleidung und unzureichende Ernährung eine stete Zunahme an Krankheiten und Todesfällen - vor allem unter den Männern. Auch die harte und unmenschliche Arbeit in den Werken der WUMAG trug wesentlich zur Sterblichkeit der Häftlinge bei. Abgesehen von Bonus-Rationen für überplanmäßige Arbeit sind seitens der WUMAG keinerlei Versuche bekannt, die Lebensbedingungen der Häftlinge zu verbessern. Gleiches gilt für die Lagerleitung - namentlich für den Lagerkommandanten Erich Rechenberg, dem der Kommandant des Bunzlauer Außenlagers (Hauptsturmführer Michael) im Februar 1945 den katastrophalen Umgang mit kriegswichtigen Arbeitskräften im Lager Görlitz vorwarf. Angesichts der kriegswichtigen Produktionsaufträge, welche die WUMAG zu erfüllen hatte, erscheint die zügellose Führungsrolle der SS geradezu als paradoxe Sabotage an der deutschen Rüstung. 

Eine weitere Sonderheit des KZ-Außenlagers Görlitz betrifft seine räumliche und gesellschaftliche Integration im Vergleich zu anderen Groß-Rosener Arbeitslagern. Zum einen befanden sich die Produktionsstandorte der WUMAG relativ zentral im Stadtgebiet, zum anderen grenzte das Lager unmittelbar an dicht besiedelte Wohngebiete, von denen aus die Anwohner praktisch alle Vorgänge im Lager beobachten konnten. Damit einher gingen persönliche Kontakte zwischen Einheimischen und Lagerinsassen innerhalb der WUMAG, aber auch in örtlichen Handwerksbetrieben, in den umliegenden Kleingärten und bei Besorgungen in der Stadt.     

Jeweils verheerend auf die Situation der Häftlinge im KZ-Außenlager Görlitz wirkten sich die von Alfred Konieczny beschriebenen drei Etappen der Evakuierung schlesischer KZ-Außenlager aus. In der ersten Etappe funktionierte das KZ-Außenlager als Durchgangs- und Auffanglager anderer evakuierter Lager. Während der zweiten Etappe wurde das Lager selbst über Kunnerwitz in ein provisorisches Ausweichlager nach Rennersdorf verlegt.
In der dritten Etappe kommt es zu der einmaligen Begebenheit, dass die Gefangenen im Zuge der letzten Fronterfolge der Deutschen zur Rückkehr nach Görlitz gezwungen werden, um beim Festungsausbau der Stadt eingesetzt zu werden. Die Folgen dieser letzten Verblendung der Görlitzer Kreisleitung mündeten in eine humanitäre Katastrophe, die 173 Gefangene mit ihrem Leben bezahlten. Mindestens jeder fünfte Lagerinsasse erlebte den Tag der Befreiung am 8. Mai 1945 nicht mehr.

Sowohl in der sowjetischen, als auch in den westlichen Besatzungszonen bzw. den daraus hervorgegangenen Staaten wurden Anstrengungen unternommen, die im Görlitzer KZ-Außenlager begangenen Verbrechen gegen die Menschlichkeit zu ahnden. Eine umfassende juristische Einschätzung des komplexen Verantwortungsgeflechts von SS, Funktionshäftlingen, WUMAG und \mbox{NSDAP} Kreisleitung sowie den nachgeordneten Ämtern und Institutionen kam jedoch nicht zustande. Dies lag zum einen an der Zurückhaltung relevanter Informationen durch das Ministerium für Staatssicherheit und zum anderen am Unwillen den Gesamtkomplex überhaupt betrachten zu wollen (Staatsanwaltschaft Berlin im Fall Erich Rechenberg). Insbesondere der Prozess gegen den Görlitzer Kreisleiter Malitz und Oberbürgermeister Meinshausen ist ein Beispiel für die Entlastung vieler Täter, Mittäter und Mitwisser zulasten einzelner Personen, derentwegen die öffentlichen Debatte um die Schuldfrage beendet werden konnte. Erst nach dem Fall des \glqq Eisernen Vorhangs\grqq~ und der Öffnung der Archive war es möglich, die Geschehnisse der KZ-Außenlager Görlitz und Rennersdorf aus verschiedenen Blickwinkeln kritisch zu betrachten, zu dokumentieren und insbesondere die beschriebenen Erinnerungsorte ins Bewusstsein zu rufen. 

% Desiderate: Aufarbeitung Malitz-Meinshausen-Prozess



%% ENGLISCH
\begin{comment}
\section{Summary} gnaz schlecht:
In Görlitz the history of jewish forced labor during World War II began with the erection of the \emph{Zentralen Arbeitslager} (ZAL) of the Organsation Schmelt in spring 1943. The ZAL was only one of XXX so called /human ware houses/ of Görlitz' biggest industrial manufacturer, the Waggon und Maschninenbau AG (WUMAG). Initially the ZAL has been a barack camp /war prisoners/ on the property of the city of Görlitz, closely to housing settlements. In Spring 1944 when the organisation Schmelt has been liquidated, the branch in Görlitz became a part of concentration camp Groß Rosen. Inmates of the former Schmelt camp were dortorted to other forced labor camps respectivly concentration camps. In the same time the Görlitz camp got under control of the SS. On 8th of August 1944 the first prisoners from KZ Groß Rosen were oblidged to enhance the security standards to the level of concentartions camps. Until autum 1944 the SS deported 1500 people to Görlitz. Most of them were jews from Poland and Hungaria. 300 of them were women and therefor accomodated in a separate part of the camp.

The inprisoners were a heterogenous group. Their age differed widly, same as their nationality and their former social status. All of them could hardly withstand the system of terror of the concentration camp. So called Funktionshäftlinge were prisoners oblidged with special duties in order to organize the camp almost independently. Therefor they were previliged in contrast to their fellow prisoners. The camp leadership put inmates in these positions of the self administration and suffer attacks as well as corruption around food and other coveted goods. Most of the documented tortures, ill-treatments and raptures could be attributed to prisoners in duty. The relativly freedo of action of the Lagerältesten (prisoner in chief) and other prisioners with special obligations encouraged bartering accross the camp borders as well as the WUMAG. The femal inmates benefit from that barter goods, if they had agreed to rewards. 

Beside the chaotic organisation of the camp ...

Neben der chaotischen Lagerorganisation bedingte die mangelnde medizinische Versorgung, widrige hygienische Bedingungen, schlechte Kleidung und unzureichende Ernährung eine stete Zunahme an Krankheiten und Todesfällen - vor allem unter den Männern. Auch die harte und unmenschliche Arbeit in den Werken der WUMAG trug wesentlich zur Sterblichkeit der Häftlinge bei. Abgesehen von Bonus-Rationen für überplanmäßige Arbeit sind seitens der WUMAG keinerlei Versuche bekannt, die Lebensbedingungen der Häftlinge zu verbessern. Gleiches gilt für die Lagerleitung - namentlich für den Lagerkommandanten Erich Rechenberg, dem der Kommandant des Bunzlauer Außenlagers (Hauptsturmführer Michael) im Februar 1945 den katastrophalen Umgang mit kriegswichtigen Arbeitskräften im Lager Görlitz vorwarf. Angesichts der kriegswichtigen Produktionsaufträge, welche die WUMAG zu erfüllen hatte, erscheint die zügellose Führungsrolle der SS geradezu als paradoxe Sabotage an der deutschen Rüstung. 
\end{comment}




%% POLNISCH
\section{Streszczenie}
Historia pracy przymusowej w Görlitz (Zgorzelec) podczas drugiej wojny światowej ma swój początek wraz z uruchomieniem centralnego obozu pracy organizacji „Schmelt“ wiosną 1943 roku. Obóz ten był jednym z 25 „magazynów ludzi“ należących do  fabryki wagonów i maszyn WUMAG, największego zakładu przemysłowego oraz pracodawcy w Görlitz. Centarlny obóz pracy był budowany jako obóz jeńców wojennych na terenie należącym do miasta, a obozowe baraki stanęły niedaleko osiedli mieszkaniowych Görlitz. Po rozwiązaniu organizacji „Schmelt“ wiosną 1944 roku obóz stał się częścią obozu koncentracyjnego Gross-Rosen. Więźniowie Schmelta zostali przeniesieni do innych obozów pracy przymusowej, a począwszy od 8 sierpnia nastąpiła przebudowa obozu, pod kierownictwem SS,  do standardów obowiązujących w innych obozach koncentracyjnych. Jako siłę roboczą wykorzystywano więźniów obozu Gross-Rosen. Do jesieni 1944 roku SS deportowało 1500 osób do obozu w Görlitz. W większości byli to węgierscy oraz polscy Żydzi. W oddzielnej części obozu umieszczono około 300 kobiet.
Więźniowie tworzyli zróżnicowaną wiekowo, narodowościowo oraz społecznie grupę ludzi stawiającą każdego dnia czoła terrorowi obozu koncentracyjnego. Wyjątkową pozycję zajmowali więźniowie funkcyjni. Kształtowali oni w dużej mierze organizację życia obozowego oraz cieszyli się licznymi przywilejami. Stanowiska funkcyjne obsadzali kierujący obozem esesmani, którzy tolerowali ich akty przemocy wobec więźniów oraz  system korupcyjny. Właśnie więźniom funkcyjnym przypisuje się większość z udokumentowanych przypadków tortur, znęcania się oraz gwałtów. „Starszy obozu“ oraz paru innych wysoko postawionych więźniów dzięki swoim przywilejom mogli rozwijać handel wymienny sięgający daleko poza granice samego obozu jak i fabryki wagonów i maszyn WUMAG. Szczególnie kobiety mogły liczyć na dodatkowe racje żywnościowe pod warunkiem, że umiały się za nie w odpowiedni sposób odwdzięczyć.
Katastrofalne warunki obozowe, a w szczególności brak opieki medycznej oraz niekorzystne warunki higieniczne, braki w odzieży i niewystarczająca ilość żywności, prowadziły do licznych chorób i zgonów przede wszystkim wśród mężczyzn. Również ciężka praca w nieludzkich warunkach zakładu WUMAG prowadziła do śmierci wielu więźniów. Za wyjątkiem dodatkowych racji żywieniowych za ponadplanową pracę, nieodnotowano żadnych starań ze strony WUMAGu w celu poprawienia warunków życiowych więźniów. To samo tyczy się kierownictwa obozu, a mianowicie komendanta Ericha Rechenberga. W lutym 1945 roku komendant obozu w Bunzelau (Bolesławiec) Michael zarzucił mu katastrofalne traktowanie podległej mu siły roboczej, tak ważnej dla pogrążonego w wojnie kraju. Biorąc pod uwagę znaczenie produkcji WUMAGu dla niemieckiego przemysłu zbrojeniowego, można uznać styl prowadzenia obozu przez SS paradoksalnie jako swego rodzaju sabotaż przemysłowy.
Obóz w Görlitz wyróżniał się wśród innych obozów pracy podlegających obozowi Gross-Rosen swoja lokalizacją. Zakłady WUMAGu leżały na terenie miasta, a sam obóz graniczył z zamieszkałymi osiedlami, których mieszkańcy mogli w sposób niemal nieograniczony obserwować życie obozowe. To umożliwiało nawiązywanie kontaktów z mieszkańcami Görlitz podczas pracy w fabryce, u lokalnych rzemieśliników, w pobliskich ogródkach działkowych lub podczas zakupów w mieście.

Tragiczna w skutkach dla więźniów obozu w Goerlitz okazała sie opisana przez Alfreda Koniecznego trójetapowa ewakuacja śląskich obozów pracy. W pierwszym etapie obóz w Görlitz służył jako oboz przejściowy dla więźniów z innych ewakuowanych obozów. W drugim etapie obóz został ewakuowany przez miejscowość Kunnerwitz do prowizorycznego obozu w miejscowości Rennersdorf. Ze względu na sukcesy frontowe niemieckiej armii więźniowie zostali w trzecim etapie zmuszeni do powrotu do Görlitz w celu rozbudowy umocnień miasta. Ta ostatnia decyzja kierownictwa okręgu Görlitz skończyła sie katastrofą humanitarną i kosztowała życie 173 więźniów. Przynajmniej co piaty więzień nie dożył wyzwolenia obozu 8 maja 1945 roku.

Zarówno w radzieckiej jak i w zachodnich strefach okupacyjnych zostały podjęte starania o ukaranie zbrodni na ludzkości popełnionych w Görlitz. Nigdy nie doszło jednak do gruntownej prawniczej analizy skomplikowanej siatki zbrodni esesmanów, więźniów funkcyjnych, zarządu WUMAGu, okręgowego NSDAP jak i również innych podległych im urzędów i instytucji. Spowodowane to było z jednej strony blokadą informacji przez MfS (Ministerstwo Bezpieczeństwa NRD, powszechnie zwanego Stasi), z drugiej zaś niechęcią zajęcia się tak skomplikowana sprawą (prokuratura miasta Berlin w sprawie Ericha Rechenberga). W szczególności proces przeciwko kierownikowi okręgu Görlitz oraz burmistrzowi miasta Görlitz Meinshausenowi jest przykładem uniewinnienia wielu sprawców, współwinnych, a także osób wtajemniczonych, co pozwoliło szybko zakończyć publiczną debatę o ich winie. Dopiero po upadku żelaznej kurtyny i otwarciu archiwów nadażyła się szansa do krytycznego rozpatrzenia wydarzeń w obozie Görliz. Pozwoliło to na uświadomienie społeczeństwa oraz udokumentowanie tego miejsca pamięci. 

