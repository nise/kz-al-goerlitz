\chapter{Verzeichnisse}
\phantomsection
\section*{Literaturverzeichnis}
\addcontentsline{toc}{section}{Literatur}%
%\begin{thebibliography}{Literaturverzeichnis} %\bibliographystyle{alpha} % highest number: 48

%\begin{footnotesize}
\begin{description}
\item[]{Abteilung Kultur beim Rat der Stadt Görlitz (Hsg.): \glqq Mit Schweiß und Blut gedüngter Boden im Biesnitzer Grund\grqq, in: \glqq Görlitzer Kulturspiegel\grqq. Ausgabe September 1955. Görlitz 1955.}
%Vgl. auch: Sächsische Zeitung vom 8. Juli 1955.

\item[]{Kitia Altman: \glqq Die Hoffnung auf das Überleben -- Alfred Roßners Hilfe für die Juden in Będzin.\grqq In Wolfgang Benz (Hrsg): Dachauer Hefte Jahrgang 20, S. 194ff.}

\item[]{Wolfgang Benz: \glqq Selbstbehauptung und Gegenwehr von Verfolgten\grqq. In: \glqq Deutscher Widerstand 1933--1945\grqq. Bundeszentrale für politische Bildung (Hrsg.). Bonn 1994.}

\item[]{Wolfgang Benz: \glqq Die Allgegenwart des Konzentrationslagers. Außenlager im nationalsozialistischen KZ-System\grqq. In: Wolfgang Benz / Barbara Distel (Hrsg.): \glqq Das Ende der Konzentrationslager\grqq. Dachauer Hefte 20. Jahrgang. Heft 20. Dachau 2004. }

\item[]{David van Biema: \glqq Poisoned Lives -- Two men met in a Nazi slave labor camp 44 years ago. One was a victim, the other a torturer, and both were Jews. Since then, one has prayed for justice, the other to forget. Both have failed, miserably.\grqq Washington Post, 24. April 1988.}

\item[]{Heinz Boberach / Rolf Thomas / Hermann Weiss: \glqq Ämter, Abkürzungen, Aktionen des NS-Staates\grqq. K. G. Saur (Hrg.). Instituts für Zeitgeschichte. München 1997.}

\item[]{Eugenius Brzezicki / Adolf Gawalewicz / Tadeus Ho\l uj, Antoni Kępiński \ Stanis\l aw K\l odziński, W\l adys\l aw Wolter: \glqq Die Funktionshäftlinge nationalsozialistischer Konzentrationslager\grqq. 
Hamburger Institut für Sozialforschung (Hrsg.): \glqq Die Auschwitz-Hefte\grqq. Rogner \& Bernhard. Hamburg 1995.}

\item[]{Bundeszentrale für politische Bildung (Hrsg.): \glqq Gedenkstätten für die Opfer des Nationalsozialismus\grqq. Bonn 1999. }

\item[]{Danuta Czech: \glqq Kalendarium der Ereignisse im Konzentrationslager Auschwitz-Birkenau 1939--1945\grqq. Rowohlt. Reinbeck 1989.}

\item[]{Adolf Eichmann: \glqq Götzen\grqq. Unveröffentlichtes Manuskript seiner Aufzeichnung im Gefängnis in Haifa, was Eichmann mit diesem Namen versah. O.J.} 

\item[] {Alexander Fischer (Hg.): Teheran -- Jalta -- Potsdam. Die sowjetischen Protokolle von den Kriegskonferenzen der Großen Drei, S. 184f. Köln 1985.}

\item[]{Franck'sche Verlagshandlung (Hsg.): \glqq Motortechnische Zeitschrift\grqq. Ausgabe 04/1952. Stuttgart 1952.}

\item[]{Franck'sche Verlagshandlung (Hsg.): \glqq Motortechnische Zeitschrift\grqq. Ausgabe 07/1953. Stuttgart 1953.}

\item[]{Robert Gellately: \glqq Hingeschaut und weggesehen -- Hitler und sein Volk\grqq. Deutsche Verlagsanstalt München. München 2002.}

\item[]{Georg K. Glaser: \glqq Geheimnis und Gewalt -- Der autobiographische Bericht eines Einzelkämpfers\grqq. Rowohlt Taschenbuch. Hamburg 1983.}

\item[]{Schlomo Graber: \glqq Schljame -- Von Ungarn durch Auschwitz-Birkenau, Fünfteichen und Görlitz nach Israel. Jüdische Familiengeschichte. 1859--2001\grqq. Hartung-Gorre Verlag. Konstanz 2002.}

\item[]{Karl-Heinz Gräfe / Hans-Jürgen Töpfer: \glqq Ausgesondert und fast vergessen, KZ-Außenlager auf dem Territorium des heutigen Sachsen\grqq. DDP. Dresden 1996.}

\item[]{Robert Heinze: \glqq Ortschronik Rennersdorf\grqq. Unveröffentlicht. o.Z. }

\item[]{Rolf Helm: \glqq Anwalt des Volkes -- Erinnerungen\grqq. Dietz Verlag. Berlin 1978.}

\item[]{Raul Hilberg: \glqq Sonderzüge nach Auschwitz\grqq. Die amerikanische Originalausgabe wurde 1976 unter dem Titel \glqq The Role of the German Railroads in the Destruction of the Jews\grqq~veröffentlicht. Horst-Werner Dumjahn Verlag. Mainz 1981.}

\item[]{Raul Hilberg: \glqq Die Vernichtung der europäischen Juden\grqq. Band 2., 9. Auflage, Fischer-Taschenbuchverlag. Frankfurt a.M. 1999.}

\item[]{Siegfried Hittig: \glqq 750 Jahre Altbernsdorf auf dem Eigen. 1234--1984\grqq. Nowa Doba. Bautzen 1984.}

\item[]{Anna Hyndr\'akov\'a: \glqq Letter to my children\grqq. In: \glqq World without human dimension\grqq. State Jewish Museum Praque. Praque 1991.}

\item[]{Heinz Israel: \glqq Im Wandel der Zeit -- 1285--1985, Beiträge zur 700-Jahrfeier der Gemeinde Deutsch-Paulsdorf\grqq. Deutsch-Paulsdorf 1985}

\item[]{Hermann Kaienburg (Hrsg.): \glqq Wie konnte es soweit kommen?\grqq. In: \glqq Konzentrationslager und deutsche Wirtschaft\grqq. Leske + Budrich Verlag. Opladen 1996.}

\item[]{Alfred Konieczny: \glqq Frauen im Konzentrationslager Groß-Rosen in den Jahren 1944--1945\grqq. Państwowe Muzeum Gross-Rosen. Wa\l bzych 1998.} 

\item[]{Alfred Konieczny: \glqq Das KZ Groß-Rosen in Niederschlesien\grqq, S. 316, in: Ulrich Herbert, Karin Orth, Christoph Dieckmann (Hsg.): \glqq Die nationalsozialistischen Konzentrationslager in Entwicklung und Struktur, Band. 1\grqq, Wallstein Verlag, Göttingen 1998.}

\item[]{Jo\"el Kotek / Pierre Rigoulot: \glqq Das Jahrhundert der Lager - Gefangenschaft, Zwangsarbeit, Vernichtung\grqq. Propyläen Verlag. München 2001.}

\item[]{Ernst Kretzschmer: \glqq Görlitz unter dem Hakenkreuz\grqq. Edition 3. Städtische Kunstsammlung Görlitz. Görlitz 1982. }

\item[]{Hermann Langbein: \glqq Widerstand in den nationalsozialistischen Konzentrationslagern\grqq. Fischer Taschenbuchverlag. Frankfurt a.M. 1980.}

\item[]{Hannelore Lauerwald: \glqq Im fremden Land -- Kriegsgefangenenlager Stalag VIII A. Tatsachen - Briefe - Dokumente\grqq. Sächsische Landeszentrale für Politische Bildung (Hrsg.). Viadukt-Verlag. Görlitz 1997.}

\item[]{Henry Leide: \glqq NS-Verbrecher und Staatssicherheit\grqq. Vandenhoeck \& Rubrecht. Göttingen 2005. }

\item[]{Herbert Obenaus: \glqq Die Außenkommandos des Konzentrationslagers Neuengame\grqq. In: Hermann Kaienburg (Hrg.): \glqq Konzentrationslager und deutsche Wirtschaft\grqq. Leske und Budrich. Opladen 1996.}

\item[]{Karin Orth: \glqq Experten des Terrors. Die Konzentrationslager, SS und die Shoah\grqq. In: Gerhard Paul (Hrg.): \glqq Die Täter der Shoah\grqq. Wallstein Verlag. Göttingen. 2002.}

\item[]{Roland Otto: \glqq Die Verfolgung der Juden in Görlitz unter der faschistischen Diktatur 1933--1945\grqq. Stadtverwaltung Görlitz (Hrsg.). Görlitz 1990.}

\item[]{Pavla Placha und Andrea Rudorff: \glqq Reichenau (Rychnov u Jablonce nad Nison)\grqq. In: Wolf­gang Benz / Bar­bara Die­s­tel (Hrgs.): \glqq Der Ort des Terrors. Geschichte der nationalsozialistischen Konzentrationslager. Band 6: Natzweiler, Groß-Rosen, Stutthof\grqq S. 421-426.  Ver­lag C.H.Beck, Mün­chen 2008.}

\item[]{Peter Reicherl: \glqq Vergangenheitsbewältigung\grqq. C.H.Beck. München 2001.}

\item[]{Jakob Rosenbaum: \glqq Von Görlitz nach Tirol\grqq. In: \glqq Zeitschrift für Geschichte Nr. 5\grqq: Fun Lecten Churbn (from the last extermination). München. Mai 1947.}	

\item[]{Dieter Rostowski: \glqq Todesmärsche in der Oberlausitz\grqq. Selbstverlag. Kamenz 2003. }

\item[]{Dieter Rostowski / Marlies Röhle: \glqq Vom KZ-AL Niesky nach Brandhofen (Spohla). Warum ein Dorf bei Hoyerswerda 1945 geschichtsträchtig wurde.\grqq. Selbstverlag. Kamenz 2005.}

\item[]{Hans Rothfels / Theodor Eschenburg: \glqq Die wirtschaftlichen Unternehmungen der SS\grqq. Deutsche Verlags-Anstalt. Stuttgart 1963. }

\item[]{Andrea Rudorff: \glqq Das Lagersystem der Organisation Schmelt in Oberschlesien\grqq. In: \glqq Der Ort des Terrors. Geschichte der nationalsozialistischen Konzentrationslager. Band 9: Arbeitserziehungslager, Ghettos, Jungendschutzlager, Polizeihaftlager, Sonderlager, Zigeunerlager, Zwangsarbeitslager\grqq. Band. 9. C.H.Beck. München 2009. }

\item[]{Mike Schmeitzner: \glqq Der Fall Mutschmann -- Sachsens Gauleiter vor Stalins Tribunal\grqq. 2. Auflage 2011. Sax-Verlag. Beucha Markkleeberg 2011.}

\item[]{Franz Scholz: \glqq Görlitzer Tagebuch\grqq. Chronik einer Vertreibung 1945/1946. 2. Aufl. Ullstein Zeitgeschichte. Berlin 1988.}

\item[]{Jürgen Schröder (Hrg.): \glqq 50 Jahre WUMAG. 1948--1998.\grqq. WUMAG. Krefeld 1998. }

\item[]{Simon Schweitzer (Hrg.) / Milly Charon: \glqq Simons langer Weg\grqq. Büchergilde Gutenberg. Frankfurt a.M. 2002.}

\item[]{Wolfgang Sofsky: \glqq Die Ordnung des Terrors. Das Konzentrationslager\grqq. Fischer Verlag. Frankfurt a.M. 1993.}

\item[]{Isabell Sprenger: \glqq Gross Rosen. Ein Konzentrationslager in Schlesien\grqq. Böhlau Verlag. Köln 1996. }

\item[]{Dorota Sula und Andrea Rudorff: \glqq Zittau\grqq, in: Wolfgang Benz / Bar\-bara Die\-st\-el (Hrgs.): \glqq Der Ort des Terrors. Geschichte der nationalsozialistischen Konzentrationslager. Band 6: Natzweiler, Groß-Rosen, Stutthof\grqq S. 470-473. C.H.Beck. Mün\-chen 2008.}

\item[]{Jeffrey Sweet: \glqq The Action against Sol Schumann\grqq, Dramatists Play Service Inc. (Hrsg.) New York 2003.}

\item[]{Wolfgang Theurich: \glqq 160 Jahre Waggonbau in Görlitz -- 1849-2009\grqq. EK-Verlag. Freiburg 2009. }

\item[]{Karl-Heinz Thielke (Hrg.): \glqq Fall 5 -- Ausgewählte Dokumente des Flick-Prozesses\grqq. VEB Deutscher Verlag der Wissenschaften. Berlin 1965.}

\item[]{Johannes Tuchel: \glqq Inspektion der Konzentrationslager 1938--1945\grqq. Edition Hentrich. 1994.}

\item[]{Martin Weinman: \glqq Das nationalsozialistische Lagersystem (CCP)\grqq. 4. Auflage. Frankfurt a.M. 2001. }

\item[] {VEB Waggonbau Bautzen (Hrsg):\glqq Wag\-gon\-bauer pfle\-gen re\-vo\-lu\-tio\-näre Tra\-di\-tio\-nen. Aus der Ge\-schichte des KZ-Außenlagers in der Maschinen- und Wag\-gon\-fa\-brik vorm Busch Baut\-zen.\grqq. Bautzen, 1983}

\item[]{Henryk Vogler: \glqq Autoportret z Pamięci\grqq. Wydawnictwo Literackie. Kraków 1981.}

\item[]{Thomas Warkus: \glqq Kriegsgefangene und Fremdarbeiter im nationalsozialistischen Deutschland 1939--1945. Das Beispiel Görlitz.\grqq Diplomarbeit. Technische Universität Dresden, Philosophische Fakultät, Institut für Geschichte. 1995.}

\item[]{Adam W\l odek: \glqq Najcichszy Sztandar\grqq. Krakau, 1945.}

\item[]{Kurt Wolf: \glqq KZ-Außenlager Görlitz Biesnitzer Grund\grqq. Stadt Görlitz (Hrg.). Görlitz 2005.}

\item[]{Volker Zimmermann: \glqq NS-Täter vor Gericht\grqq. Justizministerium Nordrheinwestpfalen in Zusammenarbeit mit der Mahn- und Gedenkstätte der Landeshauptstadt Düsseldorf (Hrsg.). Düsseldorf 2001.}

\item[]{Roman Zg\l obicki: \glqq Obozy i cmentarze Wojenne w Zgorzelcu\grqq. Zgorzelec 1996.}


\end{description}
%\end{footnotesize}



%%%%%%%%%%%%%%%%%%%%%%%%%%%%%%%%%%%%%%%%%%%%%%
\begin{thebibliography}{Archivalien}

\begin{footnotesize}
\bibitem[LArchB]{a1}{Akten der Staatsanwaltschaft Berlin über die Vorermittlungen gegen Erich Rechenberg, 3 P (K) Js 5/71). 11 Bände. Landesarchiv Berlin, Eichborndamm 115--121, 13403 Berlin. (LArchB) B Rep. 058. }

\bibitem[ZHI]{a2} {Testimonies ehemaliger Häftlinge des Außenlager Görlitz und des Arbeitslagers der Organisation Schmelt. Zydoswskim Institycie Historycznym Warszawa, Tlomackie 3/5, 00-090 Warsaw, Polen.}

\bibitem[UArch]{a3} {Unterlagen über die Besitzverhältnisse des Gut Oberrennersdorf. Unitätsarchiv der Evangelischen Brüder-Unität, Zittauer Str. 24, 02747 Herrnhut.}

\bibitem[BStU]{a4}{Prozessakten im Fall Malitz -- Meinshausen. Bundesbeauftragte für die Unterlagen des Staatssicherheitsdienstes der ehemaligen Deutschen Demokratischen Republik, Otto Braun Str., Berlin.}
 
\bibitem[KArchLZ]{a6} {Protokoll der Gemeinderatssitzungen des Jahres 1945 Ren-nersdorf. Kreisarchiv Löbau-Zittau, Lisa Tetzner Str. 11, Zittau.}

\bibitem[StArchD]{a7} {WUMAG. Staatsarchiv Dresden, Archivstr. 1, 01069 Dresden.}
																					
\bibitem[PMGR]{a8} {Muzeum Gross Rosen. Wa\'lbrzych, Polen.}						

\bibitem[RAG]{a9} {Ratsarchiv Görlitz. Am Untermarkt 1, Görlitz.}

\bibitem[FVG]{a10} {Friedhofsverwaltung Görlitz.}

\bibitem[YV]{a11} {Yad Vashem Holocaust Gedenkstätte Israel.}

\bibitem[USHMM]{a12} {United States Holocaust Memorial Museum. Washington DC.}

\bibitem[JMP]{a13} {Jüdisches Museum Prag.}

\bibitem[VSI]{a14} {Visual Shoa Institute / University of Southern California.}

\end{footnotesize}
\end{thebibliography}


%%%%%%%%%%%%%%%%%%%%%%%%%%%%%%%%%%%
%% list of figures, maps and tables

% bugy ...listet auch karten
%\addcontentsline{toc}{section}{Abbildungsverzeichnis}%
%\listofmypics

\section*{Kartenverzeichnis}
\addcontentsline{toc}{section}{Kartenverzeichnis}%
\listofmymaps

%\addcontentsline{toc}{section}{Tabellenverzeichnis}%
%\listoftables

%%%%%%%%%%%%


{\printindex{p}{Personenverzeichnis}}

%%%%%%%%%%%%
{\printindex{o}{Ortsverzeichnis}}
%%%%%%%%%%%%


